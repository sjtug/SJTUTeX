% \iffalse meta-comment
%
% Copyright (C) SJTUG
%   2018--2023 Alexara Wu   <alexarawu@outlook.com>
%   2022--2023 Log Creative <logcreative@outlook.com>
%
% This work may be distributed and/or modified under the
% conditions of the LaTeX Project Public License, either
% version 1.3c of this license or (at your option) any later
% version. The latest version of this license is in:
%
%   http://www.latex-project.org/lppl.txt
%
% and version 1.3 or later is part of all distributions of
% LaTeX version 2005/12/01 or later.
%
% This work has the LPPL maintenance status `maintained'.
%
% The Current Maintainers of this work are Alexara Wu and Log Creative.
%
%<*internal>
\iffalse
%</internal>
%
%<*internal>
\fi
\begingroup
  \def\NameOfLaTeXe{LaTeX2e}
\expandafter\endgroup\ifx\NameOfLaTeXe\fmtname\else
\csname fi\endcsname
%</internal>
%
%<*install>
\input l3docstrip.tex
\keepsilent
\askforoverwritefalse

\preamble

    Copyright (C) SJTUG
      2018--2023 Alexara Wu   <alexarawu@outlook.com>
      2022--2023 Log Creative <logcreative@outlook.com>

    This work may be distributed and/or modified under the
    conditions of the LaTeX Project Public License, either
    version 1.3c of this license or (at your option) any later
    version. The latest version of this license is in:

      http://www.latex-project.org/lppl.txt

    and version 1.3 or later is part of all distributions of
    LaTeX version 2005/12/01 or later.

    This work has the LPPL maintenance status `maintained'.

    The Current Maintainers of this work are Alexara Wu and Log Creative.

\endpreamble

\generate{
  \usedir{tex/latex/sjtutex}
    \file{sjtuthesis.cls}               {\from{\jobname.dtx}{class,thesis}}
    \file{sjtureport.cls}               {\from{\jobname.dtx}{class,report}}
    \file{sjtuarticle.cls}              {\from{\jobname.dtx}{class,article}}
    \file{sjtu-lang-thesis-zh.def}      {\from{\jobname.dtx}{lang,thesis,zh}}
    \file{sjtu-lang-thesis-en.def}      {\from{\jobname.dtx}{lang,thesis,en}}
    \file{sjtu-lang-thesis-de.def}      {\from{\jobname.dtx}{lang,thesis,de}}
    \file{sjtu-lang-thesis-ja.def}      {\from{\jobname.dtx}{lang,thesis,ja}}
    \file{sjtu-lang-generic-zh.def}     {\from{\jobname.dtx}{lang,generic,zh}}
    \file{sjtu-lang-generic-en.def}     {\from{\jobname.dtx}{lang,generic,en}}
    \file{sjtu-lang-generic-de.def}     {\from{\jobname.dtx}{lang,generic,de}}
    \file{sjtu-lang-generic-ja.def}     {\from{\jobname.dtx}{lang,generic,ja}}
    \file{sjtu-name-thesis-zh.def}      {\from{\jobname.dtx}{name,thesis,zh}}
    \file{sjtu-name-thesis-en.def}      {\from{\jobname.dtx}{name,thesis,en}}
    \file{sjtu-name-thesis-de.def}      {\from{\jobname.dtx}{name,thesis,de}}
    \file{sjtu-name-thesis-ja.def}      {\from{\jobname.dtx}{name,thesis,ja}}
    \file{sjtu-name-generic-zh.def}     {\from{\jobname.dtx}{name,generic,zh}}
    \file{sjtu-name-generic-en.def}     {\from{\jobname.dtx}{name,generic,en}}
    \file{sjtu-name-generic-de.def}     {\from{\jobname.dtx}{name,generic,de}}
    \file{sjtu-name-generic-ja.def}     {\from{\jobname.dtx}{name,generic,ja}}
    \file{sjtu-scheme-zh.def}           {\from{\jobname.dtx}{scheme,zh}}
    \file{sjtu-scheme-en.def}           {\from{\jobname.dtx}{scheme,en}}
    \file{sjtu-scheme-de.def}           {\from{\jobname.dtx}{scheme,de}}
    \file{sjtu-scheme-ja.def}           {\from{\jobname.dtx}{scheme,ja}}
    \file{sjtu-text-font-newtx.def}     {\from{\jobname.dtx}{font,text,newtx}}
    \file{sjtu-text-font-newpx.def}     {\from{\jobname.dtx}{font,text,newpx}}
    \file{sjtu-text-font-lm.def}        {\from{\jobname.dtx}{font,text,lm}}
    \file{sjtu-text-font-libertinus.def}{\from{\jobname.dtx}{font,text,libertinus}}
    \file{sjtu-text-font-stixtwo.def}   {\from{\jobname.dtx}{font,text,stixtwo}}
    \file{sjtu-text-font-xits.def}      {\from{\jobname.dtx}{font,text,xits}}
    \file{sjtu-text-font-newcm.def}     {\from{\jobname.dtx}{font,text,newcm}}
    \file{sjtu-text-font-cambria.def}   {\from{\jobname.dtx}{font,text,cambria}}
    \file{sjtu-text-font-times.def}     {\from{\jobname.dtx}{font,text,times}}
    \file{sjtu-math-font-newtx.def}     {\from{\jobname.dtx}{font,math,newtx}}
    \file{sjtu-math-font-newpx.def}     {\from{\jobname.dtx}{font,math,newpx}}
    \file{sjtu-math-font-lm.def}        {\from{\jobname.dtx}{font,math,lm}}
    \file{sjtu-math-font-libertinus.def}{\from{\jobname.dtx}{font,math,libertinus}}
    \file{sjtu-math-font-stixtwo.def}   {\from{\jobname.dtx}{font,math,stixtwo}}
    \file{sjtu-math-font-xits.def}      {\from{\jobname.dtx}{font,math,xits}}
    \file{sjtu-math-font-newcm.def}     {\from{\jobname.dtx}{font,math,newcm}}
    \file{sjtu-math-font-cambria.def}   {\from{\jobname.dtx}{font,math,cambria}}
    \file{sjtu-math-font-times.def}     {\from{\jobname.dtx}{font,math,times}}
    \file{sjtu-cjk-font-windows.def}    {\from{\jobname.dtx}{font,cjk,windows,und}}
    \file{sjtu-cjk-font-windows-zh.def} {\from{\jobname.dtx}{font,cjk,windows,zh}}
    \file{sjtu-cjk-font-windows-ja.def} {\from{\jobname.dtx}{font,cjk,windows,ja}}
    \file{sjtu-cjk-font-mac.def}        {\from{\jobname.dtx}{font,cjk,mac,und}}
    \file{sjtu-cjk-font-mac-zh.def}     {\from{\jobname.dtx}{font,cjk,mac,zh}}
    \file{sjtu-cjk-font-mac-ja.def}     {\from{\jobname.dtx}{font,cjk,mac,ja}}
    \file{sjtu-cjk-font-ubuntu.def}     {\from{\jobname.dtx}{font,cjk,ubuntu,und}}
    \file{sjtu-cjk-font-ubuntu-zh.def}  {\from{\jobname.dtx}{font,cjk,ubuntu,zh}}
    \file{sjtu-cjk-font-ubuntu-ja.def}  {\from{\jobname.dtx}{font,cjk,ubuntu,ja}}
    \file{sjtu-cjk-font-adobe.def}      {\from{\jobname.dtx}{font,cjk,adobe,und}}
    \file{sjtu-cjk-font-adobe-zh.def}   {\from{\jobname.dtx}{font,cjk,adobe,zh}}
    \file{sjtu-cjk-font-adobe-ja.def}   {\from{\jobname.dtx}{font,cjk,adobe,ja}}
    \file{sjtu-cjk-font-fandol.def}     {\from{\jobname.dtx}{font,cjk,fandol,und}}
    \file{sjtu-cjk-font-fandol-zh.def}  {\from{\jobname.dtx}{font,cjk,fandol,zh}}
    \file{sjtu-cjk-font-fandol-ja.def}  {\from{\jobname.dtx}{font,cjk,fandol,ja}}
    \file{sjtu-cjk-font-founder.def}    {\from{\jobname.dtx}{font,cjk,founder,und}}
    \file{sjtu-cjk-font-founder-zh.def} {\from{\jobname.dtx}{font,cjk,founder,zh}}
    \file{sjtu-cjk-font-founder-ja.def} {\from{\jobname.dtx}{font,cjk,founder,ja}}
%</install>
%<*internal>
  \usedir{source/latex/sjtutex}
    \file{\jobname.ins}                 {\from{\jobname.dtx}{install}}
%</internal>
%<*install>
}

\obeyspaces
\Msg{*************************************************************}
\Msg{*                                                           *}
\Msg{* To finish the installation you have to move the following *}
\Msg{* files into a directory searched by TeX:                   *}
\Msg{*                                                           *}
\Msg{* The recommended directory is TDS:tex/latex/sjtutex        *}
\Msg{*                                                           *}
\Msg{*     sjtuthesis.cls                                        *}
\Msg{*     sjtureport.cls                                        *}
\Msg{*     sjtuarticle.cls                                       *}
\Msg{*     sjtu-lang-generic-de.def                              *}
\Msg{*     sjtu-lang-generic-en.def                              *}
\Msg{*     sjtu-lang-generic-ja.def                              *}
\Msg{*     sjtu-lang-generic-zh.def                              *}
\Msg{*     sjtu-lang-thesis-de.def                               *}
\Msg{*     sjtu-lang-thesis-en.def                               *}
\Msg{*     sjtu-lang-thesis-ja.def                               *}
\Msg{*     sjtu-lang-thesis-zh.def                               *}
\Msg{*     sjtu-name-generic-de.def                              *}
\Msg{*     sjtu-name-generic-en.def                              *}
\Msg{*     sjtu-name-generic-ja.def                              *}
\Msg{*     sjtu-name-generic-zh.def                              *}
\Msg{*     sjtu-name-thesis-de.def                               *}
\Msg{*     sjtu-name-thesis-en.def                               *}
\Msg{*     sjtu-name-thesis-ja.def                               *}
\Msg{*     sjtu-name-thesis-zh.def                               *}
\Msg{*     sjtu-scheme-de.def                                    *}
\Msg{*     sjtu-scheme-en.def                                    *}
\Msg{*     sjtu-scheme-ja.def                                    *}
\Msg{*     sjtu-scheme-zh.def                                    *}
\Msg{*     sjtu-text-font-cambria.def                            *}
\Msg{*     sjtu-text-font-libertinus.def                         *}
\Msg{*     sjtu-text-font-lm.def                                 *}
\Msg{*     sjtu-text-font-newcm.def                              *}
\Msg{*     sjtu-text-font-newpx.def                              *}
\Msg{*     sjtu-text-font-newtx.def                              *}
\Msg{*     sjtu-text-font-stixtwo.def                            *}
\Msg{*     sjtu-text-font-times.def                              *}
\Msg{*     sjtu-text-font-xits.def                               *}
\Msg{*     sjtu-math-font-cambria.def                            *}
\Msg{*     sjtu-math-font-libertinus.def                         *}
\Msg{*     sjtu-math-font-lm.def                                 *}
\Msg{*     sjtu-math-font-newcm.def                              *}
\Msg{*     sjtu-math-font-newpx.def                              *}
\Msg{*     sjtu-math-font-newtx.def                              *}
\Msg{*     sjtu-math-font-stixtwo.def                            *}
\Msg{*     sjtu-math-font-times.def                              *}
\Msg{*     sjtu-math-font-xits.def                               *}
\Msg{*     sjtu-cjk-font-adobe-ja.def                            *}
\Msg{*     sjtu-cjk-font-adobe-zh.def                            *}
\Msg{*     sjtu-cjk-font-adobe.def                               *}
\Msg{*     sjtu-cjk-font-fandol-ja.def                           *}
\Msg{*     sjtu-cjk-font-fandol-zh.def                           *}
\Msg{*     sjtu-cjk-font-fandol.def                              *}
\Msg{*     sjtu-cjk-font-founder-ja.def                          *}
\Msg{*     sjtu-cjk-font-founder-zh.def                          *}
\Msg{*     sjtu-cjk-font-founder.def                             *}
\Msg{*     sjtu-cjk-font-mac-ja.def                              *}
\Msg{*     sjtu-cjk-font-mac-zh.def                              *}
\Msg{*     sjtu-cjk-font-mac.def                                 *}
\Msg{*     sjtu-cjk-font-ubuntu-ja.def                           *}
\Msg{*     sjtu-cjk-font-ubuntu-zh.def                           *}
\Msg{*     sjtu-cjk-font-ubuntu.def                              *}
\Msg{*     sjtu-cjk-font-windows-ja.def                          *}
\Msg{*     sjtu-cjk-font-windows-zh.def                          *}
\Msg{*     sjtu-cjk-font-windows.def                             *}
\Msg{*     sjtu-vi-badge-black.pdf                               *}
\Msg{*     sjtu-vi-badge-blue.pdf                                *}
\Msg{*     sjtu-vi-badge-red.pdf                                 *}
\Msg{*     sjtu-vi-logo-black.pdf                                *}
\Msg{*     sjtu-vi-logo-blue.pdf                                 *}
\Msg{*     sjtu-vi-logo-red.pdf                                  *}
\Msg{*     sjtu-vi-name-black.pdf                                *}
\Msg{*     sjtu-vi-name-blue.pdf                                 *}
\Msg{*     sjtu-vi-name-red.pdf                                  *}
\Msg{*                                                           *}
\Msg{* To produce the documentation, run the file sjtutex.dtx    *}
\Msg{* through XeLaTeX.                                          *}
\Msg{*                                                           *}
\Msg{* Happy TeXing!                                             *}
\Msg{*                                                           *}
\Msg{*************************************************************}

\endbatchfile
%</install>
%
%<*internal>
\fi
%</internal>
%
%<class>\NeedsTeXFormat{LaTeX2e}[2020/10/01]
%<*!(driver|install)>
%<+!driver>\GetIdInfo $Id: sjtutex.dtx 2.0.3 2023-09-25 19:50:00Z Alexara Wu <alexarawu@outlook.com> $
%<class&thesis>  {Thesis template for Shanghai Jiao Tong University}
%<class&thesis>\ProvidesExplClass{sjtuthesis}
%<class&report>  {Report template for Shanghai Jiao Tong University}
%<class&report>\ProvidesExplClass{sjtureport}
%<class&article>  {Article template for Shanghai Jiao Tong University}
%<class&article>\ProvidesExplClass{sjtuarticle}
%<lang&thesis&zh>  {Chinese specific definition for thesis (SJTUTeX)}
%<lang&thesis&zh>\ProvidesExplFile{sjtu-lang-thesis-zh.def}
%<lang&thesis&en>  {English specific definition for thesis (SJTUTeX)}
%<lang&thesis&en>\ProvidesExplFile{sjtu-lang-thesis-en.def}
%<lang&thesis&de>  {German specific definition for thesis (SJTUTeX)}
%<lang&thesis&de>\ProvidesExplFile{sjtu-lang-thesis-de.def}
%<lang&thesis&ja>  {Japanese specific definition for thesis (SJTUTeX)}
%<lang&thesis&ja>\ProvidesExplFile{sjtu-lang-thesis-ja.def}
%<lang&generic&zh>  {Chinese specific definition for generic (SJTUTeX)}
%<lang&generic&zh>\ProvidesExplFile{sjtu-lang-generic-zh.def}
%<lang&generic&en>  {English specific definition for generic (SJTUTeX)}
%<lang&generic&en>\ProvidesExplFile{sjtu-lang-generic-en.def}
%<lang&generic&de>  {German specific definition for generic (SJTUTeX)}
%<lang&generic&de>\ProvidesExplFile{sjtu-lang-generic-de.def}
%<lang&generic&ja>  {Japanese specific definition for generic (SJTUTeX)}
%<lang&generic&ja>\ProvidesExplFile{sjtu-lang-generic-ja.def}
%<name&thesis&zh>  {Chinese name config for thesis (SJTUTeX)}
%<name&thesis&zh>\ProvidesExplFile{sjtu-name-thesis-zh.def}
%<name&thesis&en>  {English name config for thesis (SJTUTeX)}
%<name&thesis&en>\ProvidesExplFile{sjtu-name-thesis-en.def}
%<name&thesis&de>  {German name config for thesis (SJTUTeX)}
%<name&thesis&de>\ProvidesExplFile{sjtu-name-thesis-de.def}
%<name&thesis&ja>  {Japanese name config for thesis (SJTUTeX)}
%<name&thesis&ja>\ProvidesExplFile{sjtu-name-thesis-ja.def}
%<name&generic&zh>  {Chinese name config for generic (SJTUTeX)}
%<name&generic&zh>\ProvidesExplFile{sjtu-name-generic-zh.def}
%<name&generic&en>  {English name config for generic (SJTUTeX)}
%<name&generic&en>\ProvidesExplFile{sjtu-name-generic-en.def}
%<name&generic&de>  {German name config for generic (SJTUTeX)}
%<name&generic&de>\ProvidesExplFile{sjtu-name-generic-de.def}
%<name&generic&ja>  {Japanese name config for generic (SJTUTeX)}
%<name&generic&ja>\ProvidesExplFile{sjtu-name-generic-ja.def}
%<scheme&zh>  {Chinese scheme (SJTUTeX)}
%<scheme&zh>\ProvidesExplFile{sjtu-scheme-zh.def}
%<scheme&en>  {English scheme (SJTUTeX)}
%<scheme&en>\ProvidesExplFile{sjtu-scheme-en.def}
%<scheme&de>  {German scheme (SJTUTeX)}
%<scheme&de>\ProvidesExplFile{sjtu-scheme-de.def}
%<scheme&ja>  {Japanese scheme (SJTUTeX)}
%<scheme&ja>\ProvidesExplFile{sjtu-scheme-ja.def}
%<font&text&newtx>  {New TX text fonts definition (SJTUTeX)}
%<font&text&newtx>\ProvidesExplFile{sjtu-text-font-newtx.def}
%<font&text&newpx>  {New PX text fonts definition (SJTUTeX)}
%<font&text&newpx>\ProvidesExplFile{sjtu-text-font-newpx.def}
%<font&text&stixtwo>  {STIX Two text fonts definition (SJTUTeX)}
%<font&text&stixtwo>\ProvidesExplFile{sjtu-text-font-stixtwo.def}
%<font&text&xits>  {XITS text fonts definition (SJTUTeX)}
%<font&text&xits>\ProvidesExplFile{sjtu-text-font-xits.def}
%<font&text&lm>  {Latin Modern text fonts definition (SJTUTeX)}
%<font&text&lm>\ProvidesExplFile{sjtu-text-font-lm.def}
%<font&text&libertinus>  {Libertinus text fonts definition (SJTUTeX)}
%<font&text&libertinus>\ProvidesExplFile{sjtu-text-font-libertinus.def}
%<font&text&newcm>  {New Computer Modern text fonts definition (SJTUTeX)}
%<font&text&newcm>\ProvidesExplFile{sjtu-text-font-newcm.def}
%<font&text&cambria>  {Cambria text fonts definition (SJTUTeX)}
%<font&text&cambria>\ProvidesExplFile{sjtu-text-font-cambria.def}
%<font&text&times>  {Times text fonts definition (SJTUTeX)}
%<font&text&times>\ProvidesExplFile{sjtu-text-font-times.def}
%<font&math&newtx>  {New TX math fonts definition (SJTUTeX)}
%<font&math&newtx>\ProvidesExplFile{sjtu-math-font-newtx.def}
%<font&math&newpx>  {New PX math fonts definition (SJTUTeX)}
%<font&math&newpx>\ProvidesExplFile{sjtu-math-font-newpx.def}
%<font&math&stixtwo>  {STIX Two math fonts definition (SJTUTeX)}
%<font&math&stixtwo>\ProvidesExplFile{sjtu-math-font-stixtwo.def}
%<font&math&xits>  {XITS math fonts definition (SJTUTeX)}
%<font&math&xits>\ProvidesExplFile{sjtu-math-font-xits.def}
%<font&math&lm>  {Latin Modern math fonts definition (SJTUTeX)}
%<font&math&lm>\ProvidesExplFile{sjtu-math-font-lm.def}
%<font&math&libertinus>  {Libertinus math fonts definition (SJTUTeX)}
%<font&math&libertinus>\ProvidesExplFile{sjtu-math-font-libertinus.def}
%<font&math&newcm>  {New Computer Modern math fonts definition (SJTUTeX)}
%<font&math&newcm>\ProvidesExplFile{sjtu-math-font-newcm.def}
%<font&math&cambria>  {Cambria math fonts definition (SJTUTeX)}
%<font&math&cambria>\ProvidesExplFile{sjtu-math-font-cambria.def}
%<font&math&times>  {Times math fonts definition (SJTUTeX)}
%<font&math&times>\ProvidesExplFile{sjtu-math-font-times.def}
%<font&cjk&windows&und>  {Windows CJK fonts definition (SJTUTeX)}
%<font&cjk&windows&und>\ProvidesExplFile{sjtu-cjk-font-windows.def}
%<font&cjk&windows&zh>  {Windows CJK fonts definition for Chinese (SJTUTeX)}
%<font&cjk&windows&zh>\ProvidesExplFile{sjtu-cjk-font-windows-zh.def}
%<font&cjk&windows&ja>  {Windows CJK fonts definition for Japanese (SJTUTeX)}
%<font&cjk&windows&ja>\ProvidesExplFile{sjtu-cjk-font-windows-ja.def}
%<font&cjk&mac&und>  {macOS CJK fonts definition (SJTUTeX)}
%<font&cjk&mac&und>\ProvidesExplFile{sjtu-cjk-font-mac.def}
%<font&cjk&mac&zh>  {macOS CJK fonts definition for Chinese (SJTUTeX)}
%<font&cjk&mac&zh>\ProvidesExplFile{sjtu-cjk-font-mac-zh.def}
%<font&cjk&mac&ja>  {macOS CJK fonts definition for Japanese (SJTUTeX)}
%<font&cjk&mac&ja>\ProvidesExplFile{sjtu-cjk-font-mac-ja.def}
%<font&cjk&ubuntu&und>  {Ubuntu CJK fonts definition (SJTUTeX)}
%<font&cjk&ubuntu&und>\ProvidesExplFile{sjtu-cjk-font-ubuntu.def}
%<font&cjk&ubuntu&zh>  {Ubuntu CJK fonts definition for Chinese (SJTUTeX)}
%<font&cjk&ubuntu&zh>\ProvidesExplFile{sjtu-cjk-font-ubuntu-zh.def}
%<font&cjk&ubuntu&ja>  {Ubuntu CJK fonts definition for Japanese (SJTUTeX)}
%<font&cjk&ubuntu&ja>\ProvidesExplFile{sjtu-cjk-font-ubuntu-ja.def}
%<font&cjk&adobe&und>  {Adobe CJK fonts definition (SJTUTeX)}
%<font&cjk&adobe&und>\ProvidesExplFile{sjtu-cjk-font-adobe.def}
%<font&cjk&adobe&zh>  {Adobe CJK fonts definition for Chinese (SJTUTeX)}
%<font&cjk&adobe&zh>\ProvidesExplFile{sjtu-cjk-font-adobe-zh.def}
%<font&cjk&adobe&ja>  {Adobe CJK fonts definition for Japanese (SJTUTeX)}
%<font&cjk&adobe&ja>\ProvidesExplFile{sjtu-cjk-font-adobe-ja.def}
%<font&cjk&fandol&und>  {Fandol CJK fonts definition (SJTUTeX)}
%<font&cjk&fandol&und>\ProvidesExplFile{sjtu-cjk-font-fandol.def}
%<font&cjk&fandol&zh>  {Fandol CJK fonts definition for Chinese (SJTUTeX)}
%<font&cjk&fandol&zh>\ProvidesExplFile{sjtu-cjk-font-fandol-zh.def}
%<font&cjk&fandol&ja>  {Fandol CJK fonts definition for Japanese (SJTUTeX)}
%<font&cjk&fandol&ja>\ProvidesExplFile{sjtu-cjk-font-fandol-ja.def}
%<font&cjk&founder&und>  {Founder CJK fonts definition (SJTUTeX)}
%<font&cjk&founder&und>\ProvidesExplFile{sjtu-cjk-font-founder.def}
%<font&cjk&founder&zh>  {Founder CJK fonts definition for Chinese (SJTUTeX)}
%<font&cjk&founder&zh>\ProvidesExplFile{sjtu-cjk-font-founder-zh.def}
%<font&cjk&founder&ja>  {Founder CJK fonts definition for Japanese (SJTUTeX)}
%<font&cjk&founder&ja>\ProvidesExplFile{sjtu-cjk-font-founder-ja.def}
%<!driver>  {\ExplFileDate}{\ExplFileVersion}{\ExplFileDescription}
%</!(driver|install)>
%
%<*driver>
\PassOptionsToPackage{fontset=ubuntu}{ctex}
\documentclass{ctxdoc}
\newcommand{\sjtutex}{SJTU\TeX}
\usepackage{graphicx}
\usepackage{multirow}
\usepackage{float}
\usepackage{longtable}
\usepackage{listings}
\AtEndOfClass{\sloppy}
\definecolor{sjtu-red}{RGB}{200,22,30}
\definecolor{sjtu-dark-red}{RGB}{167,32,56}
\definecolor{sjtu-amber}{RGB}{253,208,0}
\definecolor{sjtu-orange}{RGB}{240,131,0}
\definecolor{sjtu-blue}{RGB}{0,134,209}
\definecolor{sjtu-dark-blue}{RGB}{0,64,152}
\definecolor{sjtu-green}{RGB}{51,141,39}
\definecolor{sjtu-dark-green}{RGB}{0,81,78}
\lstdefinestyle{style@base}{^^A
  basewidth         = 0.5 em,
  gobble            = 2,
  lineskip          = 2 pt,
  frame             = l,
  framerule         = 1.2 pt,
  framesep          = 0 pt,
  escapeinside      = {(*}{*)},
  basicstyle        = \small\ttfamily,
  xleftmargin       = 0.5 em,
  xrightmargin      = 0.5 em,
  framexleftmargin  = 0.5 em,
  framexrightmargin = 0.5 em
}
\lstdefinestyle{style@shell}{^^A
  style             = style@base,
  rulecolor         = \color{sjtu-dark-green},
  backgroundcolor   = \color{sjtu-dark-green!2},
  language          = bash,
  alsoletter        = {-},
  keywordstyle      = \color{sjtu-green},
  emphstyle         = \color{sjtu-amber}
}
\lstdefinestyle{style@latex}{^^A
  style             = style@base,
  rulecolor         = \color{sjtu-dark-blue},
  backgroundcolor   = \color{sjtu-dark-blue!2},
  language          = [LaTeX]TeX,
  alsoletter        = {*, -},
  keywordstyle      = \color{sjtu-dark-red!80},
  texcsstyle        = *\color{sjtu-blue},
  emphstyle         = [1]\color{sjtu-orange},
  emphstyle         = [2]\color{sjtu-green},
  emphstyle         = [3]\color{sjtu-dark-blue!80}
}
\lstnewenvironment{shell}[1][]{^^A
  \lstset{style=style@shell, #1}}{}
\lstnewenvironment{latex}[1][]{^^A
  \lstset{style=style@latex, #1}}{}
\newcommand\TNA[1]{\textcolor{sjtu-blue}{\tn{#1}}}
\newcommand\TNB[1]{\textcolor{sjtu-dark-red!80}{\tn{#1}}}
\newcommand\OPT[1]{\textcolor{sjtu-dark-blue!80}{#1}}
\newcommand\OPS[1]{\textcolor{sjtu-green}{#1}}
\newcommand\note[1]{{^^A
  \color{sjtu-dark-green}{\noindent\bfseries 说明:}~\emph{#1}}}
\hypersetup{^^A
  bookmarksnumbered = true,
  bookmarksopen     = true,
}
\makeatletter
\pdfstringdefDisableCommands{^^A
  \let\color\@gobble}
\makeatother
\begin{document}
  \DocInput{\jobname.dtx}
  \IndexLayout
  \PrintChanges
  \PrintIndex
\end{document}
%</driver>
% \fi
%
% \changes{v2.0}{2021/09/10}{应用 \LaTeXiii{} 重构代码。}
% \changes{v2.0}{2022/12/27}{新增文档类 \cls{sjtuarticle} 和 \cls{sjtureport}。}
%
% \GetFileInfo{\jobname.dtx}
%
% \title{\bfseries\color{sjtu-red}\sjtutex{}: 上海交通大学文档类集}
% \author{SJTUG}
% \date{2023/9/25\quad v2.0.3}
%
% \hypersetup{
%   pdfsubject  = {上海交通大学文档类集使用说明},
%   pdfkeywords = {模板; 上海交通大学; 使用说明},
% }
%
% \maketitle
%
% \begin{abstract}
% 此宏包旨在建立一个简单易用的上海交通大学文档类集,包括学位论文文档类
% \cls{sjtuthesis} 以及普通文档类 \cls{sjtuarticle} 和 \cls{sjtureport}。
% \end{abstract}
%
% \def\abstractname{免责声明}
% \begin{abstract}
% \noindent
% \begin{enumerate}
% \item 本模板的发布遵守 \href{https://www.latex-project.org/lppl/lppl-1-3c.txt}
%   {\LaTeX{} Project Public License (1.3c)},使用前请认真阅读协议内容。
% \item 学位论文模板根据 \href{https://www.gs.sjtu.edu.cn/post/detail/Z3MxNDc=}
%   {《上海交通大学博士、硕士学位论文撰写指南》} ,并参考
%   \href{https://bysj.jwc.sjtu.edu.cn/ShowNews.aspx?newsno=Yr2wg3ibJJhIZh0A52amXw....}
%   {《上海交通大学本科生毕业设计(论文)撰写规范》}
%   编写而成。旨在供上海交通大学准毕业生撰
%   写学位论文使用。
% \item 学位论文模板仅为撰写指南的参考实现,不保证审查老师不提意见。任何由于使用本模板
%   而引起的论文格式审查问题均与本模板作者无关。
% \item 任何个人或组织以本模板为基础进行修改、扩展而生成的新的专用模板,请严格遵
%   守 \LaTeX{} Project Public License 协议。由于违犯协议而引起的任何纠纷
%   争端均与本模板作者无关。
% \end{enumerate}
% \end{abstract}
%
% \tableofcontents
%
% \begin{documentation}
%
% \section{介绍}
%
% 最早的一版 \LaTeX{} 学位论文模板由一位热心的物理系同学制作,中文字符处理采用了当时最为流行的
% CJK--\LaTeX{} 方案。在此基础上,weijianwen 根据交大研究生院对学位论文的要求,完成
% 了一份基本可用的交大 \LaTeX{} 学位论文模板。由于 CJK--\LaTeX{} 方案不易使用,
% weijianwen 与 William Wang 开始着手把模板向 \XeTeX{} 引擎移植。之后 weijianwen
% 又断断续续做了一些完善模板的工作,在原有硕士学位论文模板的基础上完成了交大学士
% 和博士学位论文模板。
%
% 2012 年 5 月模板开始在 GitHub^^A
% \footnote{\url{https://github.com/weijianwen/SJTUThesis},项目转移后该链接已重定向。}^^A
% 上管理和更新,2018 年 1 月项目转移至 SJTUG 名
% 下。2019 年 6 月 Alexara Wu 重构了整个宏包的代码,并使用 Doc\TeX{} 文档和
% \textsc{DocStrip} 工具进行代码的管理,升级版本号为 1.0。2022 年 11 月,论文模板
% 改版后,使用 \LaTeXiii{} 重构了代码,添加 \cls{sjtureport} 和 \cls{sjtuarticle}
% 文档类,升级版本号为 2.0。
%
% 现在,\sjtutex{} 代码在 GitHub^^A
% \footnote{\url{https://github.com/sjtug/SJTUTeX}}^^A
% 上维护。
% 原 \href{https://github.com/sjtug/SJTUThesis}{SJTUThesis} 仓库则作为学位论文示例文档,提供开箱即用的模板。
% 学位论文模板用户可以在 \href{https://github.com/sjtug/SJTUThesis/discussions}{Discussions}
% 上提问使用问题,也可以在 \href{https://github.com/sjtug/SJTUThesis/issues}{Issues}
% 中进行 Bug 反馈与新功能提案。如果需要对文档类代码进行修改,欢迎前往
% \href{https://github.com/sjtug/SJTUTeX}{\sjtutex} 仓库进行 Pull Request。
% \sjtutex{} 模板的许多实现细节离不开
% \href{https://github.com/sjtug/SJTUThesis/graphs/contributors}
% {热心同学们} 的贡献,在此感谢所有为模板贡献过代码的同学们,以及所有测试和使用
% 模板的各位同学!
%
% \note{模板的作用在于减少论文写作过程中格式调整的时间,前提是遵守模板的用法,否
%   则即便用了 \sjtutex{} 也难以保证输出的论文符合学校规范。}
%
% \section{简明教程}
%
% \subsection{安装 \TeX{} 发行版}
%
% 因为 \LaTeXiii{} 和相关宏包在不断更新,所以推荐使用最新的 \TeX{} 发行版。
% \sjtutex{} 支持主流的 \TeX{} 发行版,包括 \TeX{} Live、\hologo{MiKTeX}、
% Mac\TeX{},但不支持 \CTeX{} 套装。安装方法具体可以参考
% \href{https://github.com/sjtug/SJTUThesis/wiki/TeX-发行版及其安装}
% {Wiki 页面《\TeX{} 发行版及其安装》}。
%
% \sjtutex{} 最低支持至 2021 年发行的 \TeX{} 发行版,\pkg{ctex} 宏包版本应当在
% v2.5 及以上。版本过低将无法正常编译。
%
% \subsection{文件组成}
%
% 表~\ref{tab:files} 列出了 \sjtutex{} 的主要文件及其功能介绍。
%
% \begin{table}[!hbt]
%   \centering
%   \caption{模板的文件组成}
%   \label{tab:files}
%   \begin{tabular}{l|l|p{4cm}}
%     \toprule
%     \strong{类别} & \strong{文件}               & \strong{说明}  \\
%     \midrule
%     文档类        & \file{sjtuthesis.cls}       & 学位论文文档类 \\
%                   & \file{sjtureport.cls}       & 报告文档类     \\
%                   & \file{sjtuarticle.cls}      & 文稿文档类     \\
%                   & \file{sjtu-name-*.def}      & 文档类名称配置 \\
%                   & \file{sjtu-lang-*.def}      & 文档类语言配置 \\
%                   & \file{sjtu-scheme-*.def}    & 文档类语言方案 \\
%     \midrule
%     字体配置文件  & \file{sjtu-text-font-*.def} & 西文字体配置   \\
%                   & \file{sjtu-math-font-*.def} & 数学字体配置   \\
%                   & \file{sjtu-cjk-font-*.def}  & CJK 字体配置   \\
%     \midrule
%     视觉形象系统  & \file{sjtu-vi-logo-*.pdf}   & 校标图片       \\
%                   & \file{sjtu-vi-badge-*.pdf}  & 校徽图片       \\
%                   & \file{sjtu-vi-name-*.pdf}   & 校名图片       \\
%
%     \bottomrule
%   \end{tabular}
% \end{table}
%
% \note{\href{https://vi.sjtu.edu.cn}{交大视觉形象系统}
%   的相关图像资源版权归上海交通大学所有。}
%
% \subsection{使用文档类}
%
% 共有 3 种文档类,\cls{sjtuthesis} 用于学位论文的排版,\cls{sjtureport} 用于课程大报告的排版,
% \cls{sjtuarticle} 用于课程小论文的排版。
%
% 将文档保存在下载好的 \sjtutex{} 模板根目录下,文件以 \texttt{.tex} 后缀结尾。
% 注意在使用文档类时,需要将涉及到的所有源文件使用 UTF-8 编码保存。对于不同的文档类,使用方法
% 略有不同,这里给出这三种文档类的最小使用示例。
%
% 下面这份 \TeX{} 文档展示了 \cls{sjtuthesis} 文档类的基本用法,一般需要指定中英文名称。
% 该文档将包含中英文封面,页眉为文档主题及章节名称。
% \begin{latex}[moretexcs={\sjtusetup,\maketitle,\tableofcontents*,\chapter},
%   morekeywords={\frontmatter,\mainmatter}, emph={[1]sjtuthesis,document},
%   emph={[2]info},emph={[3]type,zh,en,title,author}]
% \documentclass[type=master]{sjtuthesis}
% \sjtusetup{
%   info = {
%     zh/title  = {上海交通大学学位论文模板示例文档},
%     en/title  = {A Sample Document for SJTU Thesis Template},
%     zh/author = {某某},
%     en/author = {Mo Mo},
%   }
% }
% \begin{document}
%   \maketitle
%   \frontmatter
%   \tableofcontents*
%   \mainmatter
%   \chapter{欢迎}
%   \section{欢迎使用 SJTUThesis}
%   你好,\LaTeX{}!
% \end{document}
% \end{latex}
%
% 下面这份 \TeX{} 文档展示了 \cls{sjtureport} 文档类的基本用法,建议使用标准命令定义中文名称。
% 该文档将包含标题页,页眉为校标图片、文档主题及章节名称。
% \begin{latex}[moretexcs={\subject,\keywords,\maketitle,\chapter},
%   emph={[1]sjtureport,document}]
% \documentclass{sjtureport}
% \title{上海交通大学报告模板示例文档}
% \author{某某}
% \subject{XX期末课程论文}
% \keywords{上海交大, 饮水思源, 爱国荣校}
% \begin{document}
%   \maketitle
%   \chapter{欢迎}
%   \section{欢迎使用 SJTUReport}
%   你好,\LaTeX{}!
% \end{document}
% \end{latex}
%
% 下面这份 \TeX{} 文档展示了 \cls{sjtuarticle} 文档类的基本用法,建议使用标准命令定义中文名称,
% 不能够使用 \tn{chapter} 这一级。该文档包含标题栏,页眉为校标图片、文档主题及章节名称。
% \begin{latex}[moretexcs={\maketitle}, emph={[1]sjtuarticle,document}]
% \documentclass{sjtuarticle}
% \title{示例文档}
% \author{某某}
% \begin{document}
%   \maketitle
%   \section{欢迎使用 SJTUArticle}
%   你好,\LaTeX{}!
% \end{document}
% \end{latex}
%
% \subsection{编译文档}
%
% 文档类推荐使用 \XeLaTeX{} 或 \LuaLaTeX{}编译,同时也支持 \pdfTeX{} 引擎。为了
% 生成正确的目录、脚注以及交叉引用,至少需要连续编译两次。
%
% 在实际使用中,一般推荐使用自动生成工具 \pkg{latexmk} 编译文档。\pkg{latexmk}
% 命令可以自动进行多步编译,直到交叉引用都被解决。
%
% 假设您的 \TeX{} 源文件名为 \file{main.tex},
% 可在命令行中执行如下命令使用 \XeLaTeX{} 编译文档
% \begin{shell}[morekeywords={latexmk},emph={-xelatex}]
% latexmk -xelatex main
% \end{shell}
% 也可通过修改 \file{latexmkrc} 配置文件来控制 \pkg{latexmk} 的行为,
% 具体可以参考 \pkg{latexmk} 文档。
%
% \section{文档配置}
%
% 本模板中的选项、命令或环境可以分为以下三类:
% \begin{itemize}
%   \item 名字后面带有 \rexptarget\rexpstar{} 的,表示只能在 \cls{sjtuthesis} 文档类中使用;
%   \item 名字后面带有 \exptarget\expstar{} 的,表示只能在 \cls{sjtureport} 和 \cls{sjtuarticle} 文档类中使用;
%   \item 名字后面不带有特殊符号的,一般表示在 \cls{sjtuthesis}、\cls{sjtureport} 和 \cls{sjtuarticle} 文档类中都可以使用,特殊情况另作说明。
% \end{itemize}
%
% \subsection{文档类选项}
%
% 本节所指“文档类选项”是指需要在引入文档类的时候指定的选项:
%
% \begin{latex}[emph={[1]sjtuthesis}]
% \documentclass(*\oarg{文档类选项}*){sjtuthesis}
% \end{latex}
%
% 部分选项采用 \meta{key}|=|\meta{value} 的形式,需要使用逗号分隔各选项。
% 当 \meta{value} 省略时,将采用默认值。在下文的说明中,将用\textbf{粗体}表示默认值。
%
% \subsubsection{通用选项}
%
% \begin{function}[rEXP,updated=2022-12-03]{type}
%   \begin{syntax}
%     \OPT{type} = <bachelor|(master)|doctor>
%   \end{syntax}
%   论文类型。三种选项分别代表学士学位论文、硕士学位论文、博士学位论文。
% \end{function}
%
% \begin{function}[updated=2023-03-23]{lang}
%   \begin{syntax}
%     \OPT{lang} = <(zh)|en|de|ja>
%   \end{syntax}
%   论文主要语言。可选中文、英文、德文或日文,该选项会改变文档中的一些标题的名字。
%   下文中 \meta{lang} 可以指定为这些选项中的其中之一。
% \end{function}
%
% \begin{function}{draft, final}
%   是否开启草稿模式。
%   \opt{draft} 开启草稿模式,所有的图片将不会被加载,超过边界的区域将会被涂上黑色色块。
%   \opt{final} 关闭草稿模式。
%   默认为 \opt{final}。
% \end{function}
%
% \begin{function}[rEXP]{review}
%   盲审模式。开启盲审模式将隐去作者姓名、导师姓名、班级、学号等个人信息,
%   删去版权使用授权书、原创性声明和致谢页。默认关闭。
% \end{function}
%
% \subsubsection{页面设置}
%
% \begin{function}{oneside, twoside}
%   指明论文的单双面模式。
%   \opt{oneside} 为单面模式,
%   \opt{twoside} 为双面模式。
%   \begin{itemize}
%     \item 在 \cls{sjtuthesis} 文档类中,默认为 \opt{twoside}。
%     \item 在 \cls{sjtureport} 和 \cls{sjtuarticle} 文档类中,默认为 \opt{oneside}。
%   \end{itemize}
% \end{function}
%
% \begin{function}{openright, openany}
%   指明论文是否奇数页开章。
%   \opt{openright} 为从奇数页开始新章,
%   \opt{openany} 为从任意页开始新章。
%   \begin{itemize}
%     \item 在 \cls{sjtuthesis} 文档类中,默认为 \opt{openright}。
%     \item 在 \cls{sjtureport} 文档类中,默认为 \opt{openany}。
%     \item 在 \cls{sjtuarticle} 文档类中,该选项不可用。
%   \end{itemize}
% \end{function}
%
% \begin{function}[added=2022-12-23]{titlepage, notitlepage}
%   指明论文的标题形式。
%   \opt{titlepage} 为使用标题页。
%   \opt{notitlepage} 为使用标题块。
%   \begin{itemize}
%     \item 在 \cls{sjtuthesis} 文档类中,该选项不可用,只能使用标题页。
%     \item 在 \cls{sjtureport} 文档类中,默认为 \opt{titlepage}。
%     \item 在 \cls{sjtuarticle} 文档类中,默认为 \opt{notitlepage}。
%   \end{itemize}
% \end{function}
%
% \subsubsection{字体选项}
% \label{sec:fontsetup}
%
% \begin{function}[updated=2022-12-18]{zihao}
%   \begin{syntax}
%     \OPT{zihao} = <(-4)|5>
%   \end{syntax}
%   论文默认字号,可以设定为小四号或五号。
%   \begin{itemize}
%     \item 在 \cls{sjtuthesis} 和 \cls{sjtureport} 文档类中默认为小四号;
%     \item 在 \cls{sjtuarticle} 文档类中默认为五号。
%   \end{itemize}
% \end{function}
%
% \begin{function}[added=2023-10-24]{baselineskip}
%   \begin{syntax}
%     \OPT{baselineskip} = <长度|false>
%   \end{syntax}
%   正文基线间距。
%   \begin{itemize}
%     \item 在 \cls{sjtuthesis} 文档类中,默认为 20 磅。
%     \item 在 \cls{sjtureport} 和 \cls{sjtuarticle} 文档类中,默认为 \opt{false};
%           此时正文基线间距为字号的 1.2 倍。
%   \end{itemize}
% \end{function}
%
% \begin{function}{cjk-font}
%   \begin{syntax}
%     \OPT{cjk-font} = <(auto)|(fandol)|windows|mac|ubuntu|adobe|founder|none>
%   \end{syntax}
%   指定 CJK 字体集。\sjtutex{} 预定义了一些 CJK 字体组合,
%   具体配置见表 \ref{tab:cjkfonts}。
%   默认情况下会根据操作系统自动配置:
%   Windows 系统默认使用 \opt{windows},
%   macOS 系统默认使用 \opt{mac},
%   Linux 系统默认使用 \opt{fandol}。
%   找不到对应定义的 CJK 字体集时的回退选项为 \opt{fandol}。
%
%   \textbf{仅} \opt{windows} 和 \opt{founder} 字体集支持 \pdfLaTeX{} 直接生成 PDF。
%   其他 CJK 字体集使用 \pdfTeX{} 引擎需要先通过 \LaTeX{} 生成 DVI,
%   然后再使用 DVIPDFM\textit{x} 转换为 PDF。
% \end{function}
%
% \begin{table}[ht]
% \centering\small
% \setlength\leftskip{0pt plus 1 fil minus \marginparwidth}
% \begin{threeparttable}
%   \caption{CJK 字体配置}
%   \label{tab:cjkfonts}
%   \tabcolsep=3pt
%   \begin{tabular}{ccccccc}
%     \toprule
%     	                   & \strong{宋体}     & \strong{黑体}    & \strong{仿宋} & \strong{楷体}   & \strong{明朝体}\tnote{b} & \strong{哥特体}\tnote{b}  \\
%     \midrule
%       |fandol|\tnote{a}  & Fandol 宋体       & Fandol 黑体      & Fandol 仿宋   & Fandol 楷体     & HaranoAjiMincho          & HaranoAjiGothic           \\
%       |windows|          & (中易)宋体      & (中易)黑体     & (中易)仿宋  & (中易)楷体    & MS Mincho                & MS Gothic                 \\
%       |mac|              & (华文)宋体-简   & (华文)黑体-简  & 华文仿宋      & (华文)楷体-简 & Hiragino Mincho ProN     & Hiragino Kaku Gothic ProN \\
%       |ubuntu|           & Noto Serif CJK SC & Noto Sans CJK SC & ---           & 文鼎 PL 简中楷  & Noto Serif CJK JP        & Noto Sans CJK JP          \\
%       |adobe|            & Adobe 宋体        & Adobe  黑体      & Adobe  仿宋   & Adobe 楷体      & Kozuka Mincho Pr6N       & Kozuka Gothic Pr6N        \\
%       |founder|          & 方正书宋          & 方正黑体         & 方正仿宋      & 方正楷体        & IPAMincho                & IPAGothic                 \\
%     \bottomrule
%   \end{tabular}
%   \begin{tablenotes}
%     \item[a] 发行版中自带的 \opt{fandol} 中文字库容易出现缺字的情况;
%       我们建议 Linux 用户使用 \opt{ubuntu} 或自行配置合适的字体;参见
%       \href{https://github.com/sjtug/SJTUThesis/wiki/在线使用说明}
%       {Wiki 页面《在线使用说明》}。
%     \item[b] 日文模板才需要日文明朝体与哥特体;日文字体使用 \pkg{fontspec}
%       宏包设置,故日文模板不支持 \pdfTeX{} 引擎,请使用 \XeLaTeX{} 或
%       \LuaLaTeX{} 编译。
%   \end{tablenotes}
% \end{threeparttable}
% \end{table}
%
% \begin{function}{text-font}
%   \begin{syntax}
%     \OPT{text-font} = <(newtx)|times|stixtwo|xits|newpx|cambria|newcm|lm|libertinus|none>
%   \end{syntax}
%   指定西文字体集。\sjtutex{} 预定义了一些西文字体组合,
%   具体配置见表 \ref{tab:latinfonts}。
%   找不到定义的西文字体集时的回退选项为 \opt{newtx}。
% \end{function}
%
% \begin{function}{math-font}
%   \begin{syntax}
%     \OPT{math-font} = <(auto)|(newtx)|times|stixtwo|xits|newpx|cambria|newcm|lm|libertinus|none>
%   \end{syntax}
%   指定数学字体集。\sjtutex{} 预定义了一些数学字体组合,
%   具体配置见表 \ref{tab:latinfonts} 数学字体列。
%   默认跟随西文字体 \opt{text-font} 的设置。
%   找不到定义的数学字体集时的回退选项为 \opt{newtx}。
%
%   \opt{xits},\opt{newcm},\opt{cambria} 选项仅支持 \XeLaTeX/\LuaLaTeX{} 编译。
% \end{function}
%
% \begin{table}[ht]
% \centering\small
% \begin{threeparttable}
%   \caption{西文字体与数学字体配置}
%   \label{tab:latinfonts}
%   \begin{tabular}{ccccc}
%     \toprule
%       & \strong{正文字体} & \strong{无衬线字体} & \strong{等宽字体} & \strong{数学字体} \\
%     \midrule
%       |newtx|                  & TG Termes X\tnote{a}     & TG Heros        & TG Cursor   & newtx           \\
%       \multirow{2}{*}{|times|} & Times New Roman\tnote{b} & Arial           & Courier New & \multirow{2}{*}{mathptmx} \\
%                                & Times\tnote{c}           & Helvetica       & Courier     &                 \\
%       |stixtwo|                & STIX Two Text            & TG Heros        & TG Cursor   & STIX Two Math   \\
%       |xits|                   & XITS                     & TG Heros        & TG Cursor   & XITS Math       \\
%       |newpx|                  & TG Pagella X             & TG Heros        & TG Cursor   & newpx           \\
%       |cambria|                & Cambria                  & Calibri         & Consolas    & Cambria Math    \\
%       |newcm|                  & New CM\tnote{d}          & New CM Sans     & New CM Mono & New CM Math     \\
%       |lm|                     & LM Roman\tnote{e}        & LM Sans         & LM Mono     & LM Math         \\
%       |libertinus|             & Libertinus Serif         & Libertinus Sans & LM Mono     & Libertinus Math \\
%     \bottomrule
%   \end{tabular}
%   \begin{tablenotes}
%     \item[a] “TG”是 TeX Gyre 的缩写。
%     \item[b] 本行中,Times New Roman、Arial 和 Courier New 是商业字体,
%       在 Windows 和 macOS 系统上均默认安装。
%     \item[c] 使用 \pdfTeX{} 引擎时,实际使用对应字体的 Type 1 开源版本。
%     \item[d] “CM”是 Computer Modern 的缩写。
%     \item[e] “LM”是 Latin Modern 的缩写。
%   \end{tablenotes}
% \end{threeparttable}
% \end{table}
%
% \begin{function}[added=2022-12-03,updated=2023-01-05]{math-style}
%   \begin{syntax}
%     \OPT{math-style} = <(ISO)|TeX>
%   \end{syntax}
%   数学符号样式。该选项将影响 \opt{uppercase-greek}、\opt{integral}、
%   \opt{integral-limits} 选项。
%   默认遵循 ISO 80000-2 标准设置,即斜体的大写希腊字母、直立的积分号
%   以及积分号上下限置于上下方。
%   用户也可以逐项修改数学样式。
% \end{function}
%
% \begin{function}[added=2023-01-05]{uppercase-greek}
%   \begin{syntax}
%     \OPT{uppercase-greek} = <slanted|upright>
%   \end{syntax}
%   大写希腊字母的正/斜体。
% \end{function}
%
% \begin{function}[added=2023-01-05]{integral}
%   \begin{syntax}
%     \OPT{integral} = <slanted|upright>
%   \end{syntax}
%   积分号的正/斜体。
% \end{function}
%
% \begin{function}[added=2023-01-05]{integral-limits}
%   \begin{syntax}
%     \OPT{integral-limits} = <true|false>
%   \end{syntax}
%   行间公式中积分号上下限的位置,
%   \opt{true} 使得上下限在积分号上下方,
%   \opt{false} 使得上下限在积分号右侧。
%   该选项只影响行间公式,行内公式统一居右侧,不受影响。
% \end{function}
%
% \subsection{论文信息设置}
%
% \begin{function}{\sjtusetup}
%   \begin{syntax}
%     \TNA{sjtusetup}\marg{键值列表}
%   \end{syntax}
%   本模板提供了一系列选项,可由您自行配置。载入文档类之后,以下
%   所有选项均可通过统一的命令 \tn{sjtusetup} 来设置。
% \end{function}
%
% \tn{sjtusetup} 的参数是一组由(英文)逗号隔开的选项列表,列表中的选项通常是
% \meta{key}|=|\meta{value} 的形式。对于同一项,后面的设置将会覆盖前面的设置。在
% 下文的说明中,将用\textbf{粗体}表示默认值。\tn{sjtusetup} 支持不同类型以及多种
% 层次的选项设定。键值列表中,“|=|”左右的空格不影响设置;但需注意,参数列表中不
% 可以出现空行。
%
% \begin{latex}[moretexcs={\sjtusetup}, emph={[2]info,style,name},
%   emph={[3]zh,en,title,author,float-num-sep,achv}]
% \sjtusetup{
%   info = {
%     zh/title      = {上海交通大学学位论文模板示例文档},
%     en/title      = {A Sample Document for SJTU Thesis Template},
%     zh/author     = {某某},
%     en/author     = {Mo Mo},
%   },
%   style = {
%     float-num-sep = {-},
%   },
%   name = {
%     achv          = {攻读学位期间完成的论文},
%   },
% }
% \end{latex}
%
% \subsubsection{信息域}
%
% \begin{function}[updated=2023-03-14]{info}
%   \begin{syntax}
%     \OPS{info} = \marg{键值列表}
%   \end{syntax}
%   该选项包含许多子项目,用于录入论文信息。具体内容见下。
%   \begin{itemize}
%     \item 在 \cls{sjtuthesis} 文档类中,现在推荐使用以语言代码前缀 \meta{lang}
%           (比如 \opt{zh} 或 \opt{en})下的键来设定对应语言的论文信息,见第
%           \ref{sec:sjtuthesiskey} 节;省略语言前缀不带“*”的项目表示对应的中文
%           字段、带“*”的项目表示对应的英文字段属于老用法,仍然兼容但请及时更新
%           至新用法。
%     \item 在 \cls{sjtureport} 和 \cls{sjtuarticle} 文档类中,不需要使用语言代
%           码前缀。此时推荐直接使用标准接口来设定这些信息,这些标准接口不属于键
%           值列表,应当直接写在导言区内,见第 \ref{sec:sjtureportcmd} 节。
%   \end{itemize}
% \end{function}
%
% \paragraph{适用于 \cls{sjtuthesis} 文档类的键}
% \label{sec:sjtuthesiskey}
%
% \begin{function}[rEXP,updated=2023-03-14]{info/<lang>/title}
%   \begin{syntax}
%     \OPT{\meta{lang}/title} = \marg{标题}
%   \end{syntax}
%   标题。
% \end{function}
%
% \begin{function}[rEXP,updated=2023-03-14]{info/<lang>/display-title}
%   \begin{syntax}
%     \OPT{\meta{lang}/display-title} = \marg{标题页标题}
%   \end{syntax}
%   标题页中的题目。默认为跟随对应语言的标题。
%   如果标题过长,可以尝试使用“|\\|”手动断行。
% \end{function}
%
% \begin{function}[rEXP,added=2022-12-17,updated=2023-03-14]{info/<lang>/subject}
%   \begin{syntax}
%     \OPT{\meta{lang}/subject} = \marg{主题}
%   \end{syntax}
%   文档主题。一般显示在中文标题页校徽下方。
%   默认值类似于 “上海交通大学学士学位论文” 或 “A Dissertation Submitted to
%   Shanghai Jiao Tong University for the Degree of Bachelor”。
% \end{function}
%
% \begin{function}[rEXP,updated=2023-03-14]{info/<lang>/keywords}
%   \begin{syntax}
%     \OPT{\meta{lang}/keywords} = \marg{中文关键字}
%   \end{syntax}
%   关键字列表。各关键字之间需使用英文逗号隔开。
%   为防止歧义,可以用分组括号“|{...}|”把各字段括起来。
% \end{function}
%
% \begin{function}[rEXP,updated=2023-03-14]{info/<lang>/author}
%   \begin{syntax}
%     \OPT{\meta{lang}/author} = \marg{姓名}
%   \end{syntax}
%   作者姓名。
% \end{function}
%
% \begin{function}[rEXP]{info/id}
%   \begin{syntax}
%     \OPT{id} = \marg{学号}
%   \end{syntax}
%   学号。该键不需要语言前缀。
% \end{function}
%
% \begin{function}[rEXP,updated=2023-03-14]{info/<lang>/supervisor,
%   info/<lang>/assoc-supervisor,info/<lang>/co-supervisor}
%   \begin{syntax}
%     \OPT{\meta{lang}/supervisor}       = \marg{导师姓名}
%     \OPT{\meta{lang}/assoc-supervisor} = \marg{副导师姓名}
%     \OPT{\meta{lang}/co-supervisor}    = \marg{联合导师姓名}
%   \end{syntax}
%   导师、副导师、联合导师姓名。
% \end{function}
%
% \begin{function}[rEXP,updated=2023-03-14]{info/<lang>/degree}
%   \begin{syntax}
%     \OPT{\meta{lang}/degree} = \marg{学位名称}
%   \end{syntax}
%   申请学位中英文名称。
%   包括申请的学位类别和级别,如“工学硕士”、“理学博士”等。
%   学士论文无需标注。
% \end{function}
%
% \begin{function}[rEXP,updated=2023-03-14]{info/<lang>/department}
%   \begin{syntax}
%     \OPT{\meta{lang}/department} = \marg{院系名称}
%   \end{syntax}
%   院系名称。
% \end{function}
%
% \begin{function}[rEXP,updated=2023-03-14]{info/<lang>/major}
%   \begin{syntax}
%     \OPT{\meta{lang}/major} = \marg{专业名称}
%   \end{syntax}
%   专业名称。
% \end{function}
%
% \begin{function}[rEXP,updated=2023-03-14]{info/<lang>/fund}
%   \begin{syntax}
%     \OPT{\meta{lang}/fund} = \marg{资助基金名称}
%   \end{syntax}
%   资助基金列表。各资助基金名称之间需使用英文逗号隔开。
%   为防止歧义,可以用分组括号“|{...}|”把各字段括起来。
% \end{function}
%
% \begin{function}[rEXP,updated=2023-02-25]{info/date}
%   \begin{syntax}
%     \OPT{date} = \marg{ISO 日期}
%   \end{syntax}
%   日期。默认值为文档编译日期。也可以自己指定,要求使用 ISO 格式,
%   即 |yyyy-mm-dd| 或 |yyyy-mm|,否则设定无效。该键语言前缀不是必须的。
% \end{function}
%
% \begin{function}[rEXP,updated=2023-03-14]{info/<lang>/display-date}
%   \begin{syntax}
%     \OPT{\meta{lang}/display-date} = \marg{日期文字}
%   \end{syntax}
%   显示日期,可以显示不同于标准日期格式的日期,日期文字将会被原样输出。
%   设定该键时,将会覆盖 \opt{info/date} 键在对应语言下的设定。
% \end{function}
%
% \paragraph{适用于 \cls{sjtureport} 和 \cls{sjtuarticle} 文档类的命令}
% \label{sec:sjtureportcmd}
%
% \begin{function}[EXP,added=2022-12-17,updated=2023-03-14]{\title}
%   \begin{syntax}
%     \TNA{title}\marg{标题}
%   \end{syntax}
%   设置标题,覆盖 \opt{info/title} 键的值。
% \end{function}
%
% \begin{function}[EXP,added=2022-12-17,updated=2023-03-14]{\author}
%   \begin{syntax}
%     \TNA{author}\marg{姓名}
%   \end{syntax}
%   设置作者姓名,覆盖 \opt{info/author} 键的值。
% \end{function}
%
% \begin{function}[EXP,added=2022-12-17,updated=2023-03-14]{\date}
%   \begin{syntax}
%     \TNA{date}\marg{日期}
%   \end{syntax}
%   设置日期,覆盖 \opt{info/display-date} 键的值。
%   日期会被原样显示。
% \end{function}
%
% \begin{function}[EXP,added=2023-03-14]{\subject}
%   \begin{syntax}
%     \TNA{subject}\marg{主题}
%   \end{syntax}
%   文档主题。覆盖 \opt{info/subject} 键的值。
% \end{function}
%
% \begin{function}[EXP,added=2023-03-14]{\keywords}
%   \begin{syntax}
%     \TNA{keywords}\marg{关键词}
%   \end{syntax}
%   文档关键词,使用英文逗号隔开不同的关键词。覆盖 \opt{info/keywords} 键的值。
% \end{function}
%
% \subsubsection{样式域}
%
% \begin{function}{style}
%   \begin{syntax}
%     \OPS{style} = \marg{键值列表}
%   \end{syntax}
%   该选项包含许多子项目,用于设置论文样式。具体内容见下。
% \end{function}
%
% \begin{function}[added=2023-11-30]{style/equation-font}
%   \begin{syntax}
%     \OPT{equation-font} = \marg{字体设置}
%   \end{syntax}
%   行间数学公式的字体设置,该选项主要用于调整行间公式的行距,
%   不建议修改字号字形。
%   \cls{sjtuthesis} 中默认数学公式的行距为字号的 1.2 倍。
% \end{function}
%
% \begin{function}[added=2022-12-03,updated=2022-12-27]{style/float-font}
%   \begin{syntax}
%     \OPT{float-font} = \marg{字体设置}
%   \end{syntax}
%   图、表等浮动体的额外字体设置。
%   默认为 |\zihao{5}|,五号字。
% \end{function}
%
% \begin{function}[added=2022-12-20]{style/caption-font}
%   \begin{syntax}
%     \OPT{caption-font} = \marg{字体设置}
%   \end{syntax}
%   题注字体。
%   默认为 |\zihao{5}\bfseries|,粗体五号字。
% \end{function}
%
% \begin{function}[added=2022-12-20]{style/subcaption-font}
%   \begin{syntax}
%     \OPT{subcaption-font} = \marg{字体设置}
%   \end{syntax}
%   子图题注字体。
%   默认为 |\zihao{5}\normalfont|,正常字重五号字。
% \end{function}
%
% \begin{function}[added=2023-03-28]{style/fnmark-style}
%   \begin{syntax}
%     \OPT{fnmark-style} = <plain|circled>
%   \end{syntax}
%   脚注数字编号样式。
%   \opt{plain} 表示使用普通数字编号;
%   \opt{circled} 表示使用带圈数字编号。
%   在 \opt{zh} 和 \opt{ja} 语言设置中,默认为 \opt{circled};
%   在 \opt{en} 和 \opt{de} 语言设置中,默认为 \opt{plain}。
%
%   使用带圈数字编号时,由于超过 50 的带圈数字没有对应的 Unicode 码位,
%   所以每页脚注最好不要超过 50 个。
%   带圈数字默认使用 CJK 字体。通常情况下默认字体不一定包含所有带圈数字的字符,
%   此时可以设置 \opt{fnmark-font} 选项给带圈数字设置合适的字体。
% \end{function}
%
% \begin{function}[added=2022-12-03,updated=2023-03-28]{style/fnmark-font}
%   \begin{syntax}
%     \OPT{fnmark-font} = <haranoaji|\marg{字体设置}>
%   \end{syntax}
%   脚注编号的额外字体设置。
%   默认为空。
%   可以使用预设 \opt{haranoaji},支持在 Unicode 引擎中使用 HaranoAjiMincho 字体
%   中的带圈数字。
% \end{function}
%
% \begin{function}[added=2023-12-02]{num-sep}
%   \begin{syntax}
%     \OPT{num-sep} = \marg{分隔符}
%   \end{syntax}
%   图、表、公式以及定理编号中的分隔符。该选项将统一设定 \opt{float-num-sep}、
%   \opt{equation-num-sep}、\opt{theorem-num-sep} 选项。
%   用户也可以逐项修改编号分隔符。
%   默认为 \opt{.} 句点。
% \end{function}
%
% \begin{function}[updated=2023-11-29]{style/float-num-sep}
%   \begin{syntax}
%     \OPT{float-num-sep} = \marg{分隔符}
%   \end{syntax}
%   图、表等浮动体编号中的分隔符。
% \end{function}
%
% \begin{function}[updated=2023-11-29]{style/equation-num-sep}
%   \begin{syntax}
%     \OPT{equation-num-sep} = \marg{分隔符}
%   \end{syntax}
%   公式编号中的分隔符。
% \end{function}
%
% \begin{function}[added=2023-12-02]{style/theorem-num-sep}
%   \begin{syntax}
%     \OPT{theorem-num-sep} = \marg{分隔符}
%   \end{syntax}
%   定理编号中的分隔符。
% \end{function}
%
% \begin{function}{style/title-logo-color}
%   \begin{syntax}
%     \OPT{title-logo-color} = <(red)|blue|black>
%   \end{syntax}
%   选择论文标题页中校徽的颜色。
% \end{function}
%
% \begin{function}[EXP,updated=2022-12-20]{style/header-logo-color}
%   \begin{syntax}
%     \OPT{header-logo-color} = <(red)|blue|black>
%   \end{syntax}
%   选择 \cls{sjtureport} 和 \cls{sjtuarticle} 文档类中页眉的校标颜色。
% \end{function}
%
% \begin{function}[added=2022-12-20,updated=2023-03-14]{style/header-uppercase}
%   \begin{syntax}
%     \OPT{header-uppercase} = <true|(false)>
%   \end{syntax}
%   页眉英文字母是否大写。默认为 \opt{false}。
% \end{function}
%
% \begin{function}[added=2022-12-20]{style/header-font}
%   \begin{syntax}
%     \OPT{header-font} = \marg{页眉字体}
%   \end{syntax}
%   页眉字体。
%   \begin{itemize}
%     \item 在 \cls{sjtuthesis} 文档类中,默认为 |\zihao{-5}|,小五号字。
%     \item 在 \cls{sjtureport} 和 \cls{sjtuarticle} 文档类中,默认为 |\zihao{-5}\sffamily|,小五号字黑体。
%   \end{itemize}
% \end{function}
%
% \begin{function}[added=2022-12-20]{style/footer-font}
%   \begin{syntax}
%     \OPT{footer-font} = \marg{页脚字体}
%   \end{syntax}
%   页脚字体。默认为 |\zihao{-5}|,小五号字。
% \end{function}
%
% \begin{function}[added=2022-12-03]{style/page-number}
%   \begin{syntax}
%     \OPT{page-number} = \marg{页码设置}
%   \end{syntax}
%   设置页码的显示样式,其中 |#1| 代表当前页码。
%   默认为 |{#1}|,即仅显示页码本身。
% \end{function}
%
% \begin{function}[added=2023-11-30]{style/keywords-format}
%   \begin{syntax}
%     \OPT{keywords-format} = <(plain)|hang>
%   \end{syntax}
%   设置关键词格式。默认为 \opt{plain} 无缩进的普通段落,另可选 \opt{hang} 悬挂格式。
% \end{function}
%
% \subsubsection{名称域}
%
% \begin{function}{name}
%   \begin{syntax}
%     \OPS{name} = \marg{键值列表}
%   \end{syntax}
%   选项包含许多子项目,用于设置论文中一些标题的名称。部分选项只能在 \cls{sjtuthesis} 中使用。
%   具体内容见表 \ref{tab:names}。
% \end{function}
%
% \begin{function}[updated=2023-03-18]{name/contents,name/listfigure,name/listtable,
%   name/figure,name/table,name/abstract,name/index,name/appendix,name/proof,name/bib,
%   name/figure*,name/table*,name/algorithm,name/listalgorithm,
%   name/abbr,name/nom,name/ack,name/resume,name/digest,name/achv}
% \end{function}
%
% \begingroup
% \vspace{-48ex}
% \small \tabcolsep=3pt
% \begin{longtable}{l|p{7em}|>{\raggedright\arraybackslash}p{9em}|p{11em}|p{8em}}
%   \caption{\opt{name} 选项的默认设置} \label{tab:names} \\
%   \toprule
%     \strong{选项}        & \strong{|lang = zh|}   & \strong{|lang = en|}          & \strong{|lang = de|}   & \strong{|lang = ja|} \\
%   \midrule
%   \endfirsthead
%     \multicolumn{5}{r}{续表~\thetable} \\
%   \toprule
%     \strong{选项}        & \strong{|lang = zh|}   & \strong{|lang = en|}          & \strong{|lang = de|}   & \strong{|lang = ja|} \\
%   \midrule
%   \endhead
%     \bottomrule
%     \multicolumn{5}{r}{续下页}
%   \endfoot
%     \bottomrule
%   \endlastfoot
%     |contents|           & 目录                   & Contents                      & Inhaltsverzeichnis     & 目次                 \\
%     |listfigure|         & 插图                   & List of Figures               & Abbildungsverzeichnis  & 図目次               \\
%     |listtable|          & 表格                   & List of Tables                & Tabellenverzeichnis    & 表目次               \\
%     |figure|             & 图                     & Figure                        & Abbildung              & 図                   \\
%     |table|              & 表                     & Table                         & Tabelle                & 表                   \\
%     |abstract| \rexpstar & 摘要                   & Abstract                      & Zusammenfassung        & 概要                 \\
%     |index|              & 索引                   & Index                         & Index                  & 索引                 \\
%     |appendix|           & 附录                   & Appendix                      & Anhang                 & 付録                 \\
%     |proof|              & 证明                   & Proof                         & Beweis                 & 证明                 \\
%     |bib|                & 参考文献               & Bibliography                  & Literaturverzeichnis   & 参考文献             \\
%   \midrule
%     |figure*|            & Figure                 & 图                            & Figure                 & Figure               \\
%     |table*|             & Table                  & 表                            & Table                  & Table                \\
%     |algorithm|          & 算法                   & Algorithm                     & Algorithmus            & アルゴリズム         \\
%     |listalgorithm|      & 算法                   & List of Algorithms            & Algorithmenverzeichnis & アルゴリズム目次     \\
%     |abbr| \rexpstar     & 缩略语对照表           & Abbreviation                  & Abkürzungsverzeichnis  & 略語表               \\
%     |nom| \rexpstar      & 主要符号对照表         & Nomenclature                  & Symbolverzeichnis      & 記号表               \\
%     |ack| \rexpstar      & 致谢                   & Acknowledgements              & Danksagungen           & 謝辞                 \\
%     |resume| \rexpstar   & 个人简历               & Resume                        & Lebenslauf             & 履歴書               \\
%     |digest| \rexpstar   & 大摘要                 & Digest                        & Kurzfassung            & 要約                 \\
%     |achv| \rexpstar     & 学术论文和科研成果目录 & List of Research Achievements & Forschungsleistungen   & 研究業績書           \\
% \end{longtable}
% \endgroup
%
% \section{内容编写}
%
% \begin{function}{document}
% 在文档开始后进行内容编写,文档内容由 |document| 环境包裹。
%   \begin{latex}[emph={[1]document}]
% \begin{document}
%   (*\meta{文档内容}*)
% \end{document}
%   \end{latex}
% \end{function}
%
% \subsection{标题页和声明页}
%
% \begin{function}[updated=2022-12-03]{\maketitle}
%   \begin{syntax}
%     \TNA{maketitle}
%   \end{syntax}
%   生成标题。
%   \begin{itemize}
%     \item 在 \cls{sjtuthesis} 文档类中,生成标题页。
%     \item 在 \cls{sjtureport} 和 \cls{sjtuarticle} 文档类中,
%           \begin{itemize}
%             \item 若处于 \opt{titlepage} 文档类选项中,生成标题页。\cls{sjtureport} 文档类默认。
%             \item 若处于 \opt{notitlepage} 文档类选项中,生成标题块。\cls{sjtuarticle} 文档类默认。
%           \end{itemize}
%   \end{itemize}
% \end{function}
%
% \begin{function}[rEXP,updated=2022-12-20]{\copyrightpage}
%   \begin{syntax}
%     \TNA{copyrightpage}
%     \TNA{copyrightpage}\oarg{授权书扫描件}
%   \end{syntax}
%   生成空白版权使用授权书。
%   接受一个可选参数用于插入版权使用授权书扫描件,使用可选参数时需要手动加载 \pkg{pdfpages} 宏包。
% \end{function}
%
% \subsection{前文部分}
%
% \begin{function}[rEXP]{\frontmatter}
%   \begin{syntax}
%     \TNB{frontmatter}
%   \end{syntax}
%   声明前文部分开始。
% \end{function}
%
% \begin{function}[rEXP,updated=2023-11-29]{abstract,abstract*}
%   \cls{sjtuthesis} 文档类中对应语言的摘要环境,使用可选参数指定语言,默认为 \opt{zh}。
%   使用带星号的 \env{abstract*} 环境不会出现在目录中。
%   \begin{latex}[emph={[1]abstract}]
% \begin{abstract}(*\oarg{lang}*)
%   (*\meta{摘要}*)
% \end{abstract}
%   \end{latex}
% \end{function}
%
% \begin{function}[EXP,updated=2023-03-14]{abstract}
%   \cls{sjtuarticle} 和 \cls{sjtureport} 文档类中摘要环境。
%   会在结尾添加由 \tn{keywords} 命令设定的关键词。
%   \begin{latex}[emph={[1]abstract}]
% \begin{abstract}
%   (*\meta{摘要}*)
% \end{abstract}
%   \end{latex}
% \end{function}
%
% \begin{function}{\tableofcontents,\tableofcontents*,\listoffigures,\listoffigures*,
%   \listoftables,\listoftables*,\listofalgorithms,\listofalgorithms*}
%   目录、插图、表格和算法等索引命令如表 \ref{tab:list} 所示,将其插入到期望的
%   位置即可。带*的命令表示对应的索引表不会出现在目录中。
%   \begin{table}[H]
%     \centering
%     \small
%     \begin{threeparttable}
%       \caption{目录和索引表}
%       \label{tab:list}
%       \begin{tabular}{lp{4cm}lp{4cm}}
%         \toprule
%         \strong{用途}     & \strong{命令}         & \strong{用途}     & \strong{命令}          \\
%         \midrule
%         目录              & \tn{tableofcontents}  & 插图索引          & \tn{listoffigures}     \\
%                           & \tn{tableofcontents*} &                   & \tn{listoffigures*}    \\
%         \midrule
%         表格索引          & \tn{listoftables}     & 算法索引\tnote{a} & \tn{listofalgorithms}  \\
%                           & \tn{listoftables*}    &                   & \tn{listofalgorithms*} \\
%         \bottomrule
%       \end{tabular}
%       \begin{tablenotes}
%         \item[a] 启用 \pkg{algorithm2e} 或 \pkg{algorithm} 后有效。
%       \end{tablenotes}
%     \end{threeparttable}
%   \end{table}
% \end{function}
%
% \subsection{正文部分}
%
% \begin{function}[rEXP]{\mainmatter}
%   \begin{syntax}
%     \TNB{mainmatter}
%   \end{syntax}
%   声明正文部分开始。正文部分是论文的核心,您可以分章节撰写。
%   如有需求,也可以采用多文件编译的方式。
% \end{function}
%
% \begin{function}[updated=2022-12-03]{\footnote}
%   \begin{syntax}
%     \TNA{footnote}\oarg{脚注编号}\marg{脚注文字}
%   \end{syntax}
%   插入脚注。其中脚注编号参数是可选的,一般不需要输入。
% \end{function}
%
% \begin{function}{assumption,axiom,conjecture,corollary,definition,example,
%   exercise,lemma,problem,proposition,theorem}
%   \sjtutex{} 预定义了一系列数学环境。在启用 \pkg{ntheorem} 或 \pkg{amsthm}
%   宏包后有效,环境如表 \ref{tab:theorems} 所示。
%   \begin{table}[H]
%     \centering
%     \small
%     \caption{预定义的数学环境}
%     \label{tab:theorems}
%     \begin{tabular}{*{7}{l}}
%       \toprule
%       \env{assumption}  & \env{axiom}   & \env{conjecture} & \env{corollary}    & \env{definition}  & \env{example}   & \env{exercise}  \\
%       假设        & 公理    & 猜想       & 推论         & 定义        & 例        & 练习      \\
%       \midrule
%       \env{lemma}       & \env{problem} & \env{proof}      & \env{proposition}  & \env{remark}      & \env{solution}  & \env{theorem}   \\
%       引理        & 问题    & 证明       & 命题         & 注          & 解        & 定理      \\
%       \bottomrule
%     \end{tabular}
%   \end{table}
% \end{function}
%
% \begin{function}[added=2023-10-24]{\setbaselineskip}
%   \begin{syntax}
%     \TNA{setbaselineskip}\marg{长度}
%   \end{syntax}
%   设置当前的基线间距,一般在字号命令之后使用。
% \end{function}
%
% \begin{function}{\appendix}
%   \begin{syntax}
%     \TNA{appendix}
%   \end{syntax}
%   附录由 \tn{appendix} 命令开启,然后像正文一样书写。
% \end{function}
%
% \begin{function}[rEXP,updated=2022-03-02]{nomenclature,nomenclature*}
%   符号对照表环境。带星号的版本不会出现在目录中。可以使用可选参数手动设置标题。
%   符号对照表环境仅设置标题,内部实现可由用户自行决定。
%   可以使用 \pkg{longtable},也可以使用 \pkg{nomencl} 宏包。
%   \begin{latex}[emph={[1]abbreviation}]
% \begin{abbreviation}(*\oarg{标题}*)
%   (*\meta{符号对照表}*)
% \end{abbreviation}
%   \end{latex}
% \end{function}
%
% \begin{function}[rEXP,updated=2022-03-02]{abbreviation,abbreviation*}
%   缩略语对照表环境。带星号的版本不会出现在目录中。可以使用可选参数手动设置标题。
%   缩略语对照表环境仅设置标题,内部实现可由用户自行决定。
%   \begin{latex}[emph={[1]abbreviation}]
% \begin{abbreviation}(*\oarg{标题}*)
%   (*\meta{缩略语对照表}*)
% \end{abbreviation}
%   \end{latex}
% \end{function}
%
% \subsection{后文部分}
%
% \begin{function}[rEXP]{\backmatter}
%   \begin{syntax}
%     \TNB{backmatter}
%   \end{syntax}
%   声明后文部分开始。 后文部分包含致谢等。
% \end{function}
%
% \begin{function}[rEXP,updated=2022-02-24]{acknowledgements}
%   致谢环境。盲审模式下致谢将被隐去。可以使用可选参数手动设置标题。
%   \begin{latex}[emph={[1]acknowledgements}]
% \begin{acknowledgements}(*\oarg{标题}*)
%   (*\meta{致谢内容}*)
% \end{acknowledgements}
%   \end{latex}
% \end{function}
%
% \begin{function}[rEXP,updated=2022-02-24]{achievements}
%   科研成果环境,可以使用可选参数手动设置标题。内部请配合使用下面的附录用文献列表环境
%   \env{bibliolist} 和 \env{bibliolist*}。你可以在该环境中使用带星号的节次命令
%   以分隔不同的类型的成果(比如学术论文、专利等);你也可以使用多个 \env{achievements}
%   环境,配合不同的可选参数作为标题,展示不同类型的成果。
%   \begin{latex}[emph={[1]achievements}]
% \begin{achievements}(*\oarg{标题}*)
%   (*\meta{获得的科研成果}*)
% \end{achievements}
%   \end{latex}
% \end{function}
%
% \begin{function}[rEXP,updated=2022-03-23]{bibliolist,bibliolist*}
%   附录用文献环境,只允许在 \env{achievements} 环境中使用。
%   需要指定最长条目的编号作为参数,比如 \opt{99};
%   如果将该强制参数被指定为空,将不显示编号,每条以悬挂缩进做区分。
%   环境内部使用 \tn{item} 来分隔各条目,在同一个 \env{achievements} 环境内、不同的
%   \env{bibliolist} 或 \env{bibliolist*} 环境中编号连续递增、不会间断。
%   普通模式下显示 \env{bibliolist} 中的内容,盲审模式下显示 \env{bibliolist*} 中的内容。
%   \begin{latex}[emph={[1]bibliolist,bibliolist*}]
% \begin{bibliolist}(*\marg{最长条目编号}*)
%    \item (*\meta{文献条目}*)
% \end{bibliolist}
% \begin{bibliolist*}(*\marg{最长条目编号}*)
%    \item (*\meta{文献条目(隐去姓名)}*)
% \end{bibliolist*}
%   \end{latex}
% \end{function}
%
% \begin{function}[rEXP,updated=2022-02-24]{resume}
%   简历环境。盲审模式下简历将被隐去。可以使用可选参数手动设置标题。
%   \begin{latex}[emph={[1]resume}]
% \begin{resume}(*\oarg{标题}*)
%   (*\meta{简历内容}*)
% \end{resume}
%   \end{latex}
% \end{function}
%
% \begin{function}[rEXP]{digest}
%   学士论文大摘要,使用可选参数设定语言,默认为 \opt{en}。
%   \begin{latex}[emph={[1]digest}]
% \begin{digest}(*\oarg{lang}*)
%   (*\meta{大摘要}*)
% \end{digest}
%   \end{latex}
% \end{function}
%
% \section{宏包依赖情况}
%
% 使用不同编译方式、指定不同选项,会导致宏包依赖情况有所不同。
% 具体如下:
% \begin{itemize}
%   \item 在任何情况下,文档类都会\emph{显式}调用以下宏包
%     (或文档类):
%     \begin{itemize}
%       \item \cls{ctexbook}、\cls{ctexrep} 和 \cls{ctexart},
%         提供中文排版的通用框架。属于 \CTeX{} 宏集 \cite{CTeX}。
%       \item \pkg{mathtools},对 \LaTeX{} 的数学排版功能进行了全面扩展。
%         是 \pkg{amsmath} 的扩充。
%       \item \pkg{geometry},用于调整页面尺寸。
%       \item \pkg{fancyhdr},处理页眉页脚。
%       \item \pkg{tocloft},设置目录格式。
%       \item \pkg{caption}、\cls{bicaption} 和 \cls{subcaption},用于设置题注。
%       \item \pkg{xcolor},提供彩色支持。
%       \item \pkg{graphicx},提供图形插入的接口。
%       \item \pkg{enumitem},设置列表环境格式。
%     \end{itemize}
%   \item \cls{sjtuthesis} 会调用 \pkg{xtemplate} 和 \pkg{array},用于辅助
%     标题页等特殊页面的排版。
%   \item 部分西文与数学字体预设会调用相关的字体宏包,具体调用情况请参见对应的
%     字体预设文件。
%   \item 部分数字字体预设会调用 \pkg{unicode-math} 处理 Unicode 编码的
%     OpenType 数学字体。在未启用 \pkg{unicode-math} 的情况下,会调用 \pkg{bm}
%     来选择粗体数学符号。
% \end{itemize}
%
% 这里只列出了本模板直接调用的宏包。这些宏包自身的调用情况,
% 此处不再具体展开。如有需要,请参阅相关文档。
%
% \begin{thebibliography}{99}
%
% \addcontentsline{toc}{section}{\refname}
%
% \newcommand\urlprefix{\newline\hspace*{\fill}}
% \let\OldUrl=\url
% \renewcommand\url[2][]{{\small\textit{#1}~\OldUrl{#2}}}
% \newcommand\CTANurl[2][]{{^^A
%   \small\textit{#1}~\href{https://mirror.ctan.org/#2}{\texttt{CTAN://#2}}}}
%
% \subsection*{图书}
%
% \bibitem{Knuth1986}
% \textsc{Knuth D E}.
% \newblock \textit{The \TeX book: Computers \& Typesetting, volume A}\allowbreak[M].
% \newblock Boston: Addison--Wesley Publishing Company, 1986.
% \urlprefix \CTANurl[源代码^^A
%   \footnote{此代码只可作为学习之用。未经 Knuth 本人同意,您不应当编译此文档。}:]^^A
%   {systems/knuth/dist/tex/texbook.tex}
%
% \bibitem{LiuHY2013}
% 刘海洋.
% \newblock \textit{\LaTeX{} 入门} [M].
% \newblock 北京: 电子工业出版社, 2013
%
% \subsection*{标准、规范}
%
% \bibitem{gb-t-7713.1-2006}
% 国务院学位委员会办公室, 全国信息与文献标准化技术委员会.
% \newblock \textit{学位论文编写规则: GB/T 7713.1--2006}\allowbreak[S].
% \newblock 北京: 中国标准出版社, 2007
%
% \bibitem{cy-t-35-2001}
% 全国信息与文献标准化技术委员会第七分委员会, 中华人民共和国新闻出版总署.
% \newblock \textit{科技文献的章节编号方法: CY/T 35--2001}\allowbreak[S].
% \newblock [S.l. : s.n.], 2001.
%
% \bibitem{SJTUGS2023}
% 上海交通大学研究生院.
% \newblock \textit{上海交通大学博士、硕士学位论文撰写指南}\allowbreak[EB/OL].
% \newblock (2023-11-03)[2023-12-04].
% \urlprefix\url{https://www.gs.sjtu.edu.cn/post/detail/Z3MxNDc=}
%
% \subsection*{宏包、模版}
%
% \bibitem{source2e}
% \textsc{Braams J}, \textsc{Carlisle D}, \textsc{Jeffrey A}, et~al.
% \newblock \textit{The \LaTeXe{} Sources}\allowbreak[CP/OL].
% \newblock (2023-11-01).
% \urlprefix\url{https://ctan.org/pkg/latex}
% \urlprefix\CTANurl[源代码:]{macros/latex/base/source2e.pdf}
%
% \bibitem{interface3}
% \textsc{The \LaTeX{} Project}.
% \newblock \textit{The \LaTeXiii{} Interfaces}\allowbreak[EB/OL].
% \newblock (2023-11-09).
% \urlprefix\url{https://ctan.org/pkg/l3kernel}
% \urlprefix\CTANurl[文档:]{macros/latex/l3kernel/interface3.pdf}
%
% \bibitem{CTeX}
% \textsc{CTEX.ORG}.
% \newblock \textit{\CTeX{} 宏集手册}\allowbreak[EB/OL].
% \newblock version 2.5.10,
% \newblock (2022-07-14).
% \urlprefix\url{https://ctan.org/pkg/ctex}
% \urlprefix\CTANurl[文档及源代码:]{language/chinese/ctex/ctex.pdf}
%
% \bibitem{fduthesis}
% 曾祥东.
% \newblock \textit{fduthesis: 复旦大学论文模板}\allowbreak[EB/OL].
% \newblock version 0.9a,
% \newblock (2023-05-27).
% \urlprefix\url{https://ctan.org/pkg/fduthesis}
% \urlprefix\CTANurl[文档及源代码:]{macros/latex/contrib/fduthesis/fduthesis-code.pdf}
%
% \bibitem{thuthesis}
% 清华大学 TUNA 协会.
% \newblock \textit{\textsc{ThuThesis}:清华大学学位论文模板}\allowbreak[EB/OL].
% \newblock version 7.4.0,
% \newblock (2023-05-15).
% \urlprefix\url{https://ctan.org/pkg/thuthesis}
% \urlprefix\CTANurl[文档及源代码:]{macros/latex/contrib/thuthesis/thuthesis.pdf}
%
% \end{thebibliography}
%
% \end{documentation}
%
% \begin{implementation}
%
% \clearpage
% \section{代码实现}
%
% 本模板使用 \LaTeXiii{} 语法编写,依赖 \pkg{expl3} 环境,
% 并需调用 \pkg{l3packages} 中的相关宏包。
%
% \subsection{准备}
%
% \changes{unreleased}{2023/12/05}{同步 \LaTeX{} 2020/10/01,无需显式调用
% \pkg{expl3} 和 \pkg{xparse} 宏包。}
%    \begin{macrocode}
%<@@=sjtu>
%<*class>
%<thesis>\RequirePackage { xtemplate }
%    \end{macrocode}
%
% \subsection{内部变量}
%
% \begin{variable}{\l_@@_tmp_bool,\l_@@_tmp_clist,\l_@@_tmp_dim,
% \l_@@_tmp_skip,\l_@@_tmp_box}
% 临时变量。
%    \begin{macrocode}
%<thesis>\bool_new:N \l_@@_tmp_bool
%<thesis>\clist_new:N \l_@@_tmp_clist
\dim_new:N \l_@@_tmp_dim
\skip_new:N \l_@@_tmp_skip
\box_new:N \l_@@_tmp_box
%    \end{macrocode}
% \end{variable}
%
% \begin{variable}{\g_@@_thesis_type_int}
% 论文类型。
%    \begin{macrocode}
%<thesis>\int_new:N \g_@@_thesis_type_int
%    \end{macrocode}
% \end{variable}
%
% \begin{variable}{\g_@@_lang_tl,\g_@@_lang_clist}
% 论文语言。
%    \begin{macrocode}
\tl_new:N \g_@@_lang_tl
\tl_const:Nn \c_@@_lang_de_tl { de }
\tl_const:Nn \c_@@_lang_ja_tl { ja }
%<thesis>\clist_set:Nn \g_@@_lang_clist { zh, en }
%    \end{macrocode}
% \end{variable}
%
% \begin{variable}
%   {\g_@@_zihao_tl,\g_@@_font_size_int,\g_@@_font_size_dim,
%    \g_@@_baseline_skip_dim,\g_@@_fixed_baselineskip_bool,\g_@@_line_spread_fp}
% 字号大小与行距。
%    \begin{macrocode}
\tl_new:N \g_@@_zihao_tl
\int_new:N \g_@@_font_size_int
\dim_new:N \g_@@_font_size_dim
\dim_new:N \g_@@_baseline_skip_dim
\bool_new:N \g_@@_fixed_baselineskip_bool
\fp_new:N \g_@@_line_spread_fp
%    \end{macrocode}
% \end{variable}
%
% \begin{variable}
%   {\g_@@_text_font_tl,\g_@@_math_font_tl,\g_@@_cjk_font_tl}
% 字体配置。
%    \begin{macrocode}
\tl_new:N \g_@@_text_font_tl
\tl_new:N \g_@@_math_font_tl
\tl_new:N \g_@@_cjk_font_tl
\tl_new:N \g_@@_save_encodingdefault_tl
\tl_new:N \g_@@_save_rmdefault_tl
\tl_new:N \g_@@_save_sfdefault_tl
\tl_new:N \g_@@_save_ttdefault_tl
%    \end{macrocode}
% \end{variable}
%
% \begin{variable}{\g_@@_slanted_uppercase_greek_bool}
% 大写希腊字母的正/斜体。
%    \begin{macrocode}
\bool_new:N \g_@@_slanted_uppercase_greek_bool
%    \end{macrocode}
% \end{variable}
%
% \begin{variable}{\g_@@_upright_integral_bool}
% 积分号的正/斜体。
%    \begin{macrocode}
\bool_new:N \g_@@_upright_integral_bool
%    \end{macrocode}
% \end{variable}
%
% \begin{variable}{\g_@@_integral_limits_bool}
% 积分号上下限的位置。
%    \begin{macrocode}
\bool_new:N \g_@@_integral_limits_bool
%    \end{macrocode}
% \end{variable}
%
% \begin{variable}{\g_@@_math_font_options_clist}
% 传入数学字体宏包的选项列表。
%    \begin{macrocode}
\clist_new:N \g_@@_math_font_options_clist
%    \end{macrocode}
% \end{variable}
%
% \begin{variable}{\g_@@_review_bool}
% 盲审模式。
%    \begin{macrocode}
%<thesis>\bool_new:N \g_@@_review_bool
%    \end{macrocode}
% \end{variable}
%
% \begin{variable}{\g_@@_options_to_ctex_class_clist}
% 保存由 \pkg{sjtutex} 传入 \pkg{ctex} 文档类的选项列表。
% 默认 \pkg{ctex} 文档类的选项:
% 使用 UTF8 编码,不调整基础类的版式以及不载入 \pkg{ctex} 字体预设配置。
%    \begin{macrocode}
\clist_set:Nn \g_@@_options_to_ctex_class_clist
  { a4paper, UTF8, scheme = plain, fontset = none }
%    \end{macrocode}
% \end{variable}
%
% \begin{variable}{\g_@@_options_to_packages_clist}
% 保存由传入其他宏包的选项列表。
%    \begin{macrocode}
\clist_new:N \g_@@_options_to_packages_clist
%    \end{macrocode}
% \end{variable}
%
% \begin{variable}{\g_@@_twoside_bool}
% 是否开启双页模式。
%    \begin{macrocode}
\bool_new:N \g_@@_twoside_bool
%<thesis>\bool_set_true:N \g_@@_twoside_bool
%<!thesis>\bool_set_false:N \g_@@_twoside_bool
%    \end{macrocode}
% \end{variable}
%
% \begin{variable}{\g_@@_openright_bool}
% 是否在奇数页开始新章。
%    \begin{macrocode}
%<!article>\bool_new:N \g_@@_openright_bool
%<thesis>\bool_set_true:N \g_@@_openright_bool
%<report>\bool_set_false:N \g_@@_openright_bool
%    \end{macrocode}
% \end{variable}
%
% \begin{variable}{\g_@@_titlepage_bool}
% 是否生成标题页。
%    \begin{macrocode}
%<!thesis>\bool_new:N \g_@@_titlepage_bool
%<report>\bool_set_true:N \g_@@_titlepage_bool
%<article>\bool_set_false:N \g_@@_titlepage_bool
%    \end{macrocode}
% \end{variable}
%
% \begin{variable}{\g_@@_draft_bool}
% 是否开启草稿模式。
%    \begin{macrocode}
\bool_new:N \g_@@_draft_bool
%    \end{macrocode}
% \end{variable}
%
% \begin{macro}[int]{\@@_deprecated_option:n,\@@_set_deprecated_option:n}
% 对过时选项给出警告。
%    \begin{macrocode}
\cs_new_protected:Npn \@@_deprecated_option:n
  { \msg_warning:nnn { sjtutex } { deprecated-option } }
\cs_new_protected:Npn \@@_set_deprecated_option:n #1
  {
    \@@_deprecated_option:n { Option~ `#1'~ is~ set. }
    \keys_set:nn { sjtu / option } {#1}
  }
\msg_new:nnn { sjtutex } { deprecated-option }
  { Option~ `\l_keys_key_tl'~ is~ deprecated. \\ #1 }
%    \end{macrocode}
% \end{macro}
%
% \subsection{选项处理}
%
% 定义 |sjtu/option| 键值类。
%    \begin{macrocode}
\keys_define:nn { sjtu / option }
  {
%    \end{macrocode}
%
% \changes{v2.0}{2022/10/25}{\cls{sjtuthesis} 类型选项移除 \opt{course},
% 不再支持课程论文。}
% 论文类型。
%    \begin{macrocode}
%<*thesis>
    type .choice: ,
    type .value_required:n = true ,
    type .choices:nn =
      { bachelor, master, doctor }
      { \int_gset_eq:NN \g_@@_thesis_type_int \l_keys_choice_int } ,
    type .initial:n = { master } ,
%</thesis>
%    \end{macrocode}
%
% \changes{v2.0}{2023/03/19}{语言选项新增 \opt{ja},添加日文模板。}
% \changes{v2.0}{2023/03/23}{语言选项新增 \opt{de},添加德文模板。}
% 论文主要语言。
%    \begin{macrocode}
    lang .choice: ,
    lang .value_required:n = true ,
    lang .choices:nn =
      { zh, en, de, ja }
      {
        \tl_gset_eq:NN \g_@@_lang_tl \l_keys_choice_tl
%<*thesis>
        \clist_if_in:NnF \g_@@_lang_clist {#1}
          { \clist_gput_right:Nn \g_@@_lang_clist {#1} }
%</thesis>
      } ,
    lang .initial:n = { zh } ,
%    \end{macrocode}
%
% 字号大小。
%    \begin{macrocode}
    zihao .choice: ,
    zihao .value_required:n = true ,
    zihao /  5 .code:n =
      {
        \tl_gset:Nn  \g_@@_zihao_tl {#1}
        \int_gset:Nn \g_@@_font_size_int { 1 }
        \dim_gset:Nn \g_@@_font_size_dim { 10.5 bp }
      } ,
    zihao / -4 .code:n =
      {
        \tl_gset:Nn  \g_@@_zihao_tl {#1}
        \int_gset:Nn \g_@@_font_size_int { 2 }
        \dim_gset:Nn \g_@@_font_size_dim { 12   bp }
      } ,
%<!article>    zihao .initial:n = { -4 } ,
%<article>    zihao .initial:n = {  5 } ,
%    \end{macrocode}
%
% 行距倍数。
%    \begin{macrocode}
    linespread .fp_gset:N = \g_@@_line_spread_fp ,
    linespread .initial:n = { \c_nan_fp } ,
    linespread .value_required:n = true ,
%    \end{macrocode}
%
% \changes{v2.0}{2022/12/03}{添加 \opt{lineskip} 文档类选项。}
% \changes{unreleased}{2023/10/24}{添加 \opt{baselineskip} 文档类选项,
% 替换原 \opt{lineskip} 选项。}
% 基础行距。
%    \begin{macrocode}
    baselineskip .choice: ,
    baselineskip .value_required:n = true ,
    baselineskip / false   .code:n =
      { \bool_gset_false:N \g_@@_fixed_baselineskip_bool } ,
    baselineskip / unknown .code:n =
      {
        \bool_gset_true:N  \g_@@_fixed_baselineskip_bool
        \dim_gset:Nn \g_@@_baseline_skip_dim {#1}
      } ,
%<thesis>    baselineskip .initial:n = { 20 bp } ,
%<!thesis>    baselineskip .initial:n = { false } ,
%    \end{macrocode}
%
% 字体配置。
%    \begin{macrocode}
    text-font .tl_gset:N = \g_@@_text_font_tl ,
    text-font .initial:n = { newtx } ,
    math-font .tl_gset:N = \g_@@_math_font_tl ,
    cjk-font  .tl_gset:N = \g_@@_cjk_font_tl ,
%    \end{macrocode}
%
% \opt{fontset} 是过时选项。
%    \begin{macrocode}
    fontset .code:n = { \@@_set_deprecated_option:n { cjk-font = #1 } } ,
%    \end{macrocode}
%
% \changes{v2.0}{2023/01/05}{添加 \opt{math-style} 文档类选项,默认值为 \opt{ISO}。}
% 数学符号样式。
%    \begin{macrocode}
    math-style .choice: ,
    math-style .value_required:n = true,
    math-style / TeX .code:n =
      {
        \bool_gset_false:N \g_@@_slanted_uppercase_greek_bool
        \bool_gset_false:N \g_@@_upright_integral_bool
        \bool_gset_false:N \g_@@_integral_limits_bool
      } ,
    math-style / ISO .code:n =
      {
        \bool_gset_true:N \g_@@_slanted_uppercase_greek_bool
        \bool_gset_true:N \g_@@_upright_integral_bool
        \bool_gset_true:N \g_@@_integral_limits_bool
      } ,
    math-style .initial:n = { ISO } ,
%    \end{macrocode}
%
% 大写希腊字母的正/斜体。
%    \begin{macrocode}
    uppercase-greek .choice: ,
    uppercase-greek .value_required:n = true ,
    uppercase-greek / slanted .code:n =
      { \bool_gset_true:N  \g_@@_slanted_uppercase_greek_bool } ,
    uppercase-greek / upright .code:n =
      { \bool_gset_false:N \g_@@_slanted_uppercase_greek_bool } ,
%    \end{macrocode}
%
% 积分号的正/斜体。
%    \begin{macrocode}
    integral .choice: ,
    integral .value_required:n = true ,
    integral / slanted .code:n =
      { \bool_gset_false:N \g_@@_upright_integral_bool } ,
    integral / upright .code:n =
      { \bool_gset_true:N  \g_@@_upright_integral_bool } ,
%    \end{macrocode}
%
% 积分号上下限的位置。
%    \begin{macrocode}
    integral-limits .choice: ,
    integral-limits .value_required:n = true ,
    integral-limits / false .code:n =
      { \bool_gset_false:N \g_@@_integral_limits_bool } ,
    integral-limits / true  .code:n =
      { \bool_gset_true:N  \g_@@_integral_limits_bool } ,
%    \end{macrocode}
%
% 单面或双面模式。
%    \begin{macrocode}
    oneside .value_forbidden:n = true,
    twoside .value_forbidden:n = true,
    oneside .code:n =
      { \bool_gset_false:N \g_@@_twoside_bool } ,
    twoside .code:n =
      { \bool_gset_true:N  \g_@@_twoside_bool } ,
%    \end{macrocode}
%
% 是否奇数页开章。
%    \begin{macrocode}
%<*!article>
    openany   .value_forbidden:n = true,
    openright .value_forbidden:n = true,
    openany   .code:n =
      { \bool_gset_false:N \g_@@_openright_bool } ,
    openright .code:n =
      { \bool_gset_true:N  \g_@@_openright_bool } ,
%</!article>
%    \end{macrocode}
%
% 是否生成标题页。
%    \begin{macrocode}
%<*!thesis>
    titlepage   .value_forbidden:n = true,
    notitlepage .value_forbidden:n = true,
    titlepage   .code:n =
      { \bool_gset_true:N  \g_@@_titlepage_bool } ,
    notitlepage .code:n =
      { \bool_gset_false:N \g_@@_titlepage_bool } ,
%</!thesis>
%    \end{macrocode}
%
% 是否开启草稿模式。
%    \begin{macrocode}
    draft .value_forbidden:n = true,
    final .value_forbidden:n = true,
    draft .code:n =
      { \bool_gset_true:N  \g_@@_draft_bool } ,
    final .code:n =
      { \bool_gset_false:N \g_@@_draft_bool } ,
%    \end{macrocode}
%
% 盲审模式。
%    \begin{macrocode}
%<thesis>    review .bool_gset:N = \g_@@_review_bool ,
%<thesis>    review .initial:n = false ,
%    \end{macrocode}
%
% 处理未知选项。
%    \begin{macrocode}
    unknown .code:n = { \msg_error:nn { sjtutex } { unknown-option } }
  }
\msg_new:nnn { sjtutex } { unknown-option }
  { Class~ option~ "\l_keys_key_tl"~ is~ unknown. }
%    \end{macrocode}
%
% 将文档类选项传给 |sjtu/option|。
%    \begin{macrocode}
\cs_if_exist:NTF \ProcessKeyOptions
  { \ProcessKeyOptions [ sjtu / option ] }
  {
    \RequirePackage { l3keys2e }
    \ProcessKeysOptions { sjtu / option }
  }
%    \end{macrocode}
% \cls{sjtureport} 和 \cls{sjtuarticle} 文档类默认使用 1.3 行距倍数。
%    \begin{macrocode}
%<*!thesis>
\bool_if:NF \g_@@_fixed_baselineskip_bool
  {
    \fp_if_nan:nT { \g_@@_line_spread_fp }
      { \fp_set:Nn \g_@@_line_spread_fp { 1.3 } }
  }
%</!thesis>
%    \end{macrocode}
%
% 数字字体宏包选项。
%    \begin{macrocode}
\clist_set:Nx \g_@@_math_font_options_clist
  {
    \bool_if:NT \g_@@_slanted_uppercase_greek_bool
      { slantedGreek } ,
    \bool_if:NT \g_@@_upright_integral_bool
      { upint }
  }
%    \end{macrocode}
%
% 追加全局选项。
%    \begin{macrocode}
\clist_put_right:Nx \@classoptionslist
  {
    a4paper ,
    \tl_if_eq:NNT \g_@@_lang_tl \c_@@_lang_de_tl
      { german, ngerman } ,
    \bool_if:NT \g_@@_integral_limits_bool
      { intlimits } ,
    \g_@@_math_font_options_clist
  }
%    \end{macrocode}
%
% 设置传入 \pkg{ctex} 文档类的选项。
%    \begin{macrocode}
\clist_put_right:Nx \g_@@_options_to_ctex_class_clist
  {
    zihao      = \g_@@_zihao_tl ,
    \fp_if_nan:nF { \g_@@_line_spread_fp }
      { linespread = \fp_use:N \g_@@_line_spread_fp } ,
    \bool_if:NTF \g_@@_twoside_bool
      { twoside   } { oneside     } ,
%<!article>    \bool_if:NTF \g_@@_openright_bool
%<!article>      { openright } { openany     } ,
%<!thesis>    \bool_if:NTF \g_@@_titlepage_bool
%<!thesis>      { titlepage } { notitlepage } ,
    \bool_if:NTF \g_@@_draft_bool
      { draft     } { final       }
  }
%    \end{macrocode}
%
% \subsection{载入宏包、文档类}
%
% 将选项传入 \pkg{ctex} 文档类。
%    \begin{macrocode}
\exp_args:No \PassOptionsToClass
  { \g_@@_options_to_ctex_class_clist }
%<thesis>  { ctexbook }
%<report>  { ctexrep }
%<article>  { ctexart }
%    \end{macrocode}
%
% 传入各宏包选项。
%    \begin{macrocode}
\clist_set:Nx \g_@@_options_to_packages_clist
  {
    { no-math           } { fontspec     } ,
    { titles            } { tocloft      } ,
    { list = off        } { bicaption    } ,
    { warnings-off =
      {
        mathtools-overbracket,
        mathtools-colon
      }
    }                     { unicode-math } ,
    { amsmath, thmmarks } { ntheorem     } ,
%<!article>    { chapter           } { algorithm    } ,
%<!article>    { algochapter       } { algorithm2e  } ,
    {
      \bool_if:NTF \g_@@_integral_limits_bool
        { displaylimits } { nolimits }
    }                     { cmupint      }
  }
\clist_map_inline:Nn \g_@@_options_to_packages_clist
  { \PassOptionsToPackage #1 }
%    \end{macrocode}
%
% 载入 \pkg{ctex} 文档类。
% 在使用 \XeLaTeX{} 编译时,\pkg{ctex} 的底层将调用 \pkg{xeCJK}
% 宏包;而在使用 \LuaLaTeX{} 编译时,则将调用 \pkg{LuaTeX-ja} 宏包。
% 两种情况下 \pkg{ctex} 均会调用 \pkg{fontspec} 宏包。
%    \begin{macrocode}
%<thesis>\LoadClass { ctexbook }
%<report>\LoadClass { ctexrep }
%<article>\LoadClass { ctexart }
%    \end{macrocode}
%
% 载入各宏包。
%    \begin{macrocode}
\RequirePackage
  {
    mathtools,
    geometry,
    fancyhdr,
    tocloft,
%<thesis>    array,
    caption,
    bicaption,
    subcaption,
    xcolor,
    graphicx,
    enumitem
  }
%    \end{macrocode}
%
% \subsection{内部定义}
%
% \subsubsection{内部函数}
%
% \begin{macro}[int]{\cs_gset:cpo,\tl_const:Nv,
% \clist_use:NV,\clist_use:cv,
% \exp_args:NNnv,\regex_match:neTF}
% \LaTeX3{} 函数变体。
%    \begin{macrocode}
\cs_generate_variant:Nn \cs_gset:Npn { cpo }
\cs_generate_variant:Nn \tl_const:Nn { Nv }
\cs_generate_variant:Nn \clist_use:Nn { NV, cv }
\exp_args_generate:n { Nnv }
\cs_generate_variant:Nn \exp_last_unbraced:Ne { ce }
\prg_generate_conditional_variant:Nnn \regex_match:nn { ne } { T, TF }
%    \end{macrocode}
% \end{macro}
%
% \begin{macro}[int]{\@@_engine_case:nn}
% 2 个参数依次为 \pdfTeX 和 \XeTeX/\LuaTeX。
%    \begin{macrocode}
\cs_new:Npx \@@_engine_case:nn #1#2
  {
    \bool_lazy_or:nnTF
      { \sys_if_engine_xetex_p:  }
      { \sys_if_engine_luatex_p: }
      {#2}
      { \sys_if_engine_pdftex:T {#1} }
  }
%    \end{macrocode}
% \end{macro}
%
% \begin{macro}[int]{\@@_engine_case:nnn}
% 3 个参数依次为 \pdfTeX、 \XeTeX 和 \LuaTeX。
%    \begin{macrocode}
\cs_new:Npx \@@_engine_case:nnn #1#2#3
  {
    \sys_if_engine_xetex:TF
      {#2}
      {
        \sys_if_engine_luatex:TF
          {#3}
          { \sys_if_engine_pdftex:T {#1} }
      }
  }
%    \end{macrocode}
% \end{macro}
%
% \begin{macro}[int]{\@@_unicode_engine_case:nn}
% 2 个参数依次为 \XeTeX 和 \LuaTeX。
%    \begin{macrocode}
\cs_new:Npx \@@_unicode_engine_case:nn #1#2
  {
    \sys_if_engine_xetex:TF
      {#1}
      { \sys_if_engine_luatex:T {#2} }
  }
%    \end{macrocode}
% \end{macro}
%
% \begin{macro}[int]{\@@_unicode_char:n}
%    \begin{macrocode}
\@@_engine_case:nn
  {
    \cs_new:Npn \@@_unicode_char:n #1
      {
        \exp_not:N \Unicode
          { \int_div_truncate:nn {#1} { 256 } }
          { \int_mod:nn          {#1} { 256 } }
      }
  }
  { \cs_new:Npn \@@_unicode_char:n #1 { \tex_Uchar:D #1 \scan_stop: } }
%    \end{macrocode}
% \end{macro}
%
% \begin{macro}[int]{\@@_preto_cmd:Nn,\@@_appto_cmd:Nn}
% 补丁工具,来自 \pkg{ctexpatch} 宏包,在宏的原本定义前后增加钩子。
%    \begin{macrocode}
%<*!thesis>
\cs_new_protected:Npn \@@_preto_cmd:Nn #1#2
  {
    \ctex_preto_cmd:NnnTF #1 { } {#2}
      { } { \ctex_patch_failure:N #1 }
  }
\cs_new_protected:Npn \@@_appto_cmd:Nn #1#2
  {
    \ctex_appto_cmd:NnnTF #1 { } {#2}
      { } { \ctex_patch_failure:N #1 }
  }
%</!thesis>
%    \end{macrocode}
% \end{macro}
%
% \changes{v2.0.2}{2023/03/31}{区分 \texttt{dim} 与 \texttt{skip} 类型变量。}
% \begin{macro}[int]{\@@_dim_set_to_wd:Nn,\@@_skip_add_to_wd:Nn}
% 操作长度变量的辅助函数。
%    \begin{macrocode}
%<*thesis>
\cs_new:Npn \@@_dim_set_to_wd:Nn #1#2
  {
    \hbox_set:Nn \l_@@_tmp_box {#2}
    \dim_set:Nn #1 { \box_wd:N \l_@@_tmp_box }
  }
%</thesis>
\cs_new:Npn \@@_skip_add_to_wd:Nn #1#2
  {
    \hbox_set:Nn \l_@@_tmp_box {#2}
    \skip_add:Nn #1 { \box_wd:N \l_@@_tmp_box }
  }
%<thesis>\cs_generate_variant:Nn \@@_dim_set_to_wd:Nn { Nv }
\cs_generate_variant:Nn \@@_skip_add_to_wd:Nn { cv }
%    \end{macrocode}
% \end{macro}
%
% \begin{macro}[int]{\@@_cs_provide_eq:NN}
%    \begin{macrocode}
\cs_new:Npn \@@_cs_provide_eq:NN #1#2
  { \cs_if_exist:NF #1 { \cs_set_eq:NN #1 #2 } }
\cs_generate_variant:Nn \@@_cs_provide_eq:NN { cc }
%    \end{macrocode}
% \end{macro}
%
% \begin{macro}[int]{\@@_vspace:N,\@@_vspace:n,
% \@@_vspace_r:N,\@@_vspace_r:n}
% 类似 \LaTeXe{} 中的 \tn{vspace} 和 \tn{vspace*}。
%    \begin{macrocode}
%<*thesis>
\cs_new_protected:Npn \@@_vspace:N #1
  {
    \skip_vertical:N #1
    \skip_vertical:N \c_zero_skip
  }
\cs_new_protected:Npn \@@_vspace:n #1
  {
    \skip_set:Nn \l_@@_tmp_skip {#1}
    \@@_vspace:N \l_@@_tmp_skip
  }
\cs_new_protected:Npn \@@_vspace_r:N #1
  {
    \dim_set_eq:NN \l_@@_tmp_dim \prevdepth
    \hrule height \c_zero_dim
    \nobreak
    \skip_vertical:N #1
    \skip_vertical:N \c_zero_skip
    \dim_set_eq:NN \prevdepth \l_@@_tmp_dim
  }
\cs_new_protected:Npn \@@_vspace_r:n #1
  {
    \skip_set:Nn   \l_@@_tmp_skip {#1}
    \@@_vspace_r:N \l_@@_tmp_skip
  }
%</thesis>
%    \end{macrocode}
% \end{macro}
%
% \begin{macro}[int]{\@@_define_name:nn,\@@_define_name:nv,\@@_define_name:nnn,
% \@@_define_name_from_clist:nnnn}
% 定义默认名称的辅助函数。
%    \begin{macrocode}
\cs_new_protected:Npn \@@_define_name:nn #1#2
  { \tl_const:cn { c_@@_name_ #1 _tl } {#2} }
\cs_new_protected:Npn \@@_define_name:nnn #1#2#3
  { \tl_const:cn { c_@@_name_ #2 _ #1 _tl } {#3} }
%<*thesis>
\cs_new_protected:Npn \@@_define_name_from_clist:nnnn #1#2#3#4
  { \tl_const:cx { c_@@_name_ #2 _ #1 _tl } { \clist_item:nn {#4} {#3} } }
\cs_generate_variant:Nn \@@_define_name:nn { nv }
%</thesis>
%    \end{macrocode}
% \end{macro}
%
% \begin{macro}[int]{\@@_define_symbol:nn}
% 定义符号的辅助函数。
%    \begin{macrocode}
\cs_new_protected:Npn \@@_define_symbol:nn #1#2
  { \tl_const:cx { c_@@_symbol_ #1 _tl } { \@@_unicode_char:n {#2} } }
%    \end{macrocode}
% \end{macro}
%
% \begin{macro}[int]{\@@_if_lang_valid:nTF}
% 验证语言选项的函数。
%    \begin{macrocode}
%<*thesis>
\cs_new_protected:Npn \@@_if_lang_valid:nTF #1
  { \clist_if_in:NnTF \g_@@_lang_clist {#1} }
\msg_new:nnn { sjtutex } { lang-validation }
  { Invalid~ language~ argument~ `#1'! }
%    \end{macrocode}
% \end{macro}
%
% \subsubsection{页面模板}
%
% 使用 \pkg{xtemplate} 构建页面模板,用于绘制标题页与版权页。
%
% 页面部件模板。
%    \begin{macrocode}
\DeclareObjectType { sjtu } { 0 }
\DeclareTemplateInterface { sjtu } { component } { 0 }
  {
    format      : tokenlist = \c_empty_tl ,
    content     : tokenlist = \c_empty_tl ,
    bottom-skip : skip      = \c_zero_skip ,
    align       : choice { left, right, center, normal } = center
  }
\DeclareTemplateCode { sjtu } { component } { 0 }
  {
    format      = \l_@@_component_format_tl ,
    content     = \l_@@_component_content_tl ,
    bottom-skip = \l_@@_component_bottom_skip ,
    align       =
      {
        left    =
          \cs_set_eq:NN \l_@@_component_align: \raggedright ,
        right   =
          \cs_set_eq:NN \l_@@_component_align: \raggedleft ,
        center  =
          \cs_set_eq:NN \l_@@_component_align: \centering ,
        normal  =
          \cs_set_eq:NN \l_@@_component_align: \prg_do_nothing:
      }
  }
  {
    \AssignTemplateKeys
    \group_begin:
      \l_@@_component_align:
      \l_@@_component_format_tl
      \l_@@_component_content_tl
      \par
    \group_end:
    \@@_vspace:N \l_@@_component_bottom_skip
  }
%    \end{macrocode}
%
% 页面模板。
%    \begin{macrocode}
\DeclareTemplateInterface { sjtu } { page } { 0 }
  {
    bookmark      : boolean   = false ,
    bookmark-text : tokenlist = \c_empty_tl ,
    style         : tokenlist = empty ,
    format        : tokenlist = \c_empty_tl ,
    prefix        : tokenlist ,
    components    : commalist ,
    top-skip      : skip      = \c_zero_skip ,
    bottom-skip   : skip      = \c_zero_skip
  }
\DeclareTemplateCode { sjtu } { page } { 0 }
  {
    bookmark      = \l_@@_page_bookmark_bool ,
    bookmark-text = \l_@@_page_bookmark_text_tl ,
    style         = \l_@@_page_style_tl ,
    format        = \l_@@_page_format_tl ,
    prefix        = \l_@@_page_prefix_tl ,
    components    = \l_@@_page_components_clist ,
    top-skip      = \l_@@_page_top_skip ,
    bottom-skip   = \l_@@_page_bottom_skip
  }
  {
    \AssignTemplateKeys
    \bool_if:NTF \g_@@_openright_bool
      { \cleardoublepage } { \clearpage }
    \bool_if:NT \l_@@_page_bookmark_bool
      { \@@_pdf_bookmark:nn { 0 } { \l_@@_page_bookmark_text_tl } }
    \exp_args:No \thispagestyle { \l_@@_page_style_tl }
%    \end{macrocode}
% 移除页面顶部 \tn{vspace*} 的多余空白。
% 见 \url{https://tex.stackexchange.com/questions/247513}。
%    \begin{macrocode}
    \@@_vspace_r:N \l_@@_page_top_skip
    \@@_vspace:n { - \tex_parskip:D      }
    \@@_vspace:n { - \tex_baselineskip:D }
    \group_begin:
      \l_@@_page_format_tl
      \clist_map_inline:Nn \l_@@_page_components_clist
        { \UseInstance { sjtu } { \l_@@_page_prefix_tl / ##1 } }
    \group_end:
    \@@_vspace:N \l_@@_page_bottom_skip
    \clearpage
  }
%    \end{macrocode}
%
% 辅助函数。
%    \begin{macrocode}
\cs_new:Npn \@@_declare_component:nnn #1#2#3
  { \DeclareInstance { sjtu } {#1/#2} { component } {#3} }
\cs_new:Npn \@@_declare_page:nn #1#2
  { \DeclareInstance { sjtu } {#1} { page } {#2} }
%</thesis>
%    \end{macrocode}
%
% \subsection{字号行距}
%
% 重定义 \tn{normalsize},设置正文的基线间距。
%    \begin{macrocode}
\cs_new_protected:Npn \@@_set_font_size:nnNn #1#2#3#4
  { \cs_set_protected:Npn #3 { \@setfontsize #3 {#1} {#2} #4 } }
\tl_set:Nx \l_@@_font_size_tl
  {
    { \dim_to_decimal:n { \g_@@_font_size_dim     } }
    { \dim_to_decimal:n { \g_@@_baseline_skip_dim } }
  }
\bool_if:NT \g_@@_fixed_baselineskip_bool
  {
    \int_case:nn { \g_@@_font_size_int }
      {
        { 1 } {
                \exp_after:wN \@@_set_font_size:nnNn \l_@@_font_size_tl
                  \normalsize
                  {
                    \abovedisplayskip 10\p@ \@plus2\p@ \@minus5\p@
                    \abovedisplayshortskip \z@ \@plus3\p@
                    \belowdisplayshortskip 6\p@ \@plus3\p@ \@minus3\p@
                    \belowdisplayskip \abovedisplayskip
                    \let\@listi\@listI
                  }
              }
        { 2 } {
                \exp_after:wN \@@_set_font_size:nnNn \l_@@_font_size_tl
                  \normalsize
                  {
                    \abovedisplayskip 12\p@ \@plus3\p@ \@minus7\p@
                    \abovedisplayshortskip \z@ \@plus3\p@
                    \belowdisplayshortskip 6.5\p@ \@plus3.5\p@ \@minus3\p@
                    \belowdisplayskip \abovedisplayskip
                    \let\@listi\@listI
                  }
              }
      }
    \normalsize
  }
%    \end{macrocode}
%
% \begin{macro}{\setbaselineskip}
% 设置基线间距,在字号命令之后使用。
%    \begin{macrocode}
\NewDocumentCommand \setbaselineskip { m }
  { \fontsize { \f@size } {#1} \selectfont }
%    \end{macrocode}
% \end{macro}
%
% \subsection{字体配置}
%
% \begin{macro}[int]{\@@_fontset_error:nn}
% 字库不可用时给出紧急错误信息,停止读取定义文件。
%    \begin{macrocode}
\cs_new_protected:Npn \@@_fontset_error:nn #1#2
  { \msg_error:nnnn { sjtutex } { font-unavailable } {#1} {#2} }
\msg_new:nnn { sjtutex } { font-unavailable }
  { `#1-font~ =~ #2'~ is~ unavailable~ in~ current~ mode. }
%    \end{macrocode}
% \end{macro}
%
% \begin{macro}[int]{\@@_fontset_case:nn}
% 2 个参数依次为 \pdfTeX 和 \XeTeX/\LuaTeX。
%    \begin{macrocode}
\cs_new_eq:NN \@@_fontset_case:nn \@@_engine_case:nn
%    \end{macrocode}
% \end{macro}
%
% \begin{macro}[int]{\@@_fontset_case:nnn}
% 3 个参数依次为 \pdfTeX(生成 PDF)、\pdfTeX(生成 DVI) 和
% \XeTeX/\LuaTeX。
%    \begin{macrocode}
\cs_new:Npx \@@_fontset_case:nnn #1#2#3
  {
    \@@_engine_case:nn
      { \sys_if_output_pdf:TF {#1} {#2} }
      {#3}
  }
%    \end{macrocode}
% \end{macro}
%
% \begin{macro}[int]{\@@_set_slanted_greek:}
%    \begin{macrocode}
\cs_new_protected:Nn \@@_set_slanted_greek:
  {
    \clist_const:Nn \c_@@_uppercase_greek_clist
      { Gamma, Delta, Theta, Lambda, Xi, Pi, Sigma, Upsilon, Phi, Psi, Omega }
    \clist_map_inline:Nn \c_@@_uppercase_greek_clist
      {
        \cs_set_eq:cc { up ##1 } {     ##1 }
        \cs_set_eq:cc { it ##1 } { var ##1 }
      }
    \bool_if:NT \g_@@_slanted_uppercase_greek_bool
      {
        \clist_map_inline:Nn \c_@@_uppercase_greek_clist
          { \cs_set_eq:cc { ##1 } { it ##1 } }
      }
  }
%    \end{macrocode}
% \end{macro}
%
% \begin{macro}[int]{\@@_set_unimath_symbol:}
%    \begin{macrocode}
\cs_new_protected:Nn \@@_set_unimath_symbol:
  {
    \clist_map_inline:nn
      {
        { increment } { upDelta     } ,
        { QED       } { blacksquare }
      }
      { \@@_cs_provide_eq:cc ##1 }
  }
%    \end{macrocode}
% \end{macro}
%
% 如果没有指定数学字体,则根据西文字体设置匹配的数字字体。
%    \begin{macrocode}
\tl_if_empty:NT \g_@@_math_font_tl
  { \tl_gset_eq:NN \g_@@_math_font_tl \g_@@_text_font_tl }
%    \end{macrocode}
%
% 根据操作系统判断默认 CJK 字体配置。
%    \begin{macrocode}
\tl_if_empty:NT \g_@@_cjk_font_tl
  {
    \sys_if_platform_windows:TF
      { \tl_gset:Nn \g_@@_cjk_font_tl { windows } }
      {
        \ctex_if_platform_macos:TF
          { \tl_gset:Nn \g_@@_cjk_font_tl { mac    } }
          { \tl_gset:Nn \g_@@_cjk_font_tl { fandol } }
      }
  }
%    \end{macrocode}
%
% \begin{macro}[int]{\@@_load_font:nn,\@@_load_fontset:}
% 如果字体配置文件不存在,则载入默认值,并给出警告。
%    \begin{macrocode}
\cs_new_protected:Npn \@@_load_font:nn #1#2
  {
    \str_if_eq:eeF { \tl_use:c { g_@@_ #1 _font_tl } } { none }
      {
        \file_if_exist:nF
          { sjtu- #1 -font- \tl_use:c { g_@@_ #1 _font_tl } .def }
          {
            \msg_warning:nnnn { sjtutex } { invalid-font } {#1} {#2}
            \tl_gset:cn { g_@@_ #1 _font_tl } {#2}
          }
        \ctex_file_input:n
          { sjtu- #1 -font- \tl_use:c { g_@@_ #1 _font_tl } .def }
      }
  }
\msg_new:nnn { sjtutex } { invalid-font }
  {
    Invalid~ value~ `#1-font~ =~ \tl_use:c { g_@@_ #1 _font_tl }~ '! \\\\
    Using~ `#2'~ instead.
  }
\cs_new_protected:Nn \@@_load_fontset:
  {
    \clist_map_inline:nn
      {
        { math } { newtx  },
        { text } { newtx  },
        { cjk  } { fandol }
      }
      { \@@_load_font:nn ##1 }
  }
\@onlypreamble \@@_load_font:nn
\@onlypreamble \@@_load_fontset:
%</class>
%    \end{macrocode}
% \end{macro}
%
% \subsubsection{西文与数学字体}
%
% \changes{v2.0.3}{2023/09/25}{新增 \opt{libertinus} 字体配置。}
%    \begin{macrocode}
%<*font&(math|text)>
%<*stixtwo>
\@@_fontset_case:nn
  {
%<*math>
    \PassOptionsToPackage { notext } { stix2 }
    \RequirePackage { upgreek, stix2 }
    \@@_set_slanted_greek:
%</math>
%<*text>
    \tl_set:Nn \encodingdefault { T1 }
    \tl_set:Nn \rmdefault { stix2 }
    \tl_set:Nn \qhv@scale { 0.94 }
    \tl_set:Nn \sfdefault { qhv }
    \tl_set:Nn \ttdefault { qcr }
%</text>
  }
  {
%<*math>
    \RequirePackage { unicode-math }
    \bool_if:NTF \g_@@_upright_integral_bool
      {
        \setmathfont { STIXTwoMath-Regular.otf }
          [ StylisticSet = 8 ]
      }
      { \setmathfont { STIXTwoMath-Regular.otf } }
    \setmathfont { STIXTwoMath-Regular.otf }
      [
        range        = { scr, bfscr },
        StylisticSet = 1
      ]
%</math>
%<math>    \setmathrm
%<text>    \setmainfont
      { STIXTwoText }
      [
        Extension      = .otf,
        UprightFont    = *-Regular,
        BoldFont       = *-Bold,
        ItalicFont     = *-Italic,
        BoldItalicFont = *-BoldItalic
      ]
%</stixtwo>
%<*xits>
\@@_fontset_case:nn
%<math>  { \@@_fontset_error:nn { math } { xits } }
%<text>  { \@@_fontset_error:nn { text } { xits } }
  {
%<*math>
    \RequirePackage { unicode-math }
    \bool_if:NTF \g_@@_upright_integral_bool
      {
        \setmathfont { XITSMath-Regular }
          [
            Extension    = .otf,
            BoldFont     = XITSMath-Bold,
            StylisticSet = 8
          ]
      }
      {
        \setmathfont { XITSMath-Regular }
          [
            Extension    = .otf,
            BoldFont     = XITSMath-Bold,
          ]
      }
    \setmathfont { XITSMath-Regular.otf }
      [
        range        = { cal, bfcal },
        StylisticSet = 1
      ]
%</math>
%<math>    \setmathrm
%<text>    \setmainfont
      { XITS }
      [
        Extension      = .otf,
        UprightFont    = *-Regular,
        BoldFont       = *-Bold,
        ItalicFont     = *-Italic,
        BoldItalicFont = *-BoldItalic
      ]
%</xits>
%<*newtx|newpx>
%<*math>
\tl_set_eq:NN \g_@@_save_encodingdefault_tl \encodingdefault
\tl_set_eq:NN \g_@@_save_rmdefault_tl \rmdefault
\tl_set_eq:NN \g_@@_save_sfdefault_tl \sfdefault
\tl_set_eq:NN \g_@@_save_ttdefault_tl \ttdefault
\tl_set:Nn \encodingdefault { OT1 }
%<newtx>\tl_set:Nn \rmdefault { ntxtlf }
%<newpx>\tl_set:Nn \rmdefault { zplTLF }
\tl_set:Nn \qhv@scale { 0.94 }
\tl_set:Nn \sfdefault { qhv }
\tl_set:Nn \ttdefault { qcr }
%<newtx>\RequirePackage { newtxmath }
%<newpx>\RequirePackage { newpxmath }
\tl_set_eq:NN \encodingdefault \g_@@_save_encodingdefault_tl
\tl_set_eq:NN \rmdefault \g_@@_save_rmdefault_tl
\tl_set_eq:NN \sfdefault \g_@@_save_sfdefault_tl
\tl_set_eq:NN \ttdefault \g_@@_save_ttdefault_tl
\@@_set_unimath_symbol:
%</math>
%<*text>
\@@_fontset_case:nn
  {
    \tl_set:Nn \encodingdefault { T1 }
%<newtx>    \RequirePackage { newtxtext }
%<newpx>    \RequirePackage { newpxtext }
    \tl_set:Nn \ttdefault { qcr }
  }
  {
    \setmainfont
%<newtx>      { TeXGyreTermesX }
%<newpx>      { TeXGyrePagellaX }
      [
        Extension       = .otf,
        UprightFont     = *-Regular,
        BoldFont        = *-Bold,
        ItalicFont      = *-Italic,
        BoldItalicFont  = *-BoldItalic
      ]
%</text>
%</newtx|newpx>
%<*text&(newtx|newpx)|stixtwo|xits>
%<math>    \setmathsf
%<text>    \setsansfont
      { texgyreheros }
      [
        Extension      = .otf,
        UprightFont    = *-regular,
        BoldFont       = *-bold,
        ItalicFont     = *-italic,
        BoldItalicFont = *-bolditalic,
        Scale          = 0.94,
      ]
%<math>    \setmathtt
%<text>    \setmonofont
      { texgyrecursor }
      [
        Extension      = .otf,
        UprightFont    = *-regular,
        BoldFont       = *-bold,
        ItalicFont     = *-italic,
        BoldItalicFont = *-bolditalic,
        Ligatures      = CommonOff
      ]
  }
%</text&(newtx|newpx)|stixtwo|xits>
%<*lm>
%<*text>
\@@_fontset_case:nn
  {
    \tl_set:Nn \encodingdefault { T1 }
    \tl_set:Nn \rmdefault { lmr  }
    \tl_set:Nn \sfdefault { lmss }
    \tl_set:Nn \ttdefault { lmtt }
  } { }
%</text>
%<*math>
\RequirePackage { amssymb, upgreek }
\SetSymbolFont { operators    } { normal } { OT1 } { lmr  } { m  } { n  }
\SetSymbolFont { letters      } { normal } { OML } { lmm  } { m  } { it }
\SetSymbolFont { symbols      } { normal } { OMS } { lmsy } { m  } { n  }
\SetSymbolFont { largesymbols } { normal } { OMX } { lmex } { m  } { n  }
\SetSymbolFont { operators    } { bold   } { OT1 } { lmr  } { bx } { n  }
\SetSymbolFont { letters      } { bold   } { OML } { lmm  } { b  } { it }
\SetSymbolFont { symbols      } { bold   } { OMS } { lmsy } { b  } { n  }
\SetSymbolFont { largesymbols } { bold   } { OMX } { lmex } { m  } { n  }
\SetMathAlphabet { \mathbf } { normal } { OT1 } { lmr  } { bx } { n  }
\SetMathAlphabet { \mathsf } { normal } { OT1 } { lmss } { m  } { n  }
\SetMathAlphabet { \mathit } { normal } { OT1 } { lmr  } { m  } { it }
\SetMathAlphabet { \mathtt } { normal } { OT1 } { lmtt } { m  } { n  }
\SetMathAlphabet { \mathbf } { bold   } { OT1 } { lmr  } { bx } { n  }
\SetMathAlphabet { \mathsf } { bold   } { OT1 } { lmss } { bx } { n  }
\SetMathAlphabet { \mathit } { bold   } { OT1 } { lmr  } { bx } { it }
\SetMathAlphabet { \mathtt } { bold   } { OT1 } { lmtt } { m  } { n  }
\bool_if:NT \g_@@_upright_integral_bool
  { \RequirePackage { cmupint } }
\@@_set_slanted_greek:
\@@_set_unimath_symbol:
%</math>
%</lm>
%<*libertinus>
\@@_fontset_case:nn
  {
%<*text>
    \tl_set:Nn \encodingdefault { T1 }
    \tl_set:Nn \rmdefault { LibertinusSerif-TLF }
    \tl_set:Nn \sfdefault { LibertinusSans-TLF  }
    \tl_set:Nn \ttdefault { lmtt                }
%</text>
%<*math>
    \exp_args:No \PassOptionsToPackage
      { \g_@@_math_font_options_clist } { libertinust1math }
    \RequirePackage { libertinust1math }
%</math>
  }
  {
%<*math>
    \RequirePackage { unicode-math }
    \bool_if:NTF \g_@@_upright_integral_bool
      { \setmathfont { LibertinusMath-Regular.otf } }
      {
        \setmathfont { LibertinusMath-Regular.otf }
          [ StylisticSet = 8 ]
      }
    \setmathfont { latinmodern-math.otf } [ range = \checkmark ]
%</math>
%<math>    \setmathrm
%<text>    \setmainfont
      { LibertinusSerif }
      [
        Extension           = .otf,
        UprightFont         = *-Regular,
        BoldFont            = *-Bold,
        ItalicFont          = *-Italic,
        BoldItalicFont      = *-BoldItalic,
        SlantedFont         = *-Regular,
        BoldSlantedFont     = *-Bold,
        SlantedFeatures     = { FakeSlant = 0.2 },
        BoldSlantedFeatures = { FakeSlant = 0.2 }
      ]
%<math>    \setmathsf
%<text>    \setsansfont
      { LibertinusSans }
      [
        Extension           = .otf,
        UprightFont         = *-Regular,
        BoldFont            = *-Bold,
        ItalicFont          = *-Italic,
        BoldItalicFont      = *-Italic,
        BoldItalicFeatures  = { FakeBold  = 3   },
        SlantedFont         = *-Regular,
        BoldSlantedFont     = *-Bold,
        SlantedFeatures     = { FakeSlant = 0.2 },
        BoldSlantedFeatures = { FakeSlant = 0.2 }
      ]
  }
%</libertinus>
%<*times>
%<*math>
\RequirePackage { amssymb, upgreek }
\tl_set_eq:NN \g_@@_save_rmdefault_tl \rmdefault
  \RequirePackage { mathptmx }
\tl_set_eq:NN \rmdefault \g_@@_save_rmdefault_tl
\tl_set:Nn \Hv@scale { 0.94 }
\DeclareMathAlphabet { \mathsf } { OT1 } { phv } { m } { n }
\DeclareMathAlphabet { \mathtt } { OT1 } { pcr } { m } { n }
\SetMathAlphabet { \mathsf } { bold } { OT1 } { phv } { b } { n }
\SetMathAlphabet { \mathtt } { bold } { OT1 } { pcr } { b } { n }
\bool_if:NT \g_@@_upright_integral_bool
  { \RequirePackage { cmupint } }
\@@_set_unimath_symbol:
%</math>
%<*text>
\@@_fontset_case:nn
  {
    \tl_set:Nn \encodingdefault { T1 }
    \tl_set:Nn \rmdefault { ptm }
    \tl_set:Nn \Hv@scale { 0.94 }
    \tl_set:Nn \sfdefault { phv }
    \tl_set:Nn \ttdefault { pcr }
  }
  {
    \setmainfont { Times~New~Roman } [ Ligatures = Rare ]
    \setsansfont { Arial } [ Scale = 0.94 ]
    \setmonofont { Courier~New }
  }
%</text>
%</times>
%<*newcm>
\@@_fontset_case:nn
%<math>  { \@@_fontset_error:nn { math } { newcm } }
%<text>  { \@@_fontset_error:nn { text } { newcm } }
  {
%<*math>
    \RequirePackage { unicode-math }
    \bool_if:NTF \g_@@_upright_integral_bool
      {
        \setmathfont { NewCMMath-Book.otf }
          [ StylisticSet = 2 ]
      }
      { \setmathfont { NewCMMath-Book.otf } }
    \setmathfont { NewCMMath-Book.otf }
      [
        range        = { scr, bfscr },
        StylisticSet = 1
      ]
%</math>
%<math>    \setmathrm
%<text>    \setmainfont
      { NewCM10 }
      [
        Extension    = .otf,
        SizeFeatures =
          {
            {
              Size        = -9,
              Font        = NewCM08-Book,
              ItalicFont  = NewCM08-BookItalic,
              SlantedFont = NewCM08-Book,
            },
            { Size        = 9- }
          },
        UprightFont         = *-Book,
        BoldFont            = *-Bold,
        ItalicFont          = *-BookItalic,
        BoldItalicFont      = *-BoldItalic,
        SlantedFont         = *-Book,
        BoldSlantedFont     = *-Bold,
        SlantedFeatures     = { FakeSlant = 0.25 },
        BoldSlantedFeatures = { FakeSlant = 0.25 }
      ]
%<math>    \setmathsf
%<text>    \setsansfont
      { NewCMSans10 }
      [
        Extension    = .otf,
        SizeFeatures =
          {
            {
              Size       = -9,
              Font       = NewCMSans08-Book,
              ItalicFont = NewCMSans08-BookOblique,
            },
            { Size       = 9- }
          },
        UprightFont    = *-Book,
        BoldFont       = *-Bold,
        ItalicFont     = *-BookOblique,
        BoldItalicFont = *-BoldOblique
      ]
%<math>    \setmathtt
%<text>    \setmonofont
      { NewCMMono10 }
      [
        Extension           = .otf,
        UprightFont         = *-Book,
        BoldFont            = *-Bold,
        ItalicFont          = *-BookItalic,
        BoldItalicFont      = *-BoldOblique,
        SlantedFont         = *-Book,
        SlantedFeatures     = { FakeSlant = 0.25 },
        BoldSlantedFont     = *-Bold,
        BoldSlantedFeatures = { FakeSlant = 0.25 }
      ]
  }
%</newcm>
%<*cambria>
\@@_fontset_case:nn
%<math>  { \@@_fontset_error:nn { math } { cambria } }
%<text>  { \@@_fontset_error:nn { text } { cambria } }
  {
%<*math>
    \RequirePackage { unicode-math }
    \setmathfont { Cambria~Math }
    \setmathrm { Cambria }
    \setmathsf { Calibri }
    \setmathtt { Consolas } [ Scale = 0.95 ]
%</math>
%<*text>
    \setmainfont { Cambria }
    \setsansfont { Calibri }
    \setmonofont { Consolas } [ Scale = 0.95 ]
%</text>
  }
%</cambria>
%</font&(math|text)>
%    \end{macrocode}
%
% \pkg{unicode-math} 宏包设置。
%    \begin{macrocode}
%<*class>
\ctex_at_end_package:nn { unicode-math }
  {
    \DeclareDocumentCommand \bm { m }
      { { \symbf {#1} } }
    \DeclareDocumentCommand \boldsymbol { m }
      { { \symbf {#1} } }
    \bool_if:NTF \g_@@_slanted_uppercase_greek_bool
      { \keys_set:nn { unicode-math } { math-style = ISO } }
      { \keys_set:nn { unicode-math } { math-style = TeX } }
    \bool_if:NTF \g_@@_integral_limits_bool
      { \removenolimits } { \addnolimits }
      {
        \int\iint\iiint\iiiint\oint\oiint\oiiint
        \intclockwise\varointclockwise\ointctrclockwise\sumint
        \intbar\intBar\fint\cirfnint\awint\rppolint
        \scpolint\npolint\pointint\sqint\intlarhk\intx
        \intcap\intcup\upint\lowint
      }
  }
%    \end{macrocode}
%
% 若未使用 \pkg{unicode-math} 配置数学字体,则自动调用 \pkg{bm}。
%    \begin{macrocode}
\ctex_at_end_preamble:n
  {
    \@ifpackageloaded { unicode-math }
      { } { \RequirePackage { bm } }
  }
%</class>
%    \end{macrocode}
%
% \subsubsection{CJK 字体}
%
% 在字体未提供对应粗体的情况下,允许使用伪粗。
%    \begin{macrocode}
%<*font&cjk>
%<*windows>
%<*und>
\tl_if_eq:NNTF \g_@@_lang_tl \c_@@_lang_ja_tl
  { \ctex_file_input:n { sjtu-cjk-font-windows-ja.def } }
  { \ctex_file_input:n { sjtu-cjk-font-windows-zh.def } }
%</und>
%<*zh|ja>
\@@_fontset_case:nn
%<*zh>
  {
    \ctex_load_zhmap:nnnn { zhsong } { zhhei } { zhfs } { windows }
    \ctex_punct_set:n { windows }
    \ctex_punct_map_family:nn   { \CJKrmdefault } { zhsong }
    \ctex_punct_map_bfseries:nn { \CJKrmdefault } { zhhei  }
    \ctex_punct_map_itshape:nn  { \CJKrmdefault } { zhkai  }
  }
%</zh>
%<ja>  { \@@_fontset_error:nn { cjk } { windows } }
  {
%<*zh>
    \setCJKmainfont { SimSun   }
      [ AutoFakeBold = 3 , ItalicFont = KaiTi ]
    \setCJKsansfont { SimHei   } [ AutoFakeBold = 3 ]
    \setCJKmonofont { FangSong }
%</zh>
%<*ja>
    \setCJKmainfont { MS~Mincho } [ AutoFakeBold = 3 ]
    \setCJKsansfont { MS~Gothic } [ AutoFakeBold = 3 ]
    \setCJKmonofont { MS~Gothic }
    \setCJKfamilyfont { jamin  } { MS~Mincho } [ AutoFakeBold = 3 ]
    \setCJKfamilyfont { jagoth } { MS~Gothic } [ AutoFakeBold = 3 ]
%</ja>
    \setCJKfamilyfont { zhsong } { SimSun   }
      [ AutoFakeBold = 3 , ItalicFont = KaiTi ]
    \setCJKfamilyfont { zhhei  } { SimHei   } [ AutoFakeBold = 3 ]
    \setCJKfamilyfont { zhkai  } { KaiTi    }
    \setCJKfamilyfont { zhfs   } { FangSong }
  }
%</zh|ja>
%</windows>
%<*mac>
%<*und>
\tl_if_eq:NNTF \g_@@_lang_tl \c_@@_lang_ja_tl
  { \ctex_file_input:n { sjtu-cjk-font-mac-ja.def } }
  { \ctex_file_input:n { sjtu-cjk-font-mac-zh.def } }
%</und>
%<*zh|ja>
%<zh>\@@_fontset_case:nnn
%<ja>\@@_fontset_case:nn
  { \@@_fontset_error:nn { cjk } { mac } }
%<*zh>
  {
    \ctex_load_zhmap:nnnn { zhsong } { zhhei } { zhfs } { mac }
    \ctex_punct_set:n { mac }
    \ctex_punct_map_family:nn   { \CJKrmdefault } { zhsong }
    \ctex_punct_map_family:nn   { \CJKsfdefault } { zhpf   }
    \ctex_punct_map_bfseries:nn { \CJKrmdefault } { zhpf   }
    \ctex_punct_map_itshape:nn  { \CJKrmdefault } { zhkai  }
  }
%</zh>
  {
%<*zh>
    \setCJKmainfont { Songti~SC  }
      [
        UprightFont    = *~Light ,
        BoldFont       = *~Bold ,
        ItalicFont     = Kaiti~SC~Regular ,
        BoldItalicFont = Kaiti~SC~Bold
      ]
    \setCJKsansfont { Heiti~SC   }
      [
        UprightFont    = *~Medium ,
        AutoFakeBold   = 3
      ]
    \setCJKmonofont { STFangsong }
%</zh>
%<*ja>
    \setCJKmainfont { Hiragino~Mincho~ProN      }
      [
        UprightFont    = *~W3 ,
        BoldFont       = *~W6
      ]
    \setCJKsansfont { Hiragino~Kaku~Gothic~ProN }
      [
        UprightFont    = *~W3 ,
        BoldFont       = *~W6
      ]
    \setCJKmonofont { Hiragino~Kaku~Gothic~ProN }
    \setCJKfamilyfont { jamin  } { Hiragino~Mincho~ProN      }
      [
        UprightFont    = *~W3 ,
        BoldFont       = *~W6
      ]
    \setCJKfamilyfont { jagoth } { Hiragino~Kaku~Gothic~ProN }
      [
        UprightFont    = *~W3 ,
        BoldFont       = *~W6
      ]
%</ja>
    \setCJKfamilyfont { zhsong } { Songti~SC  }
      [
        UprightFont    = *~Light ,
        BoldFont       = *~Bold ,
        ItalicFont     = Kaiti~SC~Regular ,
        BoldItalicFont = Kaiti~SC~Bold
      ]
    \setCJKfamilyfont { zhhei  } { Heiti~SC   }
      [
        UprightFont    = *~Medium ,
        AutoFakeBold   = 3
      ]
    \setCJKfamilyfont { zhfs   } { STFangsong }
    \setCJKfamilyfont { zhkai  } { Kaiti~SC   }
      [
        UprightFont    = *~Regular ,
        BoldFont       = *~Bold
      ]
  }
%</zh|ja>
%</mac>
%<*ubuntu>
%<*und>
\tl_if_eq:NNTF \g_@@_lang_tl \c_@@_lang_ja_tl
  { \ctex_file_input:n { sjtu-cjk-font-ubuntu-ja.def } }
  { \ctex_file_input:n { sjtu-cjk-font-ubuntu-zh.def } }
%</und>
%<*zh|ja>
%<zh>\@@_fontset_case:nnn
%<ja>\@@_fontset_case:nn
  { \@@_fontset_error:nn { cjk } { ubuntu } }
%<*zh>
  {
    \ctex_load_zhmap:nnnn { zhsong } { zhhei } { zhsong } { ubuntu }
    \ctex_punct_set:n { ubuntu }
    \ctex_punct_map_family:nn   { \CJKrmdefault } { zhsong }
    \ctex_punct_map_bfseries:nn { \CJKrmdefault } { zhhei  }
    \ctex_punct_map_itshape:nn  { \CJKrmdefault } { zhkai  }
  }
%</zh>
  {
%<*zh>
    \setCJKmainfont { Noto~Serif~CJK~SC     }
      [
        UprightFont = *~Light ,
        BoldFont    = *~Bold ,
        ItalicFont  = AR~PL~KaitiM~GB
      ]
    \setCJKsansfont { Noto~Sans~CJK~SC      }
      [
        UprightFont = *~Medium ,
        BoldFont    = *~Bold
      ]
    \setCJKmonofont { Noto~Sans~Mono~CJK~SC }
%</zh>
%<*ja>
    \setCJKmainfont { Noto~Serif~CJK~JP     }
      [
        UprightFont = *~Light ,
        BoldFont    = *~Bold
      ]
    \setCJKsansfont { Noto~Sans~CJK~JP      }
      [
        UprightFont = *~Medium ,
        BoldFont    = *~Bold
      ]
    \setCJKmonofont { Noto~Sans~Mono~CJK~JP }
    \setCJKfamilyfont { jamin  } { Noto~Serif~CJK~JP }
      [
        UprightFont = *~Light ,
        BoldFont    = *~Bold
      ]
    \setCJKfamilyfont { jagoth } { Noto~Sans~CJK~JP  }
      [
        UprightFont = *~Medium ,
        BoldFont    = *~Bold
      ]
%</ja>
    \setCJKfamilyfont { zhsong } { Noto~Serif~CJK~SC     }
      [
        UprightFont = *~Light ,
        BoldFont    = *~Bold ,
        ItalicFont  = AR~PL~KaitiM~GB
      ]
    \setCJKfamilyfont { zhhei  } { Noto~Sans~CJK~SC      }
      [
        UprightFont = *~Medium ,
        BoldFont    = *~Bold
      ]
    \setCJKfamilyfont { zhfs   } { Noto~Sans~Mono~CJK~JP }
    \setCJKfamilyfont { zhkai  } { AR~PL~KaitiM~GB       }
  }
%</zh|ja>
%</ubuntu>
%<*adobe>
%<*und>
\tl_if_eq:NNTF \g_@@_lang_tl \c_@@_lang_ja_tl
  { \ctex_file_input:n { sjtu-cjk-font-adobe-ja.def } }
  { \ctex_file_input:n { sjtu-cjk-font-adobe-zh.def } }
%</und>
%<*zh|ja>
%<zh>\@@_fontset_case:nnn
%<ja>\@@_fontset_case:nn
  { \@@_fontset_error:nn { cjk } { adobe } }
%<*zh>
  {
    \ctex_load_zhmap:nnnn { zhsong } { zhhei } { zhfs } { adobe }
    \ctex_punct_set:n { adobe }
    \ctex_punct_map_family:nn   { \CJKrmdefault } { zhsong }
    \ctex_punct_map_bfseries:nn { \CJKrmdefault } { zhhei  }
    \ctex_punct_map_itshape:nn  { \CJKrmdefault } { zhkai  }
  }
%</zh>
  {
%<*zh>
    \setCJKmainfont { AdobeSongStd-Light       }
      [ AutoFakeBold = 3 , ItalicFont = AdobeKaitiStd-Regular ]
    \setCJKsansfont { AdobeHeitiStd-Regular    } [ AutoFakeBold = 3 ]
    \setCJKmonofont { AdobeFangsongStd-Regular }
%</zh>
%<*ja>
    \setCJKmainfont { Kozuka~Mincho~Pr6N }
      [
        UprightFont = *~L ,
        BoldFont    = *~B
      ]
    \setCJKsansfont { Kozuka~Gothic~Pr6N }
      [
        UprightFont = *~M ,
        BoldFont    = *~B
      ]
    \setCJKmonofont { Kozuka~Gothic~Pr6N }
    \setCJKfamilyfont { jamin  } { Kozuka~Mincho~Pr6N }
      [
        UprightFont = *~L ,
        BoldFont    = *~B
      ]
    \setCJKfamilyfont { jagoth } { Kozuka~Gothic~Pr6N }
      [
        UprightFont = *~M ,
        BoldFont    = *~B
      ]
%</ja>
    \setCJKfamilyfont { zhsong } { AdobeSongStd-Light       }
      [ AutoFakeBold = 3 , ItalicFont = AdobeKaitiStd-Regular ]
    \setCJKfamilyfont { zhhei  } { AdobeHeitiStd-Regular    } [ AutoFakeBold = 3 ]
    \setCJKfamilyfont { zhfs   } { AdobeFangsongStd-Regular }
    \setCJKfamilyfont { zhkai  } { AdobeKaitiStd-Regular    }
  }
%</zh|ja>
%</adobe>
%<*fandol>
%<*und>
\tl_if_eq:NNTF \g_@@_lang_tl \c_@@_lang_ja_tl
  { \ctex_file_input:n { sjtu-cjk-font-fandol-ja.def } }
  { \ctex_file_input:n { sjtu-cjk-font-fandol-zh.def } }
%</und>
%<*zh|ja>
%<zh>\@@_fontset_case:nnn
%<ja>\@@_fontset_case:nn
  { \@@_fontset_error:nn { cjk } { fandol } }
%<*zh>
  {
    \ctex_load_zhmap:nnnn { zhsong } { zhhei } { zhfs } { fandol }
    \ctex_punct_set:n { fandol }
    \ctex_punct_map_family:nn   { \CJKrmdefault } { zhsong }
    \ctex_punct_map_bfseries:nn { \CJKrmdefault } { zhhei  }
    \ctex_punct_map_itshape:nn  { \CJKrmdefault } { zhkai  }
  }
%</zh>
  {
%<*zh>
    \setCJKmainfont { FandolSong }
      [
        Extension   = .otf ,
        UprightFont = *-Regular ,
        BoldFont    = *-Bold ,
        ItalicFont  = FandolKai-Regular
      ]
    \setCJKsansfont { FandolHei  }
      [
        Extension   = .otf ,
        UprightFont = *-Regular ,
        BoldFont    = *-Bold
      ]
    \setCJKmonofont { FandolFang }
      [
        Extension   = .otf ,
        UprightFont = *-Regular
      ]
%</zh>
%<*ja>
    \setCJKmainfont { HaranoAjiMincho }
      [
        Extension   = .otf ,
        UprightFont = *-Regular ,
        BoldFont    = *-Bold
      ]
    \setCJKsansfont { HaranoAjiGothic }
      [
        Extension   = .otf ,
        UprightFont = *-Medium ,
        BoldFont    = *-Bold
      ]
    \setCJKmonofont { HaranoAjiGothic }
      [
        Extension   = .otf ,
        UprightFont = *-Regular
      ]
    \setCJKfamilyfont { jamin  } { HaranoAjiMincho }
      [
        Extension   = .otf ,
        UprightFont = *-Regular ,
        BoldFont    = *-Bold
      ]
    \setCJKfamilyfont { jagoth } { HaranoAjiGothic }
      [
        Extension   = .otf ,
        UprightFont = *-Medium ,
        BoldFont    = *-Bold
      ]
%</ja>
    \setCJKfamilyfont { zhsong } { FandolSong }
      [
        Extension   = .otf ,
        UprightFont = *-Regular ,
        BoldFont    = *-Bold ,
        ItalicFont  = FandolKai-Regular
      ]
    \setCJKfamilyfont { zhhei  } { FandolHei  }
      [
        Extension   = .otf ,
        UprightFont = *-Regular,
        BoldFont    = *-Bold
      ]
    \setCJKfamilyfont { zhfs   } { FandolFang }
      [
        Extension   = .otf ,
        UprightFont = *-Regular
      ]
    \setCJKfamilyfont { zhkai  } { FandolKai  }
      [
        Extension   = .otf ,
        UprightFont = *-Regular
      ]
  }
%</zh|ja>
%</fandol>
%<*founder>
%<*und>
\tl_if_eq:NNTF \g_@@_lang_tl \c_@@_lang_ja_tl
  { \ctex_file_input:n { sjtu-cjk-font-founder-ja.def } }
  { \ctex_file_input:n { sjtu-cjk-font-founder-zh.def } }
%</und>
%<*zh|ja>
\@@_fontset_case:nn
%<*zh>
  {
    \ctex_load_zhmap:nnnn { zhsong } { zhhei } { zhfs } { founder }
    \ctex_punct_set:n { founder }
    \ctex_punct_map_family:nn   { \CJKrmdefault } { zhsong }
    \ctex_punct_map_bfseries:nn { \CJKrmdefault } { zhhei  }
    \ctex_punct_map_itshape:nn  { \CJKrmdefault } { zhkai  }
  }
%</zh>
%<ja>  { \@@_fontset_error:nn { cjk } { founder } }
  {
%<*zh>
    \setCJKmainfont { FZShuSong-Z01  }
      [ AutoFakeBold = 3 , ItalicFont = FZKai-Z03 ]
    \setCJKsansfont { FZHei-B01      } [ AutoFakeBold = 3 ]
    \setCJKmonofont { FZFangSong-Z02 }
%</zh>
%<*ja>
    \setCJKmainfont { ipam.ttf } [ AutoFakeBold = 3 ]
    \setCJKsansfont { ipag.ttf } [ AutoFakeBold = 3 ]
    \setCJKmonofont { ipag.ttf }
    \setCJKfamilyfont { jamin  } { ipam.ttf } [ AutoFakeBold = 3 ]
    \setCJKfamilyfont { jagoth } { ipag.ttf } [ AutoFakeBold = 3 ]
%</ja>
    \setCJKfamilyfont { zhsong } { FZShuSong-Z01  }
      [ AutoFakeBold = 3 , ItalicFont = FZKai-Z03 ]
    \setCJKfamilyfont { zhhei  } { FZHei-B01      } [ AutoFakeBold = 3 ]
    \setCJKfamilyfont { zhkai  } { FZKai-Z03      }
    \setCJKfamilyfont { zhfs   } { FZFangSong-Z02 }
  }
%</zh|ja>
%</founder>
%<*!und>
\NewDocumentCommand \songti   { } { \CJKfamily { zhsong  } }
\NewDocumentCommand \heiti    { } { \CJKfamily { zhhei   } }
%<!ubuntu>\NewDocumentCommand \fangsong { } { \CJKfamily { zhfs    } }
\NewDocumentCommand \kaishu   { } { \CJKfamily { zhkai   } }
%<ja>\NewDocumentCommand \mincho   { } { \CJKfamily { jamin   } }
%<ja>\NewDocumentCommand \gothic   { } { \CJKfamily { jagoth  } }
%</!und>
%</font&cjk>
%    \end{macrocode}
%
% \begin{macro}{\CJKrmfamily,\CJKsffamily,\CJKttfamily}
% 只改变 CJK 字体族的命令。
%    \begin{macrocode}
%<*class>
\NewDocumentCommand \CJKrmfamily { } { \CJKfamily { \CJKrmdefault } }
\NewDocumentCommand \CJKsffamily { } { \CJKfamily { \CJKsfdefault } }
\NewDocumentCommand \CJKttfamily { } { \CJKfamily { \CJKttdefault } }
%    \end{macrocode}
% \end{macro}
%
% 带圈数字使用 CJK 字体。
%    \begin{macrocode}
\@@_unicode_engine_case:nn
  {
    \xeCJK_declare_char_class:nn { CJK }
      { "24EA, "2460->"2473, "3251->"32BF, "25A1 }
  }
  {
    \ltjdefcharrange { 99 }
      { "24EA, "2460-"2473, "3251-"32BF, "25A1 }
    \ltjsetparameter { jacharrange = { +99 } }
  }
%    \end{macrocode}
%
% \changes{v2.0.2}{2023/04/01}{延迟载入字体配置,修复 \pkg{unicode-math}
% 设置不生效的问题。}
% 载入字体配置。
%    \begin{macrocode}
\@@_load_fontset:
%    \end{macrocode}
%
% \subsection{名称设置}
%
% 定义 |sjtu/name| 键值类。
%
% 设置标准文档类中已定义的名称。
%    \begin{macrocode}
\keys_define:nn { sjtu / name }
  {
    contents        .meta:nn = { ctex } { contentsname   = {#1} } ,
    listfigure      .meta:nn = { ctex } { listfigurename = {#1} } ,
    listtable       .meta:nn = { ctex } { listtablename  = {#1} } ,
    figure          .meta:nn = { ctex } { figurename     = {#1} } ,
    table           .meta:nn = { ctex } { tablename      = {#1} } ,
%<!thesis>    abstract        .meta:nn = { ctex } { abstractname   = {#1} } ,
    index           .meta:nn = { ctex } { indexname      = {#1} } ,
    appendix        .meta:nn = { ctex } { appendixname   = {#1} } ,
    proof           .meta:nn = { ctex } { proofname      = {#1} } ,
    bib             .meta:nn = { ctex } { bibname        = {#1} } ,
    part           .tl_set:N = \partname ,
%<!article>    chapter        .tl_set:N = \chaptername ,
%    \end{macrocode}
%
% \changes{v2.0}{2022/12/28}{简化 \opt{sjtu/name} 中键的名称:\opt{abbr},
% \opt{nom}, \opt{ack}, \opt{achv}。}
% 标准文档类中未定义的名称。
%    \begin{macrocode}
    figure*        .tl_set:N = \SJTU@figurename@bi@second ,
    figure*       .initial:n = { 图 } ,
    table*         .tl_set:N = \SJTU@tablename@bi@second ,
    table*        .initial:n = { 表 } ,
    algorithm      .tl_set:N = \SJTU@algorithmname ,
    algorithm     .initial:n = { Algorithm } ,
    listalgorithm  .tl_set:N = \SJTU@listalgorithmname ,
    listalgorithm .initial:n = { List~of~Algorithms } ,
%<*thesis>
    abbr           .tl_set:N = \SJTU@abbrname ,
    abbr          .initial:n = { Abbreviations } ,
    nom            .tl_set:N = \SJTU@nomname ,
    nom           .initial:n = { Nomenclature } ,
    ack            .tl_set:N = \SJTU@ackname ,
    ack           .initial:n = { Acknowledgements } ,
    resume         .tl_set:N = \SJTU@resumename ,
    resume        .initial:n = { Resume } ,
    digest         .tl_set:N = \SJTU@digestname ,
    digest        .initial:n = { Digest } ,
    achv           .tl_set:N = \SJTU@achvname ,
    achv          .initial:n = { List~of~Research~Achievements },
%</thesis>
  }
%</class>
%    \end{macrocode}
%
%    \begin{macrocode}
%<*scheme>
%<*zh>
\keys_set_known:nn { sjtu / name }
  {
    contents      = { 目 \protect \quad 录   } ,
    listfigure    = { 插 \protect \quad 图   } ,
    listtable     = { 表 \protect \quad 格   } ,
    figure        = { 图                     } ,
    table         = { 表                     } ,
    abstract      = { 摘 \protect \quad 要   } ,
    index         = { 索 \protect \quad 引   } ,
    appendix      = { 附录                   } ,
    proof         = { 证明                   } ,
    bib           = { 参考文献               } ,
    figure*       = { Figure                 } ,
    table*        = { Table                  } ,
    algorithm     = { 算法                   } ,
    listalgorithm = { 算 \protect \quad 法   } ,
    abbr          = { 缩略语对照表           } ,
    nom           = { 符号对照表             } ,
    ack           = { 致 \protect \quad 谢   } ,
    resume        = { 个人简历               } ,
    digest        = { 大摘要                 } ,
    achv          = { 学术论文和科研成果目录 }
  }
%</zh>
%    \end{macrocode}
%
%    \begin{macrocode}
%<*de>
\keys_set_known:nn { sjtu / name }
  {
    contents      = { Inhaltsverzeichnis     } ,
    listfigure    = { Abbildungsverzeichnis  } ,
    listtable     = { Tabellenverzeichnis    } ,
    figure        = { Abbildung              } ,
    table         = { Tabelle                } ,
    abstract      = { Zusammenfassung        } ,
    index         = { Index                  } ,
    appendix      = { Anhang                 } ,
    proof         = { Beweis                 } ,
    bib           = { Literaturverzeichnis   } ,
    part          = { Teil                   } ,
    chapter       = { Kapitel                } ,
    figure*       = { Figure                 } ,
    table*        = { Table                  } ,
    algorithm     = { Algorithmus            } ,
    listalgorithm = { Algorithmenverzeichnis } ,
    abbr          = { Abkürzungsverzeichnis  } ,
    nom           = { Symbolverzeichnis      } ,
    ack           = { Danksagungen           } ,
    resume        = { Lebenslauf             } ,
    digest        = { Kurzfassung            } ,
    achv          = { Forschungsleistungen   }
  }
%</de>
%    \end{macrocode}
%
%    \begin{macrocode}
%<*ja>
\keys_set_known:nn { sjtu / name }
  {
    contents      = { 目 \protect \quad 次 } ,
    listfigure    = { 図目次               } ,
    listtable     = { 表目次               } ,
    figure        = { 図                   } ,
    table         = { 表                   } ,
    abstract      = { 概 \protect \quad 要 } ,
    index         = { 索 \protect \quad 引 } ,
    appendix      = { 付録                 } ,
    proof         = { 证明                 } ,
    bib           = { 参考文献             } ,
    figure*       = { Figure               } ,
    table*        = { Table                } ,
    algorithm     = { アルゴリズム         } ,
    listalgorithm = { アルゴリズム目次     } ,
    abbr          = { 略語表               } ,
    nom           = { 記号表               } ,
    ack           = { 謝 \protect \quad 辞 } ,
    resume        = { 履歴書               } ,
    digest        = { 要 \protect \quad 約 } ,
    achv          = { 研究業績書           }
  }
%</ja>
%</scheme>
%    \end{macrocode}
%
% 载入名称配置。
%    \begin{macrocode}
%<*class>
%<*thesis>
\clist_map_inline:Nn \g_@@_lang_clist
  { \file_input:n { sjtu-name-thesis- #1 .def } }
\clist_map_inline:nn
  { title_page, declaration, abstract }
  { \@@_define_name:nv {#1} { c_@@_name_ #1 _ \g_@@_lang_tl _tl } }
%</thesis>
%<!thesis>\file_input:n { sjtu-name-generic- \g_@@_lang_tl .def }
%    \end{macrocode}
%
% \subsection{页面设置}
%
% 利用 \pkg{geometry} 宏包设置页面边距以及页眉高度。
%    \begin{macrocode}
\geometry
  {
    top           = 3.5 cm,
    bottom        = 4.0 cm,
    left          = 2.5 cm,
    right         = 2.5 cm,
%<thesis>    bindingoffset = 0.5 cm,
    headheight    = 1.5 cm,
    headsep       = 0.5 cm,
    footskip      = 1.0 cm
  }
%    \end{macrocode}
%
% \changes{v2.0.3}{2023/09/24}{学位论文页面纵向顶部对齐。}
% 学位论文页面纵向顶部对齐。
%    \begin{macrocode}
%<*thesis>
\AtEndOfClass { \raggedbottom }
%</thesis>
%    \end{macrocode}
%
% \subsection{页眉页脚}
%
% \pkg{ctex} 宏包使用 \opt{heading} 选项后,会把页面格式设置为 |headings|。
% 因此必须在 \pkg{ctex} 调用之后重新设置 \tn{pagestyle} 为 |fancy|。
%    \begin{macrocode}
\pagestyle { fancy }
%    \end{macrocode}
%
% 清除所有页眉页脚。
%    \begin{macrocode}
\fancyhf { }
%    \end{macrocode}
%
% 设置页眉页脚字体。
%    \begin{macrocode}
\keys_define:nn { sjtu / style }
  {
    header-font  .tl_set:N = \l_@@_style_header_font_tl ,
%<thesis>    header-font .initial:n = \zihao { -5 } \setbaselineskip { 12 bp } ,
%<!thesis>    header-font .initial:n = \zihao { -5 } \sffamily ,
    footer-font  .tl_set:N = \l_@@_style_footer_font_tl ,
%<thesis>    footer-font .initial:n = \zihao { -5 } \setbaselineskip { 12 bp }
%<!thesis>    footer-font .initial:n = \zihao { -5 }
  }
\fancyheadinit { \l_@@_style_header_font_tl }
\fancyfootinit { \l_@@_style_footer_font_tl }
%    \end{macrocode}
%
% 页眉西文是否大写。
%    \begin{macrocode}
\keys_define:nn { sjtu / style }
  {
    header-uppercase       .choice: ,
    header-uppercase / true  .code:n =
      { \cs_set_eq:NN \@@_nouppercase:n \use:n                 } ,
    header-uppercase / false .code:n =
      { \cs_set:Nn    \@@_nouppercase:n { \nouppercase {##1} } } ,
    header-uppercase      .default:n = { true  } ,
    header-uppercase      .initial:n = { false }
  }
\cs_generate_variant:Nn \@@_nouppercase:n { V }
%    \end{macrocode}
%
% 页眉校标颜色。
%    \begin{macrocode}
%<*!thesis>
\keys_define:nn { sjtu / style }
  {
    header-logo-color       .choice: ,
    header-logo-color     .choices:nn =
      { red, blue, black }
      { \tl_set_eq:NN \l_@@_style_header_logo_color_tl \l_keys_choice_tl } ,
    header-logo-color      .initial:n = { red } ,
  }
%</!thesis>
%    \end{macrocode}
%
% 页脚页码格式。
%    \begin{macrocode}
\cs_new:Nn \@@_thepage: { \thepage }
\keys_define:nn { sjtu / style }
  {
    page-number .cs_set:Np = \@@_page:n #1 ,
    page-number .initial:n = { {#1} }
  }
%    \end{macrocode}
%
% 设置页眉内容。
%    \begin{macrocode}
\tl_set:Nn \l_@@_header_tl
  {
%<*thesis>
    {
      \@@_set_cjk_default_zh: \normalfont
      \l_@@_info_subject_zh_tl
    }
%</thesis>
%<*!thesis>
    \includegraphics [ height = 1.2 cm ]
      { sjtu-vi-logo- \l_@@_style_header_logo_color_tl .pdf }
%</!thesis>
  }
%<thesis>\tl_set:Nn \l_@@_leftmark_tl  { \leftmark }
%<!thesis>\tl_set:Nn \l_@@_leftmark_tl  { \l_@@_info_subject_tl }
\tl_set:Nn \l_@@_rightmark_tl { \leftmark }
%    \end{macrocode}
%
% 设置页眉页脚。
%    \begin{macrocode}
\bool_if:NTF \g_@@_twoside_bool
  {
    \fancyhead [ LO, RE ] { \l_@@_header_tl }
    \fancyhead [ LE ]     { \@@_nouppercase:V \l_@@_leftmark_tl  }
    \fancyhead [ RO ]     { \@@_nouppercase:V \l_@@_rightmark_tl }
  }
  {
    \fancyhead [ L ] { \l_@@_header_tl }
    \fancyhead [ R ] { \@@_nouppercase:V \l_@@_rightmark_tl }
  }
\fancyfoot [ C ] { \@@_page:n { \@@_thepage: } }
%    \end{macrocode}
%
% \cls{sjtuthesis} 的页眉线。
%    \begin{macrocode}
%<*thesis>
\cs_set:Npn \headrule
  {
    \hrule height 2.25 pt width \headwidth
    \skip_vertical:n {  0.75 pt }
    \hrule height 0.75 pt width \headwidth
    \skip_vertical:n { -3.75 pt }
  }
%</thesis>
%    \end{macrocode}
%
% 重定义 |plain| 样式。
%    \begin{macrocode}
%<*!thesis>
\bool_if:NTF \g_@@_twoside_bool
  {
    \fancypagestyle { plain }
      {
        \fancyhead [ LE, RO ] { }
        \tl_set:Nn \headrulewidth { 0 pt }
      }
  }
  {
    \fancypagestyle { plain }
      {
        \fancyhead [ R ] { }
        \tl_set:Nn \headrulewidth { 0 pt }
      }
  }
%</!thesis>
%    \end{macrocode}
%
% |SJTU@null| 样式,不对当前页面样式做任何修改。
%    \begin{macrocode}
\cs_new_eq:NN \ps@SJTU@null \prg_do_nothing:
%    \end{macrocode}
%
% 空白页清空页眉页脚。
%    \begin{macrocode}
\RenewDocumentCommand \cleardoublepage { }
  {
    \clearpage
    \bool_if:NT \g_@@_twoside_bool
      {
        \int_if_odd:nF \c@page
          { \hbox:n { } \thispagestyle { empty } \newpage }
      }
  }
%    \end{macrocode}
%
% \subsection{页码设置}
%
% 文档初始页码编码设置。
%    \begin{macrocode}
%<thesis>\pagenumbering { Alph }
%    \end{macrocode}
%
% \begin{macro}{\frontmatter}
% 前置部分使用大写罗马数字编码。
%    \begin{macrocode}
%<*thesis>
\RenewDocumentCommand \frontmatter { }
  {
    \cleardoublepage
    \@mainmatterfalse
    \pagenumbering { Roman }
  }
%</thesis>
%    \end{macrocode}
% \end{macro}
%
% \subsection{章节标题结构}
%
% 设置章节标题样式。
%    \begin{macrocode}
%<*!article>
\ctex_set:nn { chapter }
  {
    pagestyle   = SJTU@null ,
    fixskip     = true ,
%<*thesis>
    beforeskip  = 27 bp ,
    afterskip   = 27 bp ,
    format      = \zihao { 3 } \setbaselineskip{ 20 bp } \bfseries
                  \CJKsffamily \centering ,
%</thesis>
%<*report>
    beforeskip  = 30 pt ,
    afterskip   = 24 pt ,
    format      = \Large \bfseries \CJKsffamily \centering ,
%</report>
    nameformat  = ,
    titleformat = ,
    lofskip     = \c_zero_skip ,
    lotskip     = \c_zero_skip ,
    aftername   = \quad ,
    afterindent = true
  }
%</!article>
%</class>
%<*scheme&(zh|ja)>
\keys_set_known:nn { ctex / chapter }
  { name = { 第 \space , \space 章 } }
%</scheme&(zh|ja)>
%<*class>
\ctex_set:nn { section }
  {
%<*thesis>
    beforeskip  = 24 bp ,
    afterskip   =  6 bp ,
    format      = \zihao { 4 } \setbaselineskip { 18 bp } \bfseries
                  \CJKsffamily ,
%</thesis>
%<!thesis>    format      = \large \bfseries \CJKsffamily ,
    afterindent = true
  }
\ctex_set:nn { subsection }
  {
%<*thesis>
    beforeskip  = 12 bp ,
    afterskip   =  6 bp ,
    format      = \zihao { -4 } \setbaselineskip { 16 bp } \bfseries
                  \CJKsffamily ,
%</thesis>
%<!thesis>    format      = \normalsize \bfseries \CJKsffamily ,
    afterindent = true
  }
\ctex_set:nn { subsubsection }
  {
%<*thesis>
    beforeskip  =  6 bp ,
    afterskip   =  6 bp ,
    format      = \zihao { -4 } \setbaselineskip { 16 bp } \normalfont ,
%</thesis>
%<!thesis>    format      = \normalsize \normalfont ,
    afterindent = true
  }
\ctex_set:nn { paragraph }
  { afterindent = true }
\ctex_set:nn { subparagraph }
  { afterindent = true }
\ctex_set:n { secnumdepth = 3 }
%    \end{macrocode}
%
% \begin{macro}[int]{\SJTU@head,\@@_pdf_bookmark:nn,\@@_phantom_section:}
% 定义一个灵活的章节标题命令专门处理不同的需求。
%    \begin{macrocode}
\NewDocumentCommand \SJTU@head { s O{#3} m O{#2} }
  {
%<!article>    \CTEX@chapter@break
    \IfBooleanTF {#1}
      { \tl_if_empty:nF {#4} { \@@_pdf_bookmark:nn { 0 } {#4} } }
      {
        \@@_phantom_section:
%<!article>        \addcontentsline { toc } { chapter } {#4}
%<article>        \addcontentsline { toc } { section } {#4}
      }
    \cs_set_eq:NN \@@_orig_ctex_gettitle:n \CTEX@gettitle
    \cs_set:Npn \CTEX@gettitle ##1 { \@@_orig_ctex_gettitle:n {#2} }
%<!article>    \chapter* {#3}
%<article>    \section* {#3}
    \cs_set_eq:NN \CTEX@gettitle \@@_orig_ctex_gettitle:n
    \@mkboth { \MakeUppercase {#2} } { \MakeUppercase {#2} }
  }
\cs_new:Npn \@@_head_aux_s:nn #1#2
  { \SJTU@head* [#1] {#2} }
\cs_generate_variant:Nn \@@_head_aux_s:nn { nx }
\cs_new_eq:NN \@@_pdf_bookmark:nn \use_none:nn
\cs_new_eq:NN \@@_phantom_section: \prg_do_nothing:
%</class>
%    \end{macrocode}
% \end{macro}
%
% \subsection{段落}
%
% 设置全文首行缩进。
%    \begin{macrocode}
%<*scheme>
\ctex_if_autoindent_touched:F
%<zh>  { \ctex_set:n { autoindent = true } }
%<en|de>  { \ctex_set:n { autoindent = 1.5 em } }
%<ja>  { \ctex_set:n { autoindent = 1 } }
%    \end{macrocode}
%
% \begin{macro}{\verse,\quotation}
% 修改诗歌和引用环境的缩进。
%    \begin{macrocode}
%<zh|ja>\ctex_patch_cmd:Nnn \verse { -1.5em } { -2 \ccwd }
%<zh|ja>\ctex_patch_cmd:Nnn \verse {  1.5em } {  2 \ccwd }
\ctex_patch_cmd:Nnn \quotation { 1.5em } { \parindent }
%</scheme>
%    \end{macrocode}
% \end{macro}
%
% 使用 \pkg{enumitem} 调整默认列表环境的间距。
%    \begin{macrocode}
%<*class>
\setlist { nosep }
%    \end{macrocode}
%
% \subsection{数学公式}
%
% \changes{unreleased}{2023/11/30}{新增 \opt{style/equation-font} 选项。}
% 设置行间数学公式的字体。
%    \begin{macrocode}
\keys_define:nn { sjtu / style }
  {
    equation-font  .tl_set:N = \SJTU@style@equation@font ,
%<thesis>    equation-font .initial:x = \exp_not:N \linespread { }
%<thesis>                               \exp_not:N \zihao { \g_@@_zihao_tl }
%<!thesis>    equation-font .initial:V = \c_empty_tl
  }
\clist_map_inline:nn
  {
    array, matrix, pmatrix, bmatrix, Bmatrix, vmatrix, Vmatrix,
    matrix*, pmatrix*, bmatrix*, Bmatrix*, vmatrix*, Vmatrix*,
    cases, cases*, dcases, dcases*, rcases, rcases*, drcases, drcases*,
    aligned, alignedat, gathered, multlined, lgathered, rgathered
  }
  { \AtBeginEnvironment {#1} { \SJTU@style@equation@font } }
\clist_map_inline:nn
  { \start@gather, \start@align, \start@multline }
  {
    \ctex_patch_cmd:Nnn #1
      { \collect@body }
      {
        \SJTU@style@equation@font
        \collect@body
      }
  }
\ctex_patch_cmd:Nnn \gather@split
  { \spread@equation }
  {
    \SJTU@style@equation@font
    \spread@equation
  }
%    \end{macrocode}
%
% \subsection{浮动体}
%
% 下面这组命令使浮动对象的缺省值稍微宽松一点,从而防止幅度对象占据过多的
% 文本页面,也可以防止在很大空白的浮动页上放置很小的图形。
%    \begin{macrocode}
\tl_set:Nn \textfraction      { 0.15 }
\tl_set:Nn \topfraction       { 0.85 }
\tl_set:Nn \bottomfraction    { 0.65 }
\tl_set:Nn \floatpagefraction { 0.60 }
%    \end{macrocode}
%
% 设置浮动体内的字体。
%    \begin{macrocode}
\keys_define:nn { sjtu / style }
  {
    float-font  .tl_set:N = \SJTU@style@float@font ,
%<thesis>    float-font .initial:n = \zihao { 5 } \setbaselineskip { 14 bp }
%<!thesis>    float-font .initial:n = \zihao { 5 }
  }
\ctex_patch_cmd:Nnn \@floatboxreset
  { \normalsize } { \SJTU@style@float@font }
%    \end{macrocode}
%
% \changes{unreleased}{2023/11/29}{更新题注格式。}
% 题注格式。
%    \begin{macrocode}
\keys_define:nn { sjtu / style }
  {
%    \end{macrocode}
%
% \pkg{bicaption} 中双语标题之间的间距受 \pkg{caption} 字体定义 \opt{normalsize}
% 的影响,这里我们直接重定义 \opt{normalsize}。
%    \begin{macrocode}
    caption-font       .code:n =
      { \DeclareCaptionFont { normalsize    } {#1} } ,
%    \end{macrocode}
%
%    \begin{macrocode}
%<thesis>    caption-font    .initial:n = \zihao { 5 } \setbaselineskip { 14 bp }
%<thesis>                                 \bfseries ,
%<!thesis>    caption-font    .initial:n = \zihao { 5 } \bfseries ,
    subcaption-font    .code:n =
      { \DeclareCaptionFont { SJTU@sub@font } {#1} } ,
%<thesis>    subcaption-font .initial:n = \zihao { 5 } \setbaselineskip { 14 bp }
%<thesis>                                 \normalfont
%<!thesis>    subcaption-font .initial:n = \zihao { 5 } \normalfont
  }
\captionsetup
  {
    bi-slc   = off ,
    labelsep = quad ,
%<thesis>    skip     = 6 bp
  }
\captionsetup [ sub ]
  {
    bi-slc   = on ,
    font     = SJTU@sub@font ,
    format   = hang
  }
%    \end{macrocode}
%
% 双语题注。
%    \begin{macrocode}
\DeclareCaptionOption { bi-second-names } [ ]
  {
    \tl_set:Nn \figurename { \SJTU@figurename@bi@second }
    \tl_set:Nn \tablename  { \SJTU@tablename@bi@second  }
  }
\captionsetup [ bi-second ] { bi-second-names }
%    \end{macrocode}
%
% \changes{unreleased}{2023/12/02}{新增 \opt{style/num-sep}、
% \opt{style/theorem-num-sep} 选项。}
% 图、表、公式以及定理编号中的分隔符。
%    \begin{macrocode}
\keys_define:nn { sjtu / style }
  {
    num-sep            .code:n =
      {
        \tl_set:Nn \SJTU@style@fl@num@sep {#1}
        \tl_set:Nn \SJTU@style@eq@num@sep {#1}
        \tl_set:Nn \@thmcountersep        {#1}
      } ,
    num-sep         .initial:n = { . } ,
    float-num-sep    .tl_set:N = \SJTU@style@fl@num@sep ,
    equation-num-sep .tl_set:N = \SJTU@style@eq@num@sep ,
    theorem-num-sep  .tl_set:N = \@thmcountersep
  }
%    \end{macrocode}
%
% \begin{macro}[int]{\SJTU@counterwithin}
%    \begin{macrocode}
\NewDocumentCommand \SJTU@counterwithin
  { s O{ \SJTU@style@fl@num@sep } O{ \arabic } m m }
  {
    \@ifbothcounters {#4} {#5}
      {
        \@addtoreset {#4} {#5}
        \IfBooleanF {#1}
          {
            \cs_gset:cpo { the #4 }
              { \cs:w the #5 \cs_end: #2 #3 {#4} }
          }
      }
  }
%    \end{macrocode}
% \end{macro}
%
% 定义图、表、公式的编号格式。
%    \begin{macrocode}
%<*!article>
\SJTU@counterwithin { figure } { chapter }
\SJTU@counterwithin { table  } { chapter }
\SJTU@counterwithin [ \SJTU@style@eq@num@sep ] { equation } { chapter }
%</!article>
%    \end{macrocode}
%
% \begin{macro}[int]{\l_@@_counter_without_chapter_clist}
% 大摘要中不需要随章编号的各计数器。
%    \begin{macrocode}
%<*thesis>
\clist_set:Nn \l_@@_counter_without_chapter_clist
  { section, figure, table, equation }
%</thesis>
%    \end{macrocode}
% \end{macro}
%
% \subsection{脚注}
%
% 储存原始的脚注相关命令。
%    \begin{macrocode}
\cs_set_eq:NN \@@_makefnmark_plain: \@makefnmark
\cs_set_eq:NN \@@_thefootnote_plain: \thefootnote
\cs_set_eq:NN \@@_thempfootnote_plain: \thempfootnote
%    \end{macrocode}
%
% 通过 Unicode 码位调用带圈数字。
%    \begin{macrocode}
\cs_new:Npn \@@_footnote_number:N #1
  {
    \int_compare:nNnTF {#1} < { 21 }
      { \@@_unicode_char:n { \int_eval:n { "2460 - 1 + #1 } } }
      {
        \int_compare:nNnTF {#1} < { 36 }
          { \@@_unicode_char:n { \int_eval:n { "3251 - 21 + #1 } } }
          {
            \int_compare:nNnTF {#1} < { 51 }
              { \@@_unicode_char:n { \int_eval:n { "32B1 - 36 + #1 } } }
              { \msg_warning:nn { sjtutex } { too-many-footnotes } }
          }
      }
  }
\msg_new:nnn { sjtutex } { too-many-footnotes }
  { Too~ many~ footnotes. }
%    \end{macrocode}
%
% 重定义内部脚注文字命令,使用带圈数字编号时,脚注不使用上标。
% 见 \url{https://www.zhihu.com/question/53030087}。
%    \begin{macrocode}
\cs_new:Nn \@@_makefnmark_circled: { \hbox:n { \@thefnmark } }
%    \end{macrocode}
%
% 脚注编号字体。
%    \begin{macrocode}
\keys_define:nn { sjtu / style }
  {
    fnmark-font           .choice: ,
    fnmark-font / haranoaji .code:n =
      {
        \@@_engine_case:nn
          { \tl_set_eq:NN \l_@@_style_fnmark_font_tl \c_empty_tl }
          {
            \tl_set:Nn \l_@@_style_fnmark_font_tl
              {
                \CJKfontspec { HaranoAjiMincho }
                  [
                    Extension   = .otf ,
                    UprightFont = *-Regular ,
                    BoldFont    = *-Bold
                  ]
              }
          }
      } ,
    fnmark-font / unknown .tl_set:N = \l_@@_style_fnmark_font_tl ,
    fnmark-font          .initial:V = \c_empty_tl
  }
%    \end{macrocode}
%
% 使用带圈数字编号脚注。
%    \begin{macrocode}
\cs_new:Nn \@@_thefootnote_circled:
  { { \l_@@_style_fnmark_font_tl \@@_footnote_number:N \c@footnote } }
\cs_new:Nn \@@_thempfootnote_circled:
  { { \l_@@_style_fnmark_font_tl \@@_footnote_number:N \c@mpfootnote } }
%    \end{macrocode}
%
% 脚注编号样式。
%    \begin{macrocode}
\keys_define:nn { sjtu / style }
  {
    fnmark-style           .choice: ,
    fnmark-style / plain   .code:n =
      {
        \cs_set_eq:NN \@@_makefnmark: \@@_makefnmark_plain:
        \cs_set_eq:NN \thefootnote \@@_thefootnote_plain:
        \cs_set_eq:NN \thempfootnote \@@_thempfootnote_plain:
      } ,
    fnmark-style / circled .code:n =
      {
        \cs_set_eq:NN \@@_makefnmark: \@@_makefnmark_circled:
        \cs_set_eq:NN \thefootnote \@@_thefootnote_circled:
        \cs_set_eq:NN \thempfootnote \@@_thempfootnote_circled:
      }
  }
%</class>
%    \end{macrocode}
%
%    \begin{macrocode}
%<*scheme>
\keys_set:nn { sjtu / style }
%<zh|ja>  { fnmark-style = circled }
%<en|de>  { fnmark-style = plain   }
%</scheme>
%    \end{macrocode}
%
% 在导言末尾修改 \tn{@makefntext},支持使用 \pkg{footmisc} 修改脚注格式。
%    \begin{macrocode}
%<*class>
\ctex_at_end_preamble:n {
  \cs_set_eq:NN \@@_orig_make_fntext:n \@makefntext
  \cs_set:Npn \@makefntext #1
    {
      \group_begin:
        \cs_set_eq:NN \@makefnmark \@@_makefnmark:
        \@@_orig_make_fntext:n {#1}
      \group_end:
    }
}
%    \end{macrocode}
%
% \subsection{信息录入}
%
% \changes{v2.0}{2022/12/17}{添加 \opt{display-date} 键。}
% \changes{v2.0}{2022/12/23}{使用 \opt{assoc-supervisor} 键表示副导师,
%                            使用 \opt{co-supervisor} 键表示联合导师。}
% \changes{v2.0}{2022/12/23}{添加 \opt{subject} 键。}
% \changes{v2.0}{2023/03/17}{使用语言代码前缀区别不同语种的 \opt{sjtu/info} 键。}
% \begin{macro}[int]{\@@_info_keys_define:n}
% 定义 \cls{sjtuthesis} 不同语种 |sjtu/info| 键值类的辅助函数。
%    \begin{macrocode}
%<*thesis>
\cs_new:Npn \@@_info_keys_define:n #1
  {
    \clist_map_inline:nn
      {
        title, display_title, subject, author, date,
        supervisor, assoc_supervisor, department,
        co_supervisor, major, degree
      }
      { \tl_new:c { l_@@_info_ ##1 _ #1 _tl } }
    \clist_map_inline:nn
      { keywords, fund }
      { \clist_new:c { l_@@_info_ ##1 _ #1 _clist } }
    \keys_define:nn { sjtu }
      { info / #1 .meta:nn = { sjtu / info / #1 } {##1} }
    \keys_define:nn { sjtu / info }
      {        #1 .meta:nn = { sjtu / info / #1 } {##1} }
    \keys_define:nn { sjtu / info / #1 }
      {
        title               .code:n =
          {
            \tl_set:cn { l_@@_info_title_ #1 _tl } {##1}
            \tl_if_empty:cT { l_@@_info_display_title_ #1 _tl }
              { \tl_set:cn { l_@@_info_display_title_ #1 _tl } {##1} }
          } ,
        display-title     .tl_set:c = l_@@_info_display_title_ #1 _tl ,
        subject           .tl_set:c = l_@@_info_subject_ #1 _tl ,
        keywords       .clist_set:c = l_@@_info_keywords_ #1 _clist ,
        author            .tl_set:c = l_@@_info_author_ #1 _tl ,
        id                 .meta:nn = { sjtu / info } { id = {##1} } ,
        supervisor        .tl_set:c = l_@@_info_supervisor_ #1 _tl ,
        assoc-supervisor  .tl_set:c = l_@@_info_assoc_supervisor_ #1 _tl ,
        co-supervisor     .tl_set:c = l_@@_info_co_supervisor_ #1 _tl ,
        degree            .tl_set:c = l_@@_info_degree_ #1 _tl ,
        department        .tl_set:c = l_@@_info_department_ #1 _tl ,
        major             .tl_set:c = l_@@_info_major_ #1 _tl ,
        fund           .clist_set:c = l_@@_info_fund_ #1 _clist ,
        date               .meta:nn = { sjtu / info } { date = {##1} } ,
        display-date      .tl_set:c = l_@@_info_date_ #1 _tl ,
      }
  }
%    \end{macrocode}
% \end{macro}
%
% 单独处理学号。
%    \begin{macrocode}
\tl_new:N \l_@@_info_id_zh_tl
\keys_define:nn { sjtu / info }
  { id   .tl_set:N = \l_@@_info_id_zh_tl }
%    \end{macrocode}
%
% 标题中需要显示的导师列表。
%    \begin{macrocode}
\clist_set:Nn \l_@@_info_supervisors_clist { supervisor }
%    \end{macrocode}
%
% 盲审模式下隐藏作者、导师姓名等信息。
%    \begin{macrocode}
\ctex_at_end_preamble:n
  {
    \bool_if:NT \g_@@_review_bool
      {
        \clist_map_inline:Nn \g_@@_lang_clist
          {
            \clist_map_inline:nn
              { author, supervisor, assoc_supervisor, co_supervisor }
              { \tl_clear:c { l_@@_info_ ##1 _ #1 _tl } }
            \clist_clear:c { l_@@_info_fund_ #1 _clist }
          }
        \tl_clear:N \l_@@_info_id_zh_tl
      }
  }
%</thesis>
%</class>
%    \end{macrocode}
%
%    \begin{macrocode}
%<lang&thesis&zh>\@@_info_keys_define:n { zh }
%<lang&thesis&en>\@@_info_keys_define:n { en }
%<lang&thesis&de>\@@_info_keys_define:n { de }
%<lang&thesis&ja>\@@_info_keys_define:n { ja }
%    \end{macrocode}
%
%    \begin{macrocode}
%<*class&!thesis>
\tl_new:N \l_@@_info_subject_tl
\clist_new:N \l_@@_info_keywords_clist
\keys_define:nn { sjtu / info }
  {
    title             .tl_set:N = \@title ,
    author            .tl_set:N = \@author ,
    display-date      .tl_set:N = \@date ,
    subject           .tl_set:N = \l_@@_info_subject_tl ,
    keywords       .clist_set:N = \l_@@_info_keywords_clist ,
  }
%</class&!thesis>
%    \end{macrocode}
%
% \subsection{多语言支持}
%
% \changes{v2.0.3}{2023/04/08}{更新学位论文初始英文主题。}
% 初始化主题。
%    \begin{macrocode}
%<*lang>
%<*thesis>
%<zh>\keys_define:nn { sjtu / info / zh }
%<en>\keys_define:nn { sjtu / info / en }
%<de>\keys_define:nn { sjtu / info / de }
%<ja>\keys_define:nn { sjtu / info / ja }
  {
    subject          .initial:x =
      {
%<*zh>
        \exp_not:V \c_@@_name_univ_zh_tl
        \exp_not:V \c_@@_name_degree_level_zh_tl
        \exp_not:V \c_@@_name_thesis_zh_tl
%</zh>
%<*en>
        A~ Dissertation~ Submitted~ to \exp_not:N \\
        { \exp_not:V \c_@@_name_univ_en_tl }~ for~
        the~ Degree~ of~ { \exp_not:V \c_@@_name_degree_level_en_tl }
%</en>
%<*de>
        Eine~ Dissertation~ Eingereicht~ an \exp_not:N \\
        der~ { \exp_not:V \c_@@_name_univ_de_tl }~ für~
        { \exp_not:V \c_@@_name_degree_level_de_tl } titel
%</de>
%<*ja>
        \exp_not:V \c_@@_name_univ_ja_tl
        \exp_not:V \c_@@_name_degree_level_ja_tl
        \exp_not:V \c_@@_name_thesis_ja_tl
%</ja>
      }
  }
%</thesis>
%    \end{macrocode}
%
% 将形如 |yyyy-mm-dd| 或 |yyyy-mm| 的 ISO 日期格式字符串转化为日期表示。
%    \begin{macrocode}
%<*zh>
\cs_new:Npn \@@_date_aux_zh:nnn #1#2#3
  {
    \int_to_arabic:n {#1} ~ { \exp_not:V \c_@@_name_year_zh_tl  } ~
    \int_to_arabic:n {#2} ~ { \exp_not:V \c_@@_name_month_zh_tl } ~
    \int_to_arabic:n {#3} ~ { \exp_not:V \c_@@_name_day_zh_tl   }
  }
\cs_new:Npn \@@_date_aux_zh:w #1-#2-#3 \q_stop
  { \@@_date_aux_zh:nnn {#1} {#2} {#3} }
\cs_new:Npn \@@_date_aux_short_zh:nn #1#2
  {
    \int_to_arabic:n {#1} ~ { \exp_not:V \c_@@_name_year_zh_tl  } ~
    \int_to_arabic:n {#2} ~ { \exp_not:V \c_@@_name_month_zh_tl }
  }
\cs_new:Npn \@@_date_aux_short_zh:w #1-#2 \q_stop
  { \@@_date_aux_short_zh:nn {#1} {#2} }
%</zh>
%    \end{macrocode}
%
%    \begin{macrocode}
%<*en>
\cs_new:Npn \@@_ordinal_en:n #1
  {
    \int_to_arabic:n {#1}
    \exp_not:N \textsuperscript
      {
        \int_case:nnF { \int_mod:nn {#1} { 100 } }
          {
            { 11 } { th }
            { 12 } { th }
            { 13 } { th }
          }
          {
            \int_case:nnF { \int_mod:nn {#1} { 10 } }
              {
                { 1 } { st }
                { 2 } { nd }
                { 3 } { rd }
              }
              { th }
          }
      }
  }
\cs_new:Npn \@@_date_aux_en:nnn #1#2#3
  {
    \clist_item:Nn \c_@@_name_month_en_clist {#2} ~
    \@@_ordinal_en:n {#3} ,~
    \int_to_arabic:n {#1}
  }
\cs_new:Npn \@@_date_aux_en:w #1-#2-#3 \q_stop
  { \@@_date_aux_en:nnn {#1} {#2} {#3} }
\cs_new:Npn \@@_date_aux_short_en:nn #1#2
  {
    \clist_item:Nn \c_@@_name_month_en_clist {#2} ,~
    \int_to_arabic:n {#1}
  }
\cs_new:Npn \@@_date_aux_short_en:w #1-#2 \q_stop
  { \@@_date_aux_short_en:nn {#1} {#2} }
%</en>
%    \end{macrocode}
%
%    \begin{macrocode}
%<*de>
\cs_new:Npn \@@_date_aux_de:nnn #1#2#3
  {
    \clist_item:Nn \c_@@_name_month_de_clist {#2} ~
    {#3} ,~ \int_to_arabic:n {#1}
  }
\cs_new:Npn \@@_date_aux_de:w #1-#2-#3 \q_stop
  { \@@_date_aux_de:nnn {#1} {#2} {#3} }
\cs_new:Npn \@@_date_aux_short_de:nn #1#2
  {
    \clist_item:Nn \c_@@_name_month_de_clist {#2} ,~
    \int_to_arabic:n {#1}
  }
\cs_new:Npn \@@_date_aux_short_de:w #1-#2 \q_stop
  { \@@_date_aux_short_de:nn {#1} {#2} }
%</de>
%    \end{macrocode}
%
%    \begin{macrocode}
%<*ja>
\cs_new:Npn \@@_date_aux_ja:nnn #1#2#3
  {
    \int_to_arabic:n {#1} ~ { \exp_not:V \c_@@_name_year_ja_tl  } ~
    \int_to_arabic:n {#2} ~ { \exp_not:V \c_@@_name_month_ja_tl } ~
    \int_to_arabic:n {#3} ~ { \exp_not:V \c_@@_name_day_ja_tl   }
  }
\cs_new:Npn \@@_date_aux_ja:w #1-#2-#3 \q_stop
  { \@@_date_aux_ja:nnn {#1} {#2} {#3} }
\cs_new:Npn \@@_date_aux_short_ja:nn #1#2
  {
    \int_to_arabic:n {#1} ~ { \exp_not:V \c_@@_name_year_ja_tl  } ~
    \int_to_arabic:n {#2} ~ { \exp_not:V \c_@@_name_month_ja_tl }
  }
\cs_new:Npn \@@_date_aux_short_ja:w #1-#2 \q_stop
  { \@@_date_aux_short_ja:nn {#1} {#2} }
%</ja>
%    \end{macrocode}
%
% 设置语言格式辅助命令
%    \begin{macrocode}
%<*zh>
\cs_new:Nn \@@_set_cjk_default_zh:
  {
    \tl_set:Nn \CJKrmdefault { zhsong }
    \tl_set:Nn \CJKsfdefault { zhhei  }
    \tl_set:Nn \CJKttdefault { zhfs   }
  }
\cs_new_protected:Nn \@@_set_language_zh:
  {
    \tl_set:Nn \languagename { chinese }
    \ctex_set:n { autoindent = true }
    \@@_set_cjk_default_zh:
    \normalfont
  }
%</zh>
%<*en|de>
%<en>\cs_new_protected:Nn \@@_set_language_en:
%<de>\cs_new_protected:Nn \@@_set_language_de:
  {
%<en>    \tl_set:Nn \languagename { english }
%<de>    \tl_set:Nn \languagename { ngerman }
    \ctex_set:n { autoindent = 1.5 em }
    \normalfont
  }
%</en|de>
%<*ja>
\cs_new:Nn \@@_set_cjk_default_ja:
  {
    \tl_set:Nn \CJKrmdefault { jamin  }
    \tl_set:Nn \CJKsfdefault { jagoth }
    \tl_set:Nn \CJKttdefault { jagoth }
  }
\cs_new_protected:Nn \@@_set_language_ja:
  {
    \tl_set:Nn \languagename { japanese }
    \ctex_set:n { autoindent = 1 }
    \@@_set_cjk_default_ja:
    \normalfont
  }
%</ja>
%    \end{macrocode}
%
% 标题页页面样式,页脚添加资助基金信息。
%    \begin{macrocode}
%<*thesis>
%<zh>\cs_new:Npn \ps@SJTU@fund@zh
%<en>\cs_new:Npn \ps@SJTU@fund@en
%<de>\cs_new:Npn \ps@SJTU@fund@de
%<ja>\cs_new:Npn \ps@SJTU@fund@ja
  {
    \ps@empty
    \cs_set:Npn \@oddfoot
      {
        \minipage [ t ] { \textwidth }
          \centering \zihao { - 5 }
%<zh>          \clist_use:Nn \l_@@_info_fund_zh_clist { \par }
%<en>          \clist_use:Nn \l_@@_info_fund_en_clist { \par }
%<de>          \clist_use:Nn \l_@@_info_fund_de_clist { \par }
%<ja>          \clist_use:Nn \l_@@_info_fund_ja_clist { \par }
        \endminipage
      }
    \cs_set_eq:NN \@evenfoot \@oddfoot
  }
%</thesis>
%</lang>
%    \end{macrocode}
%
% 初始化语言名称。
%    \begin{macrocode}
%<*scheme>
%<zh>\tl_set:Nn \languagename { chinese  }
%<en>\tl_set:Nn \languagename { english  }
%<de>\tl_set:Nn \languagename { ngerman  }
%<ja>\tl_set:Nn \languagename { japanese }
%</scheme>
%    \end{macrocode}
%
% 载入语言配置。
%    \begin{macrocode}
%<*class>
%<thesis>\clist_map_inline:Nn \g_@@_lang_clist
%<thesis>  { \file_input:n { sjtu-lang-thesis- #1 .def } }
%<!thesis>\file_input:n { sjtu-lang-generic- \g_@@_lang_tl .def }
\file_input:n { sjtu-scheme- \g_@@_lang_tl .def }
%    \end{macrocode}
%
% 初始化日期。
%    \begin{macrocode}
\keys_define:nn { sjtu / info }
  {
    date    .code:n =
      {
        \regex_match:neTF { \d+-\d+-\d+ } {#1}
          {
%<*thesis>
            \clist_map_inline:Nn \g_@@_lang_clist
              {
                \tl_set:cx { l_@@_info_date_ ##1 _tl }
                  { \exp_last_unbraced:ce { @@_date_aux_ ##1 :w } #1 \q_stop }
              }
%</thesis>
%<*!thesis>
            \tl_set:Nx \@date
              {
                \exp_last_unbraced:ce
                  { @@_date_aux_ \g_@@_lang_tl :w } #1 \q_stop
              }
%</!thesis>
          }
          {
            \regex_match:neT { \d+-\d+ } {#1}
              {
%<*thesis>
                \clist_map_inline:Nn \g_@@_lang_clist
                  {
                    \tl_set:cx { l_@@_info_date_ ##1 _tl }
                      { \exp_last_unbraced:ce { @@_date_aux_short_ ##1 :w } #1 \q_stop }
                  }
%</thesis>
%<*!thesis>
                \tl_set:Nx \@date
                  {
                    \exp_last_unbraced:ce
                      { @@_date_aux_short_ \g_@@_lang_tl :w } #1 \q_stop
                  }
%</!thesis>
              }
          }
      } ,
    date .initial:x =
      {
        \int_to_arabic:n { \c_sys_year_int  } -
        \int_to_arabic:n { \c_sys_month_int } -
        \int_to_arabic:n { \c_sys_day_int   }
      }
  }
%<thesis>\tl_set:Nv \today { l_@@_info_date_ \g_@@_lang_tl _tl }
%<!thesis>\tl_set:NV \today \@date
%    \end{macrocode}
%
% \subsection{标题页}
%
% \subsubsection{定义内部函数}
%
% 汉字分散对齐的环境。
%    \begin{macrocode}
%<*thesis>
\@@_engine_case:nnn
  {
    \NewDocumentEnvironment { SJTU@CJK@FTS } { m b }
      {
        \mode_leave_vertical:
        \bool_set_false:N \l_@@_tmp_bool
        \cs_set_eq:NN \SJTU@CJK@FTS@Symbol \CJKsymbol
        \cs_set:Npn \CJKsymbol ##1
          {
            \bool_if:NTF \l_@@_tmp_bool
              { \hfil \SJTU@CJK@FTS@Symbol { ##1 } }
              {
                \SJTU@CJK@FTS@Symbol { ##1 }
                \bool_set_true:N \l_@@_tmp_bool
              }
          }
        \hbox_to_wd:nn {#1} {#2}
      } { }
  }
  {
    \NewDocumentEnvironment { SJTU@CJK@FTS } { m b }
      {
        \mode_leave_vertical:
        \cs_set:Npn \CJKglue
          { \skip_horizontal:n { \c_zero_dim plus 1 filll } }
        \hbox_to_wd:nn {#1} {#2}
      } { }
  }
  {
    \NewDocumentEnvironment { SJTU@CJK@FTS } { m b }
      {
        \mode_leave_vertical:
        \ltjsetparameter { kanjiskip = { \c_zero_dim plus 1 filll } }
        \hbox_to_wd:nn {#1} {#2}
      } { }
  }
%    \end{macrocode}
%
% 汉字分散对齐的表格列说明符。
%    \begin{macrocode}
\newcolumntype { \SJTU@CT@D } [ 1 ]
  { >{ \begin { SJTU@CJK@FTS } {#1} } c <{ \end { SJTU@CJK@FTS } } }
%    \end{macrocode}
%
% \changes{unreleased}{2023/11/30}{标题页信息栏改用表格实现。}
% 信息输出。
%    \begin{macrocode}
\cs_new:Npn \@@_title_page_info_i:nnn #1#2#3
  {
    \clist_clear:N \l_@@_tmp_clist
    \clist_map_inline:nn {#3}
      {
        \clist_put_right:Nx \l_@@_tmp_clist
          {
            \exp_not:o { \cs:w c_@@_name_ ##1 _ #1 _tl \cs_end: }
            &
            \exp_not:o { \cs:w l_@@_info_ ##1 _ #1 _tl \cs_end: }
          }
      }
    \group_begin:
      \tl_set:Nn \arraystretch { 1 }
      \tabular {#2}
        \clist_use:Nn \l_@@_tmp_clist { \\ }
      \endtabular
    \group_end:
  }
\cs_new:Npn \@@_title_page_info_ii:n #1
  {
    \tl_use:c { l_@@_info_department_ #1 _tl }
    \skip_vertical:N \c_zero_skip
    \tl_use:c { c_@@_name_univ_ #1 _tl }
    \skip_vertical:N \c_zero_skip
    \tl_use:c { c_@@_name_address_ #1 _tl }
    \skip_vertical:N \c_zero_skip
    \tl_use:c { l_@@_info_date_ #1 _tl }
  }
\cs_generate_variant:Nn \@@_title_page_info_i:nnn { nxx }
%    \end{macrocode}
%
% 标题页校徽颜色。
%    \begin{macrocode}
\keys_define:nn { sjtu / style }
  {
    title-logo-color       .choice: ,
    title-logo-color      .choices:nn =
      { red, blue, black }
      { \tl_set_eq:NN \l_@@_style_title_logo_color_tl \l_keys_choice_tl } ,
    title-logo-color       .initial:n = { red }
  }
%</thesis>
%</class>
%    \end{macrocode}
%
% \subsubsection{构建标题页}
%
% \changes{v2.0}{2022/12/13}{启用新版封面。}
% \changes{v2.0.2}{2023/04/01}{标题页日期底部增加空白。}
%    \begin{macrocode}
%<*lang>
%<*thesis>
%<*zh>
\clist_map_inline:nn
  {
    { logo    }
      {
        content     =
          {
            \includegraphics [ width = 3 cm ]
              { sjtu-vi-badge- \l_@@_style_title_logo_color_tl .pdf }
          }
      },
    { subject }
      {
        format      = \zihao { -2 } \setbaselineskip { 30 bp } ,
        content     = \l_@@_info_subject_zh_tl ,
        bottom-skip = \c_zero_dim plus 1 fill
      },
    { title   }
      {
        format      = \zihao { 2 } \setbaselineskip { 36 bp } \bfseries ,
        content     = \l_@@_info_display_title_zh_tl ,
        bottom-skip = 30 bp plus 1 fill
      },
    { info    }
      {
        format      = \zihao { 4 } \setbaselineskip { 30 bp } \heiti ,
        content     =
          {
            \@@_title_page_info_i:nxx { zh }
              {
                \exp_not:N \SJTU@CT@D { 5 em }
                @ { \c_@@_name_info_sep_zh_tl }
                >{ \exp_not:N \normalfont } l
              }
              {
                author,
                id,
                \l_@@_info_supervisors_clist ,
                department,
                major,
                \int_compare:nNnF { \g_@@_thesis_type_int } = { 1 }
                  { degree }
              }
          } ,
        bottom-skip = 30 bp
      },
    { date    }
      {
        format      = \zihao { 4 } \setbaselineskip { 30 bp } \bfseries ,
        content     = \l_@@_info_date_zh_tl ,
      }
  }
  {
    \@@_declare_component:nnn { title / zh } #1
  }
\@@_declare_page:nn { title / zh }
  {
    bookmark      = true ,
    bookmark-text = \c_@@_name_title_page_tl ,
    style         = SJTU@fund@zh ,
    format        = \linespread { } \@@_set_language_zh: ,
    prefix        = title / zh ,
    components    = { logo, subject, title, info, date }
  }
%</zh>
%<*en|de|ja>
\clist_map_inline:nn
  {
    { subject }
      {
        format      = \zihao { 4 } \setbaselineskip { 24 bp } \bfseries ,
%<en>        content     = \l_@@_info_subject_en_tl ,
%<de>        content     = \l_@@_info_subject_de_tl ,
%<ja>        content     = \l_@@_info_subject_ja_tl ,
        bottom-skip = \c_zero_dim plus 1 fill
      },
    { title   }
      {
        format      = \zihao { -2 } \setbaselineskip { 30 bp } \bfseries ,
%<en>        content     = \MakeUppercase \l_@@_info_display_title_en_tl ,
%<de>        content     = \MakeUppercase \l_@@_info_display_title_de_tl ,
%<ja>        content     = \l_@@_info_display_title_ja_tl ,
        bottom-skip = \c_zero_dim plus 1 fill
      },
    { info    }
      {
        format      = \zihao { 3 } \setbaselineskip { 30 bp } \bfseries ,
        content     =
          {
%<en>            \@@_title_page_info_i:nxx { en }
%<de>            \@@_title_page_info_i:nxx { de }
%<ja>            \@@_title_page_info_i:nxx { ja }
%<en>              { r @ { \c_@@_name_info_sep_en_tl } >{ \exp_not:N \normalfont } l }
%<de>              { r @ { \c_@@_name_info_sep_de_tl } >{ \exp_not:N \normalfont } l }
%<ja>              { r @ { \c_@@_name_info_sep_ja_tl } >{ \exp_not:N \normalfont } l }
              { author, \l_@@_info_supervisors_clist }
          } ,
        bottom-skip = 30 bp plus 1 fill
      },
    { date    }
      {
        format      = \zihao { 3 } \setbaselineskip { 30 bp } ,
        content     =
%<en>          { \@@_title_page_info_ii:n { en } } ,
%<de>          { \@@_title_page_info_ii:n { de } } ,
%<ja>          { \@@_title_page_info_ii:n { ja } } ,
      },
  }
  {
%<en>    \@@_declare_component:nnn { title / en } #1
%<de>    \@@_declare_component:nnn { title / de } #1
%<ja>    \@@_declare_component:nnn { title / ja } #1
  }
%<*en>
\@@_declare_page:nn { title / en }
  {
    style       = SJTU@fund@en ,
    format      = \linespread { } \@@_set_language_en: ,
    prefix      = title / en ,
    components  = { subject, title, info, date }
  }
%</en>
%<*de>
\@@_declare_page:nn { title / de }
  {
    style       = SJTU@fund@de ,
    format      = \linespread { } \@@_set_language_de: ,
    prefix      = title / de ,
    components  = { subject, title, info, date }
  }
%</de>
%<*ja>
\@@_declare_page:nn { title / ja }
  {
    style       = SJTU@fund@ja ,
    format      = \linespread { } \@@_set_language_ja: ,
    prefix      = title / ja ,
    components  = { subject, title, info, date }
  }
%</ja>
%</en|de|ja>
%</thesis>
%</lang>
%    \end{macrocode}
%
%    \begin{macrocode}
%<*class>
%<*thesis>
\RenewDocumentCommand \maketitle { }
  {
    \clist_map_inline:nn
      { assoc_supervisor, co_supervisor }
      {
        \tl_if_empty:cF { l_@@_info_ ##1 _zh_tl }
          { \clist_put_right:Nn \l_@@_info_supervisors_clist {##1} }
      }
    \clist_map_inline:Nn \g_@@_lang_clist
      { \UseInstance { sjtu } { title / ##1 } }
  }
%</thesis>
%    \end{macrocode}
%
% \subsection{原创性声明及使用授权书}
%
%    \begin{macrocode}
%<*thesis>
\cs_new_protected:Npn \@@_signature:N #1
  {
    \parbox [ t ] { 12 em }
      { #1 \c_@@_signature_text_zh_tl }
  }
\clist_map_inline:nn
  {
    { orig / title }
      {
        format      = \zihao { 3 } \setbaselineskip { 30 bp }
                      \bfseries \heiti ,
        content     =
          {
            \c_@@_name_univ_zh_tl
            \skip_vertical:N \c_zero_skip
            \c_@@_name_thesis_zh_tl
            \c_@@_name_orig_decl_zh_tl
          } ,
        bottom-skip = 12 bp
      },
    { orig / text  }
      {
        format      = \zihao { -4 } \setbaselineskip { 24 bp } ,
        content     = \c_@@_orig_decl_text_zh_tl ,
        bottom-skip = 24 bp ,
        align       = normal
      },
    { orig / sign  }
      {
        format      = \zihao { 4 } \setbaselineskip { 30 bp } ,
        content     =
          {
            \@@_signature:N \c_@@_name_decl_author_zh_tl
            \skip_horizontal:n { 4 em } \hbox:n { }
          } ,
        bottom-skip = \c_zero_dim plus 2 fill ,
        align       = right
      },
    { auth / title }
      {
        format      = \zihao { 3 } \setbaselineskip { 30 bp }
                      \bfseries \heiti ,
        content     =
          {
            \c_@@_name_univ_zh_tl
            \skip_vertical:N \c_zero_skip
            \c_@@_name_thesis_zh_tl
            \c_@@_name_auth_decl_zh_tl
          } ,
        bottom-skip = 12 bp
      },
    { auth / text  }
      {
        format      = \zihao { -4 } \setbaselineskip { 24 bp } ,
        content     = \c_@@_auth_decl_text_zh_tl ,
        bottom-skip = 24 bp ,
        align       = normal
      },
    { auth / sign  }
      {
        format      = \zihao { 4 } \setbaselineskip { 30 bp } ,
        content     =
          {
            \@@_signature:N \c_@@_name_decl_author_zh_tl
            \hfill
            \@@_signature:N \c_@@_name_decl_supervisor_zh_tl
            \skip_horizontal:n { 2 em } \hbox:n { }
          } ,
        bottom-skip = \c_zero_dim plus 1 fill ,
        align       = normal
      }
  }
  {
    \@@_declare_component:nnn { copyright } #1
  }
\@@_declare_page:nn { copyright }
  {
    bookmark      = true ,
    bookmark-text = \c_@@_name_declaration_tl ,
    format        = \linespread { } \@@_set_language_zh: ,
    prefix        = copyright ,
    components    =
      {
        orig / title, orig / text, orig / sign,
        auth / title, auth / text, auth / sign
      }
  }
\msg_new:nnn { sjtutex } { require-pdfpages }
  {
    Add~"\token_to_str:N \usepackage{pdfpages}"~ in~ your~ preamble \\
    before~ inserting~ pages~ of~ external~ PDF.
  }
%    \end{macrocode}
%
% \changes{v2.0}{2022/12/20}{不再自动载入 \pkg{pdfpages} 宏包。}
%    \begin{macrocode}
\NewDocumentCommand \copyrightpage { O{ } }
  {
    \bool_if:NF \g_@@_review_bool
      {
        \tl_if_blank:nTF {#1}
          { \UseInstance { sjtu } { copyright } }
          {
            \cs_if_exist:NTF \includepdf
              {
                \bool_if:NTF \g_@@_openright_bool
                  { \cleardoublepage } { \clearpage }
                \@@_pdf_bookmark:nn { 0 } { \c_@@_name_declaration_tl }
                \includepdf {#1}
              }
              {
                \msg_warning:nn { sjtutex } { require-pdfpages }
                \UseInstance { sjtu } { copyright }
              }
          }
      }
  }
%</thesis>
%    \end{macrocode}
%
% \subsection{摘要}
%
% \changes{unreleased}{2023/11/30}{新增 \opt{style/keywords-format} 选项。}
% 关键词排版样式。
%    \begin{macrocode}
\keys_define:nn { sjtu / style }
  {
    keywords-format       .choice: ,
    keywords-format / plain .code:n =
      { \cs_set:Nn \@@_keywords_format:n { \noindent  { \bfseries ##1 } } } ,
    keywords-format / hang  .code:n =
      { \cs_set:Nn \@@_keywords_format:n { \@hangfrom { \bfseries ##1 } } } ,
    keywords-format      .initial:n = { plain }
  }
%    \end{macrocode}
%
% \changes{v2.0}{2023/03/17}{\env{abstract} 环境新增指定语言的可选参数。}
% \changes{unreleased}{2023/11/29}{\env{abstract} 环境添加目录条目,
% \env{abstract*} 环境对应修改为不添加目录条目。}
% 学位论文摘要环境。
%    \begin{macrocode}
%<*thesis>
\DeclareDocumentEnvironment { abstract  } { O{ zh } +b }
  {
    \@@_if_lang_valid:nTF {#1}
      {
        \use:c { @@_set_language_ #1 : }
        \exp_args:Nv \SJTU@head { c_@@_name_abstract_ #1 _tl }
        #2
        \clist_if_empty:cF { l_@@_info_keywords_ #1 _clist }
          {
            \par \mode_leave_vertical: \par
            \@@_keywords_format:n
              {
                \tl_use:c { c_@@_name_keywords_ #1 _tl }
                \tl_use:c { c_@@_name_info_sep_ #1 _tl }
              }
            \clist_use:cv { l_@@_info_keywords_ #1 _clist }
              { c_@@_name_item_sep_ #1 _tl }
            \par
          }
      }
      { \msg_error:nnn { sjtutex } { lang-validation } {#1} }
  } { }
\DeclareDocumentEnvironment { abstract* } { O{ zh } +b }
  {
    \@@_if_lang_valid:nTF {#1}
      {
        \use:c { @@_set_language_ #1 : }
        \exp_args:NNv \SJTU@head* { c_@@_name_abstract_ #1 _tl }
        #2
        \clist_if_empty:cF { l_@@_info_keywords_ #1 _clist }
          {
            \par \mode_leave_vertical: \par
            \@@_keywords_format:n
              {
                \tl_use:c { c_@@_name_keywords_ #1 _tl }
                \tl_use:c { c_@@_name_info_sep_ #1 _tl }
              }
            \clist_use:cv { l_@@_info_keywords_ #1 _clist }
              { c_@@_name_item_sep_ #1 _tl }
            \par
          }
      }
      { \msg_error:nnn { sjtutex } { lang-validation } {#1} }
  } { }
%</thesis>
%    \end{macrocode}
%
% 修复通用模板摘要段首缩进。
%    \begin{macrocode}
%<*!thesis>
\bool_if:NT \g_@@_titlepage_bool
  { \@@_appto_cmd:Nn \abstract { \par } }
%    \end{macrocode}
%
% 通用模板摘要后添加关键词。
%    \begin{macrocode}
\@@_preto_cmd:Nn \endabstract
  {
    \clist_if_empty:NF \l_@@_info_keywords_clist
      {
        \par \mode_leave_vertical: \par
        \@@_keywords_format:n
          {
            \c_@@_name_keywords_tl
            \c_@@_name_info_sep_tl
          }
        \clist_use:NV \l_@@_info_keywords_clist \c_@@_name_item_sep_tl
        \par
      }
  }
%</!thesis>
%    \end{macrocode}
%
% \subsection{目录}
%
% \changes{unreleased}{2023/11/29}{\tn{tableofcontents} 添加目录条目,
% \tn{tableofcontents*} 对应修改为不添加目录条目。}
% \begin{macro}{\tableofcontents,\tableofcontents*,\listoffigures,\listoffigures*,
% \listoftables,\listoftables*}
% 目录以及图表索引。
%    \begin{macrocode}
\DeclareDocumentCommand \tableofcontents { s }
  {
    \IfBooleanTF {#1}
      { \SJTU@head* { \contentsname } }
      { \SJTU@head  { \contentsname } }
    \group_begin:
      \cs_set:Npn \makebox [##1][##2]##3 { \, ##3 }
      \@starttoc { toc }
    \group_end:
  }
\NewDocumentCommand \SJTU@listof { m m s }
  {
    \IfBooleanTF {#3}
      { \SJTU@head* {#1} }
      { \SJTU@head  {#1} }
    \group_begin:
      \cs_set:Npn \makebox [##1][##2]##3 { \, ##3 }
      \exp_args:Nv \@starttoc { ext@ #2 }
    \group_end:
  }
\DeclareDocumentCommand \listoffigures { }
  { \SJTU@listof { \listfigurename } { figure } }
\DeclareDocumentCommand \listoftables  { }
  { \SJTU@listof { \listtablename  } { table  } }
%    \end{macrocode}
% \end{macro}
%
%    \begin{macrocode}
\tl_set:Nn \cftdotsep { 0.5 }
%<!article>\tl_set:Nn \cftchapleader { \bfseries \cftdotfill { \cftdotsep } }
%<*thesis>
\clist_map_inline:nn
  {
    { cft before chap skip } { 10 bp plus 1 pt } ,
    { cft chap   numwidth  } { 3.5 em } ,
    { cft sec    indent    } { 2   em } ,
    { cft sec    numwidth  } { 1.5 em } ,
    { cft subsec indent    } { 4   em } ,
    { cft subsec numwidth  } { 2.3 em }
  }
  { \skip_set:cn #1 }
%</thesis>
%    \end{macrocode}
%
% \changes{v2.0.1}{2023/03/31}{插图、表格和算法等索引不缩进。}
% \changes{v2.0.1}{2023/03/31}{调整插图、表格和算法等索引编号宽度。}
% \begin{macro}[int]{\l_@@_cft_presnum_clist,\@@_update_cft_presnum:nn}
% 图表清单标题前添加名称。
%    \begin{macrocode}
\clist_set:Nn \l_@@_cft_presnum_clist
  {
    { fig } { \figurename } ,
    { tab } { \tablename  }
  }
\cs_new:Npn \@@_update_cft_presnum:nn #1#2
  {
    \tl_set:cn { cft #1 presnum } { #2 \c_space_tl }
    \skip_zero:c { cft #1 indent }
%<article>    \skip_set:cn { cft #1 numwidth } { 1.8 em }
%<!article>    \skip_set:cn { cft #1 numwidth } { 2.8 em }
    \@@_skip_add_to_wd:cv { cft #1 numwidth } { cft #1 presnum }
  }
\ctex_at_end_preamble:n
  {
    \clist_map_inline:Nn \l_@@_cft_presnum_clist
      { \@@_update_cft_presnum:nn #1 }
  }
%    \end{macrocode}
% \end{macro}
%
% \subsection{预定义环境}
%
% 缩略语对照表。
%    \begin{macrocode}
%<*thesis>
\NewDocumentEnvironment { abbreviation  } { O{ \SJTU@abbrname } }
  {
    \chapter    {#1}
    \tl_clear:N \SJTU@style@float@font
  } { }
\NewDocumentEnvironment { abbreviation* } { O{ \SJTU@abbrname } }
  {
    \SJTU@head* {#1}
    \tl_clear:N \SJTU@style@float@font
  } { }
%    \end{macrocode}
%
% 符号对照表。
%    \begin{macrocode}
\NewDocumentEnvironment { nomenclature  } { O{ \SJTU@nomname } }
  {
    \chapter    {#1}
    \tl_clear:N \SJTU@style@float@font
  } { }
\NewDocumentEnvironment { nomenclature* } { O{ \SJTU@nomname } }
  {
    \SJTU@head* {#1}
    \tl_clear:N \SJTU@style@float@font
  } { }
%    \end{macrocode}
%
% \changes{v2.0}{2023/03/17}{移除 \env{summary} 环境。}
%
% 致谢,盲审模式下隐藏致谢。
%    \begin{macrocode}
\NewDocumentEnvironment { acknowledgements } { O{ \SJTU@ackname } +b }
  {
    \bool_if:NF \g_@@_review_bool
      {
        \SJTU@head {#1}
        #2
      }
  } { }
%    \end{macrocode}
%
% 发表论文与学术成果。
%    \begin{macrocode}
\newcounter { SJTU@bib }
\NewDocumentEnvironment { @bibliolist } { m }
  {
    \cs_if_exist_use:N \bibfont
    \list
      {
        \tl_if_blank:nTF {#1}
          { \hfill }
          { \@biblabel { \arabic{ SJTU@bib } } }
      }
      {
        \tl_if_blank:nTF {#1}
          {
            \skip_if_exist:NTF \bibhang
              { \dim_set_eq:NN \leftmargin \bibhang }
              { \dim_set:Nn    \leftmargin { 1 em } }
            \dim_set:Nn \itemindent { - \leftmargin }
          }
          {
            \@@_dim_set_to_wd:Nn \labelwidth { \@biblabel {#1} }
            \dim_set_eq:NN \leftmargin \labelwidth
            \dim_add:Nn    \leftmargin { \labelsep }
          }
        \skip_if_exist:NTF \bibitemsep
          {
            \skip_set_eq:NN \itemsep \bibitemsep
            \skip_if_exist:NT \bibparsep
              { \skip_set_eq:NN \parsep \bibparsep }
          }
          {
            \skip_if_exist:NT \bibsep
              {
                \skip_set_eq:NN \itemsep \bibsep
                \skip_zero:N    \parsep
              }
          }
        \@nmbrlisttrue
        \tl_set:Nn  \@listctr    { SJTU@bib }
        \cs_set:Npn \p@SJTU@bib  { }
        \cs_set:Npn \theSJTU@bib { \arabic { SJTU@bib } }
      }
      \sloppy
      \int_set:Nn \clubpenalty  { 4000 }
      \int_set_eq:NN \@clubpenalty \clubpenalty
      \int_set:Nn \widowpenalty { 4000 }
      \char_set_sfcode:nn { `\. } { 1000 }
  }
  {
    \cs_set:Npn \@noitemerr
      { \msg_warning:nnn { sjtutex } { empty-environment } { bibliolist } }
    \endlist
  }
\msg_new:nnn { sjtutex } { empty-environment }
  { Empty~ `#1'~ environment. }
\bool_new:N \l_@@_achievements_bool
\NewDocumentEnvironment { achievements } { O{ \SJTU@achvname } }
  {
    \SJTU@head {#1}
    \setcounter { SJTU@bib } { 0 }
    \bool_set_true:N \l_@@_achievements_bool
  } { }
\NewDocumentEnvironment { bibliolist  } { m +b }
  {
    \bool_if:NF \l_@@_achievements_bool
      {
        \msg_error:nnnn { sjtutex } { environment-validation }
          { bibliolist  } { achievements }
      }
    \bool_if:NF \g_@@_review_bool
      {
        \cs_set:Npn \@noitemerr { }
        \begin { @bibliolist } {#1}
        #2
        \end { @bibliolist }
      }
  } { }
\NewDocumentEnvironment { bibliolist* } { m +b }
  {
    \bool_if:NF \l_@@_achievements_bool
      {
        \msg_error:nnnn { sjtutex } { environment-validation }
          { bibliolist* } { achievements }
      }
    \bool_if:NT \g_@@_review_bool
      {
        \cs_set:Npn \@noitemerr { }
        \begin { @bibliolist } {#1}
        #2
        \end { @bibliolist }
      }
  } { }
\msg_new:nnn { sjtutex } { environment-validation }
  { `#1'~ is~ only~ valid~ in~ `#2'~ environment. }
%    \end{macrocode}
%
% 简历。
%    \begin{macrocode}
\NewDocumentEnvironment { resume } { O{ \SJTU@resumename } +b }
  {
    \bool_if:NF \g_@@_review_bool
      {
        \SJTU@head {#1}
        #2
      }
  } { }
%    \end{macrocode}
%
% \changes{v2.0}{2023/03/17}{\env{digest} 环境新增指定语言的可选参数。}
% 大摘要。
%    \begin{macrocode}
\NewDocumentEnvironment { digest } { O{ en } +b }
  {
    \@@_if_lang_valid:nTF {#1}
      {
        \AtEndDocument
          {
            \use:c { @@_set_language_ #1 : }
            \bool_if:NTF \g_@@_openright_bool
              { \cleardoublepage } { \clearpage }
            \pagenumbering { roman }
            \cs_gset:Nn \@@_thepage: { \arabic { page } }
            \cs_gset_eq:NN \addcontentsline \use_none:nnn
            \clist_map_inline:Nn \l_@@_counter_without_chapter_clist
              {
                \counterwithout {##1} { chapter }
                \setcounter     {##1} { 0 }
              }
            \@@_head_aux_s:nx { \SJTU@digestname }
              {
                \exp_not:N \MakeUppercase
                  { \exp_not:v { l_@@_info_title_ #1 _tl } }
              }
            #2
          }
      }
      { \msg_error:nnn { sjtutex } { lang-validation } {#1} }
  } { }
%</thesis>
%    \end{macrocode}
%
% \subsection{设置接口}
%
% \begin{macro}{\sjtusetup}
% 用户设置接口。
%    \begin{macrocode}
\NewDocumentCommand \sjtusetup { } { \keys_set:nn { sjtu } }
%    \end{macrocode}
% \end{macro}
%
% \changes{v2.0}{2022/12/28}{重新制定 \opt{sjtu/style} 域中的接口。}
% 定义元(meta)键值对。
%    \begin{macrocode}
\keys_define:nn { sjtu }
  {
    style .meta:nn = { sjtu / style } {#1} ,
    info  .meta:nn = { sjtu / info  } {#1} ,
    name  .meta:nn = { sjtu / name  } {#1}
  }
%    \end{macrocode}
%
% 兼容 \cls{sjtuthesis} 旧接口。
%    \begin{macrocode}
%<*thesis>
\keys_define:nn { sjtu / info }
  {
    title             .meta:n = { zh / title            = {#1} } ,
    title*            .meta:n = { en / title            = {#1} } ,
    display-title     .meta:n = { zh / display-title    = {#1} } ,
    display-title*    .meta:n = { en / display-title    = {#1} } ,
    subject           .meta:n = { zh / subject          = {#1} } ,
    subject*          .meta:n = { en / subject          = {#1} } ,
    keywords          .meta:n = { zh / keywords         = {#1} } ,
    keywords*         .meta:n = { en / keywords         = {#1} } ,
    author            .meta:n = { zh / author           = {#1} } ,
    author*           .meta:n = { en / author           = {#1} } ,
    supervisor        .meta:n = { zh / supervisor       = {#1} } ,
    supervisor*       .meta:n = { en / supervisor       = {#1} } ,
    assoc-supervisor  .meta:n = { zh / assoc-supervisor = {#1} } ,
    assoc-supervisor* .meta:n = { en / assoc-supervisor = {#1} } ,
    co-supervisor     .meta:n = { zh / co-supervisor    = {#1} } ,
    co-supervisor*    .meta:n = { en / co-supervisor    = {#1} } ,
    degree            .meta:n = { zh / degree           = {#1} } ,
    degree*           .meta:n = { en / degree           = {#1} } ,
    department        .meta:n = { zh / department       = {#1} } ,
    department*       .meta:n = { en / department       = {#1} } ,
    major             .meta:n = { zh / major            = {#1} } ,
    major*            .meta:n = { en / major            = {#1} } ,
    fund              .meta:n = { zh / fund             = {#1} } ,
    fund*             .meta:n = { en / fund             = {#1} } ,
    display-date      .meta:n = { zh / date             = {#1} } ,
    display-date*     .meta:n = { en / date             = {#1} }
  }
\keys_define:nn { sjtu / name }
  {
    abbreviation      .meta:n = { abbr  = {#1} } ,
    nomenclature      .meta:n = { nom   = {#1} } ,
    acknowledgements  .meta:n = { ack   = {#1} } ,
    publications      .meta:n = { achv  = {#1} } ,
    achievements      .meta:n = { achv  = {#1} }
  }
%</thesis>
%    \end{macrocode}
%
% 通用模板新接口。
%    \begin{macrocode}
%<*!thesis>
\NewDocumentCommand \subject  { m }
  { \keys_set:nn { sjtu / info } { subject  = {#1} } }
\NewDocumentCommand \keywords { m }
  { \keys_set:nn { sjtu / info } { keywords = {#1} } }
%</!thesis>
%    \end{macrocode}
%
% \subsection{其他宏包的设置}
%
% \changes{v2.0.3}{2023/09/23}{移除 \pkg{listings} 宏包预设。}
% 这些宏包并非格式要求,但是为了方便同学们使用,在这里进行简单设置。
%
% \subsubsection{\pkg{hyperref} 宏包}
%
%    \begin{macrocode}
\ctex_at_end_package:nn { hyperref }
  {
    \hypersetup
      {
        linktoc            = all,
        bookmarksdepth     = 2,
        bookmarksnumbered  = true,
        bookmarksopen      = true,
        bookmarksopenlevel = 1,
        unicode            = true,
        psdextra           = true,
        breaklinks         = true,
        pdfdisplaydoctitle = true
      }
    \int_new:N \g_@@_bookmark_int
    \cs_gset_protected:Npn \@@_pdf_bookmark:nn #1#2
      {
        \phantomsection
        \int_gincr:N \g_@@_bookmark_int
        \pdfbookmark [#1] {#2}
          { sjtubookmark. \int_use:N \g_@@_bookmark_int }
      }
    \cs_gset_eq:NN \@@_phantom_section: \phantomsection
    \pdfstringdefDisableCommands
      {
        \cs_set_eq:NN \\       \prg_do_nothing:
        \cs_set_eq:NN \quad    \c_empty_tl
        \cs_set_eq:NN \qquad   \c_empty_tl
        \cs_set_eq:NN \hspace  \use_none:n
      }
    \ctex_after_end_preamble:n
      {
        \hypersetup
          {
%<*thesis>
            pdftitle    = \l_@@_info_title_zh_tl ,
            pdfauthor   = \l_@@_info_author_zh_tl ,
            pdfsubject  = \l_@@_info_subject_zh_tl ,
            pdfkeywords = \l_@@_info_keywords_zh_clist
%</thesis>
%<*!thesis>
            pdftitle    = \@title ,
            pdfauthor   = \@author ,
            pdfsubject  = \l_@@_info_subject_tl ,
            pdfkeywords = \l_@@_info_keywords_clist
%</!thesis>
          }
      }
  }
%    \end{macrocode}
%
% \subsubsection{\pkg{threeparttable} 宏包}
%
%    \begin{macrocode}
\ctex_at_end_package:nn { threeparttable }
  { \tl_put_right:Nn \TPTnoteSettings { \footnotesize } }
%    \end{macrocode}
%
% \subsubsection{\pkg{longtable} 宏包}
%
%    \begin{macrocode}
\ctex_at_end_package:nn { longtable }
  { \AtBeginEnvironment { longtable } { \SJTU@style@float@font } }
%    \end{macrocode}
%
% \subsubsection{\pkg{amsthm} 宏包和 \pkg{ntheorem} 宏包}
%
%    \begin{macrocode}
\cs_new_protected:Nn \@@_new_theorems:
  {
    \clist_map_inline:nn
      {
        assumption, axiom, conjecture, corollary, definition, example,
        exercise, lemma, problem, proposition, theorem
      }
%<!article>      { \exp_args:Nnv  \newtheorem  {##1} { c_@@_name_ ##1 _tl } [ chapter ] }
%<article>      { \exp_args:Nnv  \newtheorem  {##1} { c_@@_name_ ##1 _tl } }
    \clist_map_inline:nn
      { remark, solution }
      { \exp_args:NNnv \newtheorem* {##1} { c_@@_name_ ##1 _tl } }
  }
%    \end{macrocode}
%
% \pkg{amsthm} 宏包。
%    \begin{macrocode}
\ctex_at_begin_package:nn { amsthm }
  {
    \cs_if_exist:NT \openbox
      {
        \cs_new_eq:NN \@@_save_openbox: \openbox
        \cs_undefine:N \openbox
      }
  }
\ctex_at_end_package:nn { amsthm }
  {
    \@@_cs_provide_eq:NN \QED \openbox
    \cs_if_exist:NT \@@_save_openbox:
      { \cs_set_eq:NN \openbox \@@_save_openbox: }
    \tl_set:Nn \qedsymbol { \ensuremath { \QED } }
    \RenewDocumentEnvironment { proof } { O{ \proofname } }
      {
        \par \pushQED { \qed }
        \normalfont \dim_zero:N \topsep
        \trivlist
        \item
          [
            \skip_horizontal:N \labelsep
            \bfseries \CJKsffamily #1 \@addpunct { \enskip }
          ]
        \ignorespaces
      }
      { \popQED \endtrivlist \legacy_if_set_false:n { @endpe } }
    \newtheoremstyle { sjtu }
      { } { } { \normalfont } { } { \bfseries \CJKsffamily } { } { \ccwd } { }
    \theoremstyle { sjtu }
    \@@_new_theorems:
  }
%    \end{macrocode}
%
% \pkg{ntheorem} 宏包。
%    \begin{macrocode}
\ctex_at_end_package:nn { ntheorem }
  {
    \@@_cs_provide_eq:NN \QED \c_empty_tl
    \theoremheaderfont { \bfseries \CJKsffamily }
    \theorembodyfont   { \normalfont }
    \theoremseparator  { \enskip     }
    \theoremsymbol { \ensuremath { \QED } }
    \qedsymbol     { \ensuremath { \QED } }
    \newtheorem* { proof } { \proofname }
    \theoremsymbol { }
    \@@_new_theorems:
  }
%    \end{macrocode}
%
% \subsubsection{\pkg{algorithm} 宏包和 \pkg{algorithm2e} 宏包}
%
%    \begin{macrocode}
\cs_new_protected:Npn \@@_newlistof:nnnnn #1#2#3#4#5
  {
    \exp_args:Nnv \newlistentry {#2} { ext@ #3 } { 0 }
    \exp_args:Ne \newcounter { \tl_use:c { ext@ #3 } depth }
    \exp_args:Ne \setcounter { \tl_use:c { ext@ #3 } depth } { 1 }
    \clist_put_right:Nn \l_@@_cft_presnum_clist { {#2} {#4} }
    \cs_set_eq:cc { l@ #3 } { l@ #2 }
    \exp_args:Nc \DeclareDocumentCommand { listof #1 s } { }
      { \SJTU@listof {#5} {#3} }
%<!article>    \SJTU@counterwithin { #3 } { chapter }
%<thesis>    \clist_put_right:Nn \l_@@_counter_without_chapter_clist {#3}
  }
%    \end{macrocode}
%
% \pkg{algorithm} 宏包。
%    \begin{macrocode}
\ctex_at_end_package:nn { algorithm }
  {
    \tl_set:Nn \fname@algorithm   { \SJTU@algorithmname     }
    \tl_set:Nn \listalgorithmname { \SJTU@listalgorithmname }
    \@@_newlistof:nnnnn { algorithm } { alg } { algorithm }
      { \fname@algorithm } { \listalgorithmname }
  }
%    \end{macrocode}
%
% \pkg{algorithm2e} 宏包。
%    \begin{macrocode}
%<!article>\ctex_at_begin_package:nn { algorithm2e }
%<!article>  { \cs_set_eq:NN \@@_save_chapter:w \@chapter }
\ctex_at_end_package:nn { algorithm2e }
  {
%<!article>    \cs_set_eq:NN \@chapter \@@_save_chapter:w
    \SetAlgorithmName { \SJTU@algorithmname     }
                      { \SJTU@algorithmname     }
                      { \SJTU@listalgorithmname }
    \SetAlgoCaptionSeparator { \enskip }
    \@@_newlistof:nnnnn { algorithm } { alg } { algocf }
      { \algorithmcfname } { \listalgorithmcfname }
    \ctex_patch_cmd:Nnn \algocf@latexcaption
      { \addcontentsline }
      { \caption@iflist { \addcontentsline } { \@gobblethree } }
  }
%    \end{macrocode}
%
% \subsubsection{\pkg{nomencl} 宏包}
%    \begin{macrocode}
\ctex_at_end_package:nn { nomencl }
  { \tl_set:Nn \nomname { \SJTU@nomname } }
%    \end{macrocode}
%
% \changes{v2.0.3}{2023/09/25}{添加 \pkg{siunitx} 本地化支持。}
% \subsubsection{\pkg{translations} 宏包}
%    \begin{macrocode}
\ctex_at_end_package:nn { translations }
  {
    \DeclareLanguage { chinese }
    \DeclareLanguageAlias { Chinese } { chinese }
  }
%    \end{macrocode}
%
% \subsubsection{\pkg{siunitx} 宏包}
%    \begin{macrocode}
\ctex_at_end_package:nn { siunitx }
  {
    \@ifpackageloaded { translations }
      {
        \DeclareTranslation { Chinese  } { and } { 和 }
        \DeclareTranslation { Japanese } { and } { と }
        \DeclareTranslation { Chinese  } { to~(numerical~range) } { \textasciitilde }
        \DeclareTranslation { Japanese } { to~(numerical~range) } { \textasciitilde }
      } { }
  }
%    \end{macrocode}
%
%    \begin{macrocode}
%</class>
%    \end{macrocode}
%
% \subsection{名称配置}
%    \begin{macrocode}
%<*name>
%<*zh|ja>
\clist_map_inline:nn
  {
    { year  } { 年 } ,
    { month } { 月 } ,
    { day   } { 日 }
  }
%<zh>  { \__sjtu_define_name:nnn { zh } #1 }
%<ja>  { \__sjtu_define_name:nnn { ja } #1 }
%</zh|ja>
%<*en>
\clist_const:Nn \c_@@_name_month_en_clist
  {
    January, February, March, April, May, June,
    July, August, September, October, November, December
  }
%</en>
%<*de>
\clist_const:Nn \c_@@_name_month_de_clist
  {
    Januar, Februar, März, April, Mai, Juni,
    Juli, August, September, Oktober, November, Dezember
  }
%</de>
%<*thesis>
%<*zh>
\@@_define_symbol:nn { white_square } { "25A1 }
\@@_define_name_from_clist:nnnn { zh }
  { degree_level } { \g_@@_thesis_type_int }
  { 学士, 硕士, 博士 }
%</zh>
%<*en>
\@@_define_name_from_clist:nnnn { en }
  { degree_level } { \g_@@_thesis_type_int }
  { Bachelor, Master, Doctor }
%</en>
%<*de>
\@@_define_name_from_clist:nnnn { de }
  { degree_level } { \g_@@_thesis_type_int }
  { Bachelor, Master, Doktor }
%</de>
%<*ja>
\@@_define_name_from_clist:nnnn { ja }
  { degree_level } { \g_@@_thesis_type_int }
  { 学士, 修士, 博士 }
%</ja>
%</thesis>
%<*zh>
\clist_map_inline:nn
  {
%<*thesis>
    { univ             } { 上海交通大学           } ,
    { author           } { 姓名                   } ,
    { id               } { 学号                   } ,
    { supervisor       } { 导师                   } ,
    { assoc_supervisor } { 副导师                 } ,
    { department       } { 院系                   } ,
    { co_supervisor    } { 联合导师               } ,
    { major            } { 学科 / 专业            } ,
    { degree           } { 申请学位               } ,
    { thesis           } { 学位论文               } ,
    { title_page       } { 题名页                 } ,
    { declaration      } { 原创性声明及使用授权书 } ,
    { orig_decl        } { 原创性声明             } ,
    { auth_decl        } { 使用授权书             } ,
    { decl_author      } { 学位论文作者           } ,
    { decl_supervisor  } { 指导教师               } ,
    { abstract         } { 摘 \protect \quad 要   } ,
%</thesis>
    { keywords         } { 关键词                 } ,
    { info_sep         } { : \null               } ,
    { item_sep         } { ,                     }
  }
%<thesis>  { \@@_define_name:nnn { zh } #1 }
%<!thesis>  { \@@_define_name:nn #1 }
%</zh>
%<*en>
\clist_map_inline:nn
  {
%<*thesis>
    { univ             } { Shanghai~ Jiao~ Tong~ University } ,
    { address          } { Shanghai,~ P.R.~ China           } ,
    { author           } { Author                           } ,
    { supervisor       } { Supervisor                       } ,
    { assoc_supervisor } { Assoc.~ Supervisor               } ,
    { co_supervisor    } { Co-supervisor                    } ,
    { title_page       } { Title~ Page                      } ,
    { declaration      } { Statutory~ Declaration           } ,
    { abstract         } { Abstract                         } ,
%</thesis>
    { keywords         } { Key~words                        } ,
    { info_sep         } { :~                               } ,
    { item_sep         } { ,~                               }
  }
%<thesis>  { \@@_define_name:nnn { en } #1 }
%<!thesis>  { \@@_define_name:nn #1 }
%</en>
%<*de>
\clist_map_inline:nn
  {
%<*thesis>
    { univ             } { Shanghai~ Jiao~ Tong~ Universität } ,
    { address          } { Shanghai,~ VR~ China              } ,
    { author           } { Autor/in                          } ,
    { supervisor       } { Betreuer/in                       } ,
    { assoc_supervisor } { Zweitbetreuer/in                  } ,
    { co_supervisor    } { Co-Betreuer/in                    } ,
    { title_page       } { Titelblatt                        } ,
    { declaration      } { Eidesstattliche~ Erklärung        } ,
    { abstract         } { Abstrakt                          } ,
%</thesis>
    { keywords         } { Schlüsselwörter                   } ,
    { info_sep         } { :~                                } ,
    { item_sep         } { ,~                                }
  }
%<thesis>  { \@@_define_name:nnn { de } #1 }
%<!thesis>  { \@@_define_name:nn #1 }
%</de>
%<*ja>
\clist_map_inline:nn
  {
%<*thesis>
    { univ             } { 上海交通大学         } ,
    { address          } { 中国・上海           } ,
    { author           } { 氏名                 } ,
    { supervisor       } { 指導教員             } ,
    { assoc_supervisor } { 副指導教員           } ,
    { co_supervisor    } { 共同指導             } ,
    { thesis           } { 学位請求論文         } ,
    { title_page       } { 標題紙               } ,
    { declaration      } { 誓約書・公表許諾書   } ,
    { abstract         } { 要 \protect \quad 旨 } ,
%</thesis>
    { keywords         } { キーワード           } ,
    { info_sep         } { : \null             } ,
    { item_sep         } { \quad                }
  }
%<thesis>  { \@@_define_name:nnn { ja } #1 }
%<!thesis>  { \@@_define_name:nn #1 }
%</ja>
%<*thesis>
%<*zh>
\tl_const:Nn \c_@@_orig_decl_text_zh_tl
  {
    本人郑重声明:所呈交的学位论文,是本人在导师的指导下,独立进行研究工
    作所取得的成果。除文中已经注明引用的内容外,本论文不包含任何其他个人
    或集体已经发表或撰写过的作品成果。对本文的研究做出重要贡献的个人和集
    体,均已在文中以明确方式标明。本人完全知晓本声明的法律后果由本人承
    担。
  }
\tl_const:Nn \c_@@_auth_decl_text_zh_tl
  {
    本人同意学校保留并向国家有关部门或机构送交论文的复印件和电子版,允许
    论文被查阅和借阅。 \par
    \vskip 6 bp
    \noindent
    本学位论文属于: \par
    { \c_@@_symbol_white_square_tl } \textbf { 公开论文 } \par
    { \c_@@_symbol_white_square_tl } \textbf { 内部论文 },
      保密 { \c_@@_symbol_white_square_tl }~ 1~ 年 /
           { \c_@@_symbol_white_square_tl }~ 2~ 年 /
           { \c_@@_symbol_white_square_tl }~ 3~ 年,
      过保密期后适用本授权书。 \par
    { \c_@@_symbol_white_square_tl } \textbf { 秘密论文 },
      保密 \underline { \hspace { 2 em } } 年(不超过~ 10~ 年),
      过保密期后适用本授权书。 \par
    { \c_@@_symbol_white_square_tl } \textbf { 机密论文 },
      保密 \underline { \hspace { 2 em } } 年(不超过~ 20~ 年),
      过保密期后适用本授权书。 \par
    \hspace { 6 em }(请在以上方框内选择打“ \ensuremath { \checkmark } ”)
  }
\tl_const:Nn \c_@@_signature_text_zh_tl
  {
    签名: \\
    日期: \hspace { \stretch { 3 } } 年
           \hspace { \stretch { 2 } } 月
           \hspace { \stretch { 2 } } 日
  }
%</zh>
%</thesis>
%</name>
%    \end{macrocode}
%
%    \begin{macrocode}
%<*scheme>
\clist_map_inline:nn
  {
%<*zh>
    { assumption  } { 假设        } ,
    { axiom       } { 公理        } ,
    { conjecture  } { 猜想        } ,
    { corollary   } { 推论        } ,
    { definition  } { 定义        } ,
    { example     } { 例          } ,
    { exercise    } { 练习        } ,
    { lemma       } { 引理        } ,
    { problem     } { 问题        } ,
    { proposition } { 命题        } ,
    { remark      } { 注          } ,
    { solution    } { 解          } ,
    { theorem     } { 定理        }
%</zh>
%<*en>
    { assumption  } { Assumption  } ,
    { axiom       } { Axiom       } ,
    { conjecture  } { Conjecture  } ,
    { corollary   } { Corollary   } ,
    { definition  } { Definition  } ,
    { example     } { Example     } ,
    { exercise    } { Exercise    } ,
    { lemma       } { Lemma       } ,
    { problem     } { Problem     } ,
    { proposition } { Proposition } ,
    { remark      } { Remark      } ,
    { solution    } { Solution    } ,
    { theorem     } { Theorem     }
%</en>
%<*de>
    { assumption  } { Annahme     } ,
    { axiom       } { Axiom       } ,
    { conjecture  } { Hypothese   } ,
    { corollary   } { Korollar    } ,
    { definition  } { Definition  } ,
    { example     } { Beispiel    } ,
    { exercise    } { Übung       } ,
    { lemma       } { Lemma       } ,
    { problem     } { Problem     } ,
    { proposition } { Proposition } ,
    { remark      } { Anmerkung   } ,
    { solution    } { Lösung      } ,
    { theorem     } { Theorem     }
%</de>
%<*ja>
    { assumption  } { 仮定        } ,
    { axiom       } { 公理        } ,
    { conjecture  } { 予想        } ,
    { corollary   } { 系          } ,
    { definition  } { 定義        } ,
    { example     } { 例          } ,
    { exercise    } { 練習        } ,
    { lemma       } { 補題        } ,
    { problem     } { 問題        } ,
    { proposition } { 命題        } ,
    { remark      } { 注意        } ,
    { solution    } { 解法        } ,
    { theorem     } { 定理        }
%</ja>
  }
  { \@@_define_name:nn #1 }
%    \end{macrocode}
%
%    \begin{macrocode}
%</scheme>
%    \end{macrocode}
%
% \end{implementation}
%
% \Finale
%
\endinput
