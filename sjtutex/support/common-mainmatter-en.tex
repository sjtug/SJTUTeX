\chapter{Introduction}

\section{Foreword}

\lipsum[1]

\section{The main content of this paper}

\lipsum[2]

\section{The significance of this article}

\lipsum[3]

\section{Summary}

\lipsum[4]


\chapter{Guide to Formatting Body Text}

\section{Basic text format requirements}

The content of the thesis should generally consist of ten main parts\cite{Jia2000}, in order:
1. cover
2. Chinese abstract
3. English abstract
4. table of contents
5. symbol description
6. thesis body
7. references
8. appendices
9. acknowledgements
10. published academic papers during degree study
\footnote[1]{Test footnote Number 1.}
\footnote[11]{Test footnote Number 11.}
\footnote[21]{Test footnote Number 21.}
\footnote[32]{Test footnote Number 32.}
\footnote[50]{Test footnote Number 50.}

\section{Word count requirements}

\subsection{Undergraduate thesis requirements}

\lipsum[5]

\section{Summary}

\lipsum[6]


\chapter{Guide to Formatting Figure, Table and Formula}

\section{Guide to formatting figure}

\begin{figure}[ht]
  \centering
  \includegraphics[width=4cm]{example-image.pdf}
  \caption{Example image}
  \label{fig:example}
\end{figure}

\begin{ThreePartTable}
  \begin{TableNotes}
    \item[a] A note.
    \item[b] Another note.
  \end{TableNotes}
  \begin{longtable}[c]{c*{6}{r}}
    \caption{Experimental data}
    \label{tab:performance} \\
    \toprule
    Test Program & \multicolumn{1}{c}{Run} & \multicolumn{1}{c}{Sync}
      & \multicolumn{1}{c}{Checkpoint} & \multicolumn{1}{c}{Rollback}
      & \multicolumn{1}{c}{Transfer} & \multicolumn{1}{c}{Checkpoint} \\
    & \multicolumn{1}{c}{Time (s)} & \multicolumn{1}{c}{Time (s)}
      & \multicolumn{1}{c}{Time (s)} & \multicolumn{1}{c}{Time (s)}
      & \multicolumn{1}{c}{Time (s)} &  File Size (KB)\\
    \midrule
    \endfirsthead
    \multicolumn{7}{r}{\textbf{Table~\thetable~(continued)}} \\
    \toprule
    Test Program & \multicolumn{1}{c}{Run} & \multicolumn{1}{c}{Sync}
      & \multicolumn{1}{c}{Checkpoint} & \multicolumn{1}{c}{Rollback}
      & \multicolumn{1}{c}{Transfer} & \multicolumn{1}{c}{Checkpoint} \\
    & \multicolumn{1}{c}{Time (s)} & \multicolumn{1}{c}{Time (s)}
      & \multicolumn{1}{c}{Time (s)} & \multicolumn{1}{c}{Time (s)}
      & \multicolumn{1}{c}{Time (s)} &  File Size (KB)\\
    \midrule
    \endhead
    \hline
    \endfoot
    \insertTableNotes
    \endlastfoot
    CG.A.2 & 23.05 & 0.002 & 0.116 & 0.035 & 0.589 & 32491 \\
    CG.A.4 & 15.06 & 0.003 & 0.067 & 0.021 & 0.351 & 18211 \\
    CG.A.8 & 13.38 & 0.004 & 0.072 & 0.023 & 0.210 & 9890 \\
    CG.B.2 & 867.45 & 0.002 & 0.864 & 0.232 & 3.256 & 228562 \\
    CG.B.4 & 501.61 & 0.003 & 0.438 & 0.136 & 2.075 & 123862 \\
    CG.B.8 & 384.65 & 0.004 & 0.457 & 0.108 & 1.235 & 63777 \\
    MG.A.2 & 112.27 & 0.002 & 0.846 & 0.237 & 3.930 & 236473 \\
    MG.A.4 & 59.84 & 0.003 & 0.442 & 0.128 & 2.070 & 123875 \\
    MG.A.8 & 31.38 & 0.003 & 0.476 & 0.114 & 1.041 & 60627 \\
    MG.B.2 & 526.28 & 0.002 & 0.821 & 0.238 & 4.176 & 236635 \\
    MG.B.4 & 280.11 & 0.003 & 0.432 & 0.130 & 1.706 & 123793 \\
    MG.B.8 & 148.29 & 0.003 & 0.442 & 0.116 & 0.893 & 60600 \\
    LU.A.2 & 2116.54 & 0.002 & 0.110 & 0.030 & 0.532 & 28754 \\
    LU.A.4 & 1102.50 & 0.002 & 0.069 & 0.017 & 0.255 & 14915 \\
    LU.A.8 & 574.47 & 0.003 & 0.067 & 0.016 & 0.192 & 8655 \\
    LU.B.2 & 9712.87 & 0.002 & 0.357 & 0.104 & 1.734 & 101975 \\
    LU.B.4 & 4757.80 & 0.003 & 0.190 & 0.056 & 0.808 & 53522 \\
    LU.B.8 & 2444.05 & 0.004 & 0.222 & 0.057 & 0.548 & 30134 \\
    EP.A.2 & 123.81 & 0.002 & 0.010 & 0.003 & 0.074 & 1834 \\
    EP.A.4 & 61.92 & 0.003 & 0.011 & 0.004 & 0.073 & 1743 \\
    EP.A.8 & 31.06 & 0.004 & 0.017 & 0.005 & 0.073 & 1661 \\
    EP.B.2 & 495.49 & 0.001 & 0.009 & 0.003 & 0.196 & 2011 \\
    EP.B.4 & 247.69 & 0.002 & 0.012 & 0.004 & 0.122 & 1663 \\
    EP.B.8 & 126.74 & 0.003 & 0.017 & 0.005 & 0.083 & 1656 \\
    SP.A.2 & 123.81 & 0.002 & 0.010 & 0.003 & 0.074 & 1854 \\
    SP.A.4 & 51.92 & 0.003 & 0.011 & 0.004 & 0.073 & 1543 \\
    SP.A.8 & 31.06 & 0.004 & 0.017 & 0.005 & 0.073 & 1671 \\
    SP.B.2 & 495.49 & 0.001 & 0.009 & 0.003 & 0.196 & 2411 \\
    SP.B.4 \tnote{a} & 247.69 & 0.002 & 0.014 & 0.006 & 0.152 & 2653 \\
    SP.B.8 \tnote{b} & 126.74 & 0.003 & 0.017 & 0.005 & 0.082 & 1755 \\
    \bottomrule
  \end{longtable}
\end{ThreePartTable}

\section{Formula format}

\begin{equation}\label{eq:example}
  \frac{1}{\mu}\nabla^2\mathbf{A}-j\omega\sigma\mathbf{A}
  -\nabla\left(\frac{1}{\mu}\right)\times(\nabla\times\mathbf{A})
  +\mathbf{J}_0=0
\end{equation}

\section{Codeblock Example}

\begin{codeblock}[language=C]
#include <stdio.h>
#include <unistd.h>
#include <sys/types.h>
#include <sys/wait.h>

int main() {
  pid_t pid;

  switch ((pid = fork())) {
  case -1:
    printf("fork failed\n");
    break;
  case 0:
    /* child calls exec */
    execl("/bin/ls", "ls", "-l", (char*)0);
    printf("execl failed\n");
    break;
  default:
    /* parent uses wait to suspend execution until child finishes */
    wait((int*)0);
    printf("is completed\n");
    break;
  }
  return 0;
}
\end{codeblock}

\section{Algorithm Example}

\begin{algorithm}[htb]
  \caption{Algorithm Example}
  \label{algo:algorithm}
  \small
  \SetAlgoLined
  \KwData{this text}
  \KwResult{how to write algorithm with \LaTeXe }

  initialization\;
  \While{not at end of this document}{
    read current\;
    \eIf{understand}{
      go to next section\;
      current section becomes this one\;
    }{
      go back to the beginning of current section\;
    }
  }
\end{algorithm}

\section{Summary}

\lipsum[7]

\chapter{Conclusions}

\section{Main conclusions}

\lipsum[8]

\section{Research outlook}

\lipsum[9]

