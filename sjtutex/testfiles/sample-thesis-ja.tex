% !TeX encoding = UTF-8

% 载入 SJTUThesis 模版
\documentclass[type=master,lang=ja]{sjtuthesis}

\usepackage{lipsum}
\usepackage{zhlipsum}
\usepackage{booktabs}
\usepackage{longtable}
\usepackage{threeparttable}
\usepackage{threeparttablex}
\usepackage[backend=biber,style=gb7714-2015]{biblatex}
\addbibresource{ref.bib}
\usepackage{hyperref}

\usepackage{bxjalipsum}
\sjtusetup{
  %
  %******************************
  % 注意:
  %   1. 配置里面不要出现空行
  %   2. 不需要的配置信息可以删除
  %******************************
  %
  % 信息录入
  %
  info = {%
      %
      % 标题
      %
      zh / title           = {上海交通大学学位论文 \LaTeX{} 模板示例文档},
      en / title           = {A Sample Document for \LaTeX-based SJTU Thesis Template},
      %
      % 标题页标题
      %   可使用“\\”命令手动控制换行
      %
      zh / display-title   = {上海交通大学学位论文\\ \LaTeX{} 模板示例文档},
      en / display-title   = {A Sample Document \\ for \LaTeX-based SJTU Thesis Template},
      %
      % 关键词
      %
      zh / keywords        = {上海交大, 饮水思源, 爱国荣校},
      en / keywords        = {SJTU, master thesis, XeTeX/LaTeX template},
      %
      % 姓名
      %
      zh / author          = {某\quad{}某},
      en / author          = {Mo Mo},
      %
      % 指导教师
      %
      zh / supervisor      = {某某教授},
      en / supervisor      = {Prof. Mou Mou},
      %
      % 学号
      %
      id              = {0010900990},
      %
      % 专业
      %
      zh / major           = {某某专业},
      en / major           = {A Very Important Major},
      %
      % 所属院系
      %
      zh / department      = {某某系},
      en / department      = {School of XXX},
      %
      % 日期
      %   使用 ISO 格式 (yyyy-mm-dd 或 yyyy-mm);默认为当前时间
      %
      % date            = {2014-12-17},
      %
      % 资助基金
      %
      % zh / fund   = {
      %                 {国家 973 项目 (No. 2025CB000000)},
      %                 {国家自然科学基金 (No. 81120250000)},
      %               },
      % en / fund   = {
      %                 {National Basic Research Program of China (Grant No. 2025CB000000)},
      %                 {National Natural Science Foundation of China (Grant No. 81120250000)},
      %               },
    },
  style = {%
      title-logo-color = { red }
    },
}

\makeatletter
\def\@biblabel#1{(#1)}
\makeatother
% 日语模板补充字段
\sjtusetup{
  info = {
    ja / title           = {上海交通大学学位論文見本},
    ja / display-title   = {上海交通大学学位論文見本},
    ja / keywords        = {学位論文, 論文書式, 規範化, 見本},
    ja / degree          = {工学修士},
    ja / author          = {某},
    ja / supervisor      = {某教授},
    ja / major           = {某メジャー},
    ja / department      = {某カレッジ},
    % ja / fund   = {
    %                 {国家973プロジェクト (No. 2025CB000000)},
    %                 {中国国家自然科学基金会 (No. 81120250000)},
    %               },
  }
}

\begin{document}

% 标题页
\maketitle

% 原创性声明及使用授权书
\copyrightpage

% 前置部分
\frontmatter

\begin{abstract}[zh]
  中文摘要应该将学位论文的内容要点简短明了地表达出来,应该包含论文中的基本信息,
体现科研工作的核心思想。摘要内容应涉及本项科研工作的目的和意义、研究方法、研究
成果、结论及意义。注意突出学位论文中具有创新性的成果和新见解的部分。摘要中不宜
使用公式、化学结构式、图表和非公知公用的符号和术语,不标注引用文献编号。硕士学
位论文中文摘要字数为 500 字左右,博士学位论文中文摘要字数为 800 字左右。英文摘
要内容应与中文摘要内容一致。

摘要页的下方注明本文的关键词(4\textasciitilde 6个)。

\end{abstract}

\begin{abstract}[en]
  As a primary means of demonstrating research findings for undergraduate
students, dissertation is a systematic and standardized record of the new
inventions, theories or insights obtained by the author in the research work.
It can not only function as an important reference when students pursue further
studies, but also contribute to scientific research and social development.

This template is therefore made to improve the quality of undergraduates'
dissertation and to further standardize it both in content and in format.

\end{abstract}

\begin{abstract}[ja]
  学位論文は大学生が科学研究の成果をアピールする主要な形式である。
研究の中で得られた新しい発明や理論、見解を集中的に示すもので、
研究分野における重要な文献資料及び社会の貴重な財産ともなる。

学生の学位論文の品質を高め、学位論文の内容や形式上の規範化及び
統一化を実現するために、本見本を制作することに至った。

\end{abstract}

% 目录
\tableofcontents
% 插图索引
\listoffigures*
% 表格索引
\listoftables*
% 算法索引
% \listofalgorithms*

% 主体部分
\mainmatter

\chapter{序論}

\section{前書き}

\jalipsum[1]{kusamakura}

\section{本研究の主要内容}

\jalipsum[2]{kusamakura}

\section{本研究の意義}

\jalipsum[3]{kusamakura}

\section{先行研究}

\jalipsum[4]{kusamakura}


\chapter{正文の文字書式}

\section{論文の正文}

論文の正文が主体で、一般的に標題、叙述、図、表、公式などからなる。
理論分析、計算方法、実験とテスト方法などの方法を用い、...

論文は一般的に十部分の内容から構成される。
\footnote[1]{脚注 1}
\footnote[11]{脚注 11}
\footnote[21]{脚注 21}
\footnote[32]{脚注 32}
\footnote[50]{脚注 50}

\section{字数要求}

\subsection{学士論文の要求}

日本語の論文の字数は12000字以上で、読書報告書の字数は15000字以上となる。

\section{本章のまとめ}

\jalipsum[5]{kusamakura}


\chapter{図表、公式の書式}

\section{図表の書式}

\begin{figure}[ht]
  \centering
  \includegraphics[width=4cm]{example-image.pdf}
  \caption{例}
  \label{fig:example}
\end{figure}

\begin{ThreePartTable}
  \begin{TableNotes}
    \item[a] 脚注
    \item[b] 脚注
  \end{TableNotes}
  \begin{longtable}[c]{c*{6}{r}}
    \caption{実験データ}
    \label{tab:performance} \\
    \toprule
    試験プログラム & \multicolumn{1}{c}{A} & \multicolumn{1}{c}{B}
      & \multicolumn{1}{c}{C} & \multicolumn{1}{c}{D}
      & \multicolumn{1}{c}{E} & \multicolumn{1}{c}{C} \\
    & \multicolumn{1}{c}{時間 (s)} & \multicolumn{1}{c}{時間 (s)}
      & \multicolumn{1}{c}{時間 (s)} & \multicolumn{1}{c}{時間 (s)}
      & \multicolumn{1}{c}{時間 (s)} &  資料(KB)\\
    \midrule
    \endfirsthead
    \multicolumn{7}{l}{\textbf{表のつづき~\thetable}} \\
    \toprule
    試験プログラム & \multicolumn{1}{c}{A} & \multicolumn{1}{c}{B}
      & \multicolumn{1}{c}{C} & \multicolumn{1}{c}{D}
      & \multicolumn{1}{c}{E} & \multicolumn{1}{c}{C} \\
    & \multicolumn{1}{c}{時間 (s)} & \multicolumn{1}{c}{時間 (s)}
      & \multicolumn{1}{c}{時間 (s)} & \multicolumn{1}{c}{時間 (s)}
      & \multicolumn{1}{c}{時間 (s)} &  資料(KB)\\
    \midrule
    \endhead
    \hline
    \endfoot
    \insertTableNotes
    \endlastfoot
    CG.A.2 & 23.05 & 0.002 & 0.116 & 0.035 & 0.589 & 32491 \\
    CG.A.4 & 15.06 & 0.003 & 0.067 & 0.021 & 0.351 & 18211 \\
    CG.A.8 & 13.38 & 0.004 & 0.072 & 0.023 & 0.210 & 9890 \\
    CG.B.2 & 867.45 & 0.002 & 0.864 & 0.232 & 3.256 & 228562 \\
    CG.B.4 & 501.61 & 0.003 & 0.438 & 0.136 & 2.075 & 123862 \\
    CG.B.8 & 384.65 & 0.004 & 0.457 & 0.108 & 1.235 & 63777 \\
    MG.A.2 & 112.27 & 0.002 & 0.846 & 0.237 & 3.930 & 236473 \\
    MG.A.4 & 59.84 & 0.003 & 0.442 & 0.128 & 2.070 & 123875 \\
    MG.A.8 & 31.38 & 0.003 & 0.476 & 0.114 & 1.041 & 60627 \\
    MG.B.2 & 526.28 & 0.002 & 0.821 & 0.238 & 4.176 & 236635 \\
    MG.B.4 & 280.11 & 0.003 & 0.432 & 0.130 & 1.706 & 123793 \\
    MG.B.8 & 148.29 & 0.003 & 0.442 & 0.116 & 0.893 & 60600 \\
    LU.A.2 & 2116.54 & 0.002 & 0.110 & 0.030 & 0.532 & 28754 \\
    LU.A.4 & 1102.50 & 0.002 & 0.069 & 0.017 & 0.255 & 14915 \\
    LU.A.8 & 574.47 & 0.003 & 0.067 & 0.016 & 0.192 & 8655 \\
    LU.B.2 & 9712.87 & 0.002 & 0.357 & 0.104 & 1.734 & 101975 \\
    LU.B.4 & 4757.80 & 0.003 & 0.190 & 0.056 & 0.808 & 53522 \\
    LU.B.8 & 2444.05 & 0.004 & 0.222 & 0.057 & 0.548 & 30134 \\
    EP.A.2 & 123.81 & 0.002 & 0.010 & 0.003 & 0.074 & 1834 \\
    EP.A.4 & 61.92 & 0.003 & 0.011 & 0.004 & 0.073 & 1743 \\
    EP.A.8 & 31.06 & 0.004 & 0.017 & 0.005 & 0.073 & 1661 \\
    EP.B.2 & 495.49 & 0.001 & 0.009 & 0.003 & 0.196 & 2011 \\
    EP.B.4 & 247.69 & 0.002 & 0.012 & 0.004 & 0.122 & 1663 \\
    EP.B.8 & 126.74 & 0.003 & 0.017 & 0.005 & 0.083 & 1656 \\
    SP.A.2 & 123.81 & 0.002 & 0.010 & 0.003 & 0.074 & 1854 \\
    SP.A.4 & 51.92 & 0.003 & 0.011 & 0.004 & 0.073 & 1543 \\
    SP.A.8 & 31.06 & 0.004 & 0.017 & 0.005 & 0.073 & 1671 \\
    SP.B.2 & 495.49 & 0.001 & 0.009 & 0.003 & 0.196 & 2411 \\
    SP.B.4 \tnote{a} & 247.69 & 0.002 & 0.014 & 0.006 & 0.152 & 2653 \\
    SP.B.8 \tnote{b} & 126.74 & 0.003 & 0.017 & 0.005 & 0.082 & 1755 \\
    \bottomrule
  \end{longtable}
\end{ThreePartTable}

\section{公式の書式}

\begin{equation}\label{eq:example}
  \frac{1}{\mu}\nabla^2\mathbf{A}-j\omega\sigma\mathbf{A}
  -\nabla\left(\frac{1}{\mu}\right)\times(\nabla\times\mathbf{A})
  +\mathbf{J}_0=0
\end{equation}

\begin{equation}
  \int_{a}^b f(x)\,\mathrm{d}x=\lim_{|P|\rightarrow 0}\sum_{i=1}^n f(\xi_i)\increment x_i
\end{equation}

\section{コード環境}

\begin{codeblock}[language=C]
#include <stdio.h>
#include <unistd.h>
#include <sys/types.h>
#include <sys/wait.h>

int main() {
  pid_t pid;

  switch ((pid = fork())) {
  case -1:
    printf("fork failed\n");
    break;
  case 0:
    /* child calls exec */
    execl("/bin/ls", "ls", "-l", (char*)0);
    printf("execl failed\n");
    break;
  default:
    /* parent uses wait to suspend execution until child finishes */
    wait((int*)0);
    printf("is completed\n");
    break;
  }
  return 0;
}
\end{codeblock}

\section{アルゴリズム環境}

\begin{algorithm}[htb]
  \caption{アルゴリズム例}
  \label{algo:algorithm}
  \small
  \SetAlgoLined
  \KwData{this text}
  \KwResult{how to write algorithm with \LaTeXe }

  initialization\;
  \While{not at end of this document}{
    read current\;
    \eIf{understand}{
      go to next section\;
      current section becomes this one\;
    }{
      go back to the beginning of current section\;
    }
  }
\end{algorithm}

\section{本章のまとめ}

\jalipsum[6]{kusamakura}

\chapter{結論}

\section{主要結論}

\jalipsum[7]{kusamakura}

\section{今後の展望}

\jalipsum[8]{kusamakura}



% 参考文献
\begin{thebibliography}{99}
  \item 大内田鶴子「草の根の自治の技術」『社会学評論』1999(4),513-530
  \item K・J・アロー 長名寛明訳『社会的選択と個人的評価』日本経済新聞社 1977,94
  \item 鈴木光男・武藤滋夫『協力ゲームの理論』東京大学出版会 1985,149
  \item 鈴木孝夫『ことばと文化』岩波書店 1973
  \item 舩橋晴俊「環境問題の未来と社会変動――社会の自己破壊性と自己組織性」
        舩橋晴俊・飯島伸子編『講座社会学 12 環境』東京大学出版会 1998,191-224
  \item 真田信治 他『社会言語学』おうふう 1992
  \item 真木悠介『現代社会の存立構造』筑摩書房 1977a
  \item ------『気流の鳴る音』筑摩書房 1977b
  \item 横山紀子「インプットの効果を高める教室活動:日本語教育における実践」国際交流基金『日本語国際センター紀要』第10号 1999,37-53
  \item http://www.ninjal.ac.jp/(国立国語研究所)(作成日時、更新日時、アクセス日時)
\end{thebibliography}

% 附录
\appendix

% 附录中图表不加入索引
\captionsetup{list=no}

\begin{nomenclature}

\begin{longtable}{rl}
  $\epsilon$ & 誘電率 \\
  $\mu$      & 透過性 \\
  $\epsilon$ & 誘電率 \\
  $\mu$      & 透過性 \\
  $\epsilon$ & 誘電率 \\
  $\mu$      & 透過性 \\
  $\epsilon$ & 誘電率 \\
  $\mu$      & 透過性 \\
  $\epsilon$ & 誘電率 \\
  $\mu$      & 透過性 \\
  $\epsilon$ & 誘電率 \\
  $\mu$      & 透過性 \\
  $\epsilon$ & 誘電率 \\
  $\mu$      & 透過性 \\
  $\epsilon$ & 誘電率 \\
  $\mu$      & 透過性 \\
  $\epsilon$ & 誘電率 \\
  $\mu$      & 透過性 \\
  $\epsilon$ & 誘電率 \\
  $\mu$      & 透過性 \\
  $\epsilon$ & 誘電率 \\
  $\mu$      & 透過性 \\
  $\epsilon$ & 誘電率 \\
  $\mu$      & 透過性 \\
  $\epsilon$ & 誘電率 \\
  $\mu$      & 透過性 \\
  $\epsilon$ & 誘電率 \\
  $\mu$      & 透過性 \\
  $\epsilon$ & 誘電率 \\
  $\mu$      & 透過性 \\
  $\epsilon$ & 誘電率 \\
  $\mu$      & 透過性 \\
  $\epsilon$ & 誘電率 \\
  $\mu$      & 透過性 \\
  $\epsilon$ & 誘電率 \\
  $\mu$      & 透過性 \\
  $\epsilon$ & 誘電率 \\
  $\mu$      & 透過性 \\
  $\epsilon$ & 誘電率 \\
  $\mu$      & 透過性 \\
  $\epsilon$ & 誘電率 \\
  $\mu$      & 透過性 \\
  $\epsilon$ & 誘電率 \\
  $\mu$      & 透過性 \\
  $\epsilon$ & 誘電率 \\
  $\mu$      & 透過性 \\
  $\epsilon$ & 誘電率 \\
  $\mu$      & 透過性 \\
  $\epsilon$ & 誘電率 \\
  $\mu$      & 透過性 \\
  $\epsilon$ & 誘電率 \\
  $\mu$      & 透過性 \\
  $\epsilon$ & 誘電率 \\
  $\mu$      & 透過性 \\
\end{longtable}

\end{nomenclature}


% 结尾部分
\backmatter

% 用于盲审的论文需隐去致谢、发表论文、科研成果、简历

\begin{achievements}

  \subsection*{学術論文}

  \begin{bibliolist}{00}
    \item Chen H, Chan C~T. Acoustic cloaking in three dimensions using acoustic metamaterials[J]. Applied Physics Letters, 2007, 91:183518.
    \item Chen H, Wu B~I, Zhang B, et al. Electromagnetic Wave Interactions with a Metamaterial Cloak[J]. Physical Review Letters, 2007, 99(6):63903.
  \end{bibliolist}

  \begin{bibliolist*}{00}
    \item 著者. 学術論文, 2007.
    \item 著者. 国際会議論文, 2006.
  \end{bibliolist*}

  \subsection*{特許}

  \begin{bibliolist}{00}
    \item 発明者,“永久機関”,No. 202510149890.0
  \end{bibliolist}

  \begin{bibliolist*}{00}
    \item 発明者,“永久機関”,No. XXXXXXXXXXXX.X
  \end{bibliolist*}

\end{achievements}


\begin{acknowledgements}
  \jalipsum[9-10]{kusamakura}
\end{acknowledgements}

\begin{digest}
  \lipsum[11]
\end{digest}

\end{document}
